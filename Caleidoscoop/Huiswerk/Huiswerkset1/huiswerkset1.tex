\documentclass{article}
\usepackage[utf8]{inputenc}
\usepackage[dutch]{babel}
\usepackage{amsmath, amssymb, amsfonts, amsthm}
\usepackage[margin=2cm]{geometry}
\usepackage{cancel}
\usepackage{enumitem}

\setlength{\parindent}{0pt}

\newcommand{\f}[2]{\frac{#1}{#2}}
\newcommand{\tx}[1]{\text{#1}}
\newcommand{\cn}[1]{\cancel{#1}}
\newcommand{\ol}[1]{\overline{#1}}

\newcommand{\N}{\mathbb{N}}
\newcommand{\Z}{\mathbb{Z}}
\newcommand{\Q}{\mathbb{Q}}
\newcommand{\R}{\mathbb{R}}

\newcommand{\Rho}{\mathcal{P}}
\newcommand{\q}[1]{#1/_\sim}

\newcommand{\en}{\tx{ en }}
\newcommand{\of}{\tx{ of }}
\newcommand{\geldt}{\tx{ geldt }}
\newcommand{\dan}{\tx{ dan }}
\newcommand{\als}{\tx{ als }}
\newcommand{\dus}{\tx{ dus }}

\begin{document}

{\Large \textbf{Caleidoscoop}}

\bigskip

\textbf{Jasper Vos} \hfill \textbf{Inleverset A} \hfill \today \\
Studentnr: \emph{s2911159}

\rule{\textwidth}{2pt}

\bigskip

\begin{enumerate}[label=\arabic*]
    \item
          \begin{enumerate}[label=\alph*)]
              \item
                    Zij \(a, b, c \in \Q \) en \(\sim\) een equivalentie-relatie waarbij \(a - b \in \Z\).

                    Bewijs dat \(\sim\) een equivalentie-relatie is:
                    \begin{itemize}
                        \item \(\sim\)\emph{ is reflexief:}

                              \(a - a = 0 \en 0 \in \Z \) en dus is \(\sim\) reflexief.
                        \item \(\sim\)\emph{ is symmetrisch:}

                              Als \(a - b \in \Z \dan -1(a - b) = b - a \in \Z\), en dus \(b \sim a\) waaruit volgt dat \(\sim\) symmetrisch is.
                        \item \(\sim\)\emph{ is transitief:}

                              \(a-b \in \Z \en b-c \in \Z\) dan \((a-b) + (b-c) \in \Z \dus a -\cn{b} + \cn{b} - c = a-c \in \Z \)
                              hieruit volgt \(a \sim c\) en dus is \(\sim\) een transitieve relatie.
                    \end{itemize}
                    Voor \(\sim\) geldt dat hij reflexief, symmetrisch en transitief is, en daarmee
                    is \(\sim\) een equivalentie-relatie.

                    Als we nu gaan kijken naar \(\Q/_\sim\),
                    dan kunnen we elke equivalentie-klasse
                    \(\ol{q}\) kunnen schrijven als:
                    \[\ol{q} = \ol{\f{1}{k}} = \{(\f{k(i) + 1}{k}) : i \in \Z\}\]
                    Hieruit volgt dus dat we oneindig equivalentie-klassen hebben want
                    we kunnen een bijectie opstellen vanuit \(f:\Z \rightarrow (0, 1]\) met \(f(k) = \f{1}{k}\), en dus
                    zijn het aantal equivalentie-klassen aftelbaar oneindig.
                    Daarnaast heeft elke equivalentie-klasse oneindig elementen omdat:
                    \[\f{k(i) + 1}{k} = i + \f{1}{k}\]
                    We kunnen dit zien als een strikt stijgende lijn,
                    en dus moet elke equivalentie-klasse oneindig aantal elementen bevatten.
              \item
                    Ik denk niet dat dit kan. Ik stel voor dat het wel kan, en probeer een tegenspraak te herleiden.
                    \begin{proof}
                        Stel dat er een Quotiëntruimte bestaat waarbij $|\q{Q}| = n$, en $|\ol{q}| = m$, waarbij $\ol{q} \in \q{Q}$,
                        We weten dat $(\q{Q})$ partities vormen in \(\Q\).
                        Dit betekent dus dat $\Q$ partities $\ol{q}$ moet vormen waarbij
                        elk element van $\Q$ opgedeeld wordt, echter
                        geldt voor $|\ol{q}| = m$ en $|\q{Q}| = n$, en dus zijn
                        er hoogstens $n\cdot m$ aantal elementen. Dit luidt tot een
                        tegenspraak want $n \cdot m < \infty = |\Q|$.
                    \end{proof}
          \end{enumerate}
    \item
          \begin{enumerate}[label=\alph*)]
              \item \begin{enumerate}
                        \item Bekijk of $X:= \{0\} \cup \{1 - \f{1}{n+2}\}_{n \in \Z_{\geq 0}}$ zowel een infimum en een supremum heeft.
                              \begin{itemize}
                                  \item Bekijk of $X$ een infimum heeft:
                                        \begin{proof}
                                            Claim dat het infimum $i$ bestaat met $i = 0$.
                                            Allereerst moet $0$ een ondergrens zijn.
                                            Bekijk \[x \in \{0\} \cup \{1 - \f{1}{n+2}\}\]
                                            We weten dat de rechterkant van de vereniging minstens
                                            $x = \f{1}{2}$, en hoogstens $1$ benadert.

                                            We zeggen dat $0$ een ondergrens is als
                                            $0\leq x$ voor alle $x \in X$. Dit komt overeen
                                            met het minimum voor $X$ en dus is $0$ een ondergrens,
                                            en een minimum van $X$.

                                            Nu moeten we laten zien dat $0$ de grootste ondergrens is.
                                            Dit is echter waar omdat $0$ ook het minimum van $X$ is en
                                            dus is het infimum $i = 0$.
                                        \end{proof}
                                  \item Bekijk of $X$ een supremum heeft:
                                        \begin{proof}Claim dat het supremum $s$ bestaat waarbij $s = 1$.
                                            Allereerst moet $s$ een bovengrens zijn en dus moet voor
                                            elke $x \in X$ gelden dat $s \geq x$. Bekijk:
                                            \[X = \{0\} \cup \{1 - \f{1}{n+2}\}\]
                                            De rechterkant van de vereniging geeft aan dat het $1$ benadert,
                                            maar nooit $1$ kan worden en dus $1 \geq x$ voor alle $x \in X$.

                                            Nu moeten we nog laten zien dat $s = 1$ de kleinste bovengrens is.
                                            Neem $x = 1 - \f{1}{n+2}$, dan moet gelden $\forall epsilon > 0$ dat:
                                            \[1 - \epsilon < x\]
                                            Waarbij $1$ de gesuggereerde bovengrens $s$ is.
                                            substitueer $x = 1 - \f{1}{n+2}$:
                                            \[1 - \epsilon < 1- \f{1}{n+2}\]
                                            Neem aan dat $\epsilon > \f{1}{n+2}$ dan is er altijd een
                                            $x$ waarvoor epsilon groter is als we $n$ groot genoeg maken.
                                            Hieruit volgt dus dat $s = 1$ de kleinste bovengrens moet zijn.
                                        \end{proof}
                              \end{itemize}
                        \item Bekijk of $Y:= \{\f{n+2}{n+1}: n \in \N\}$, een supremum en infimum heeft.
                              \begin{itemize}
                                  \item Bekijk of $Y$ een infimum heeft:
                                        \begin{proof}
                                            Claim dat het infimum $i = 1$.
                                            Dan moet voor alle $x \in Y$ gelden dat $x \geq i = 1$.
                                            Merk op dat $x \in \f{n+2}{n+1} = 1 + \f{1}{n+1}$, dan
                                            $1 + \f{1}{n+1} > 1$, en dus is $i=1$ een ondergrens.

                                            Nu moeten we nog laten zien dat het de grootste ondergrens is.
                                            Voor alle $\epsilon > 0$, bestaat er een $x \in Y$ zodanig dat:
                                            \[\epsilon + \underbrace{1}_{\tx{Onze claim } i = 1} > x \]
                                            neem $x = 1 + \f{1}{n+1}$, dan:
                                            \[\epsilon + 1 > \f{1}{n+1} + 1\]
                                            We kunnen epsilon willekeurig klein maken en met name $\epsilon > \f{1}{n+1}$,
                                            en dus volgt dat $i = 1$ inderdaad het infimum is.
                                        \end{proof}
                                  \item Bekijk of $Y$ een supremum heeft:
                                        \begin{proof}
                                            Claim dat het supremum $s = 2$.
                                            Bekijk $x \in 1 + \f{1}{n+1}$, we kunnen dit zien als
                                            een strikt dalende functie en dus geldt voor $n=0$, dat
                                            dit de grootste waarde is voor $1 + \f{1}{n+1}$.
                                            En dus $1 + \f{1}{0+1} = 2$. Dit betekent dus dat $2$ het maximum van $Y$ is en dus
                                            automatisch het supremum.


                                        \end{proof}
                              \end{itemize}
                    \end{enumerate}
          \end{enumerate}
\end{enumerate}


\end{document}