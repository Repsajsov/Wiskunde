\documentclass{article}
\usepackage[utf8]{inputenc}
\usepackage[dutch]{babel}
\usepackage{amsmath, amssymb, amsfonts, amsthm}
\usepackage[margin=2cm]{geometry}
\usepackage{cancel}
\usepackage{enumitem}

\setlength{\parindent}{0pt}

\newcommand{\f}[2]{\frac{#1}{#2}}
\newcommand{\tx}[1]{\text{#1}}
\newcommand{\cn}[1]{\cancel{#1}}
\newcommand{\ol}[1]{\overline{#1}}

\newcommand{\N}{\mathbb{N}}
\newcommand{\Z}{\mathbb{Z}}
\newcommand{\Q}{\mathbb{Q}}
\newcommand{\R}{\mathbb{R}}

\newcommand{\Rho}{\mathcal{P}}
\newcommand{\q}[1]{#1/_\sim}

\newcommand{\en}{\tx{ en }}
\newcommand{\of}{\tx{ of }}
\newcommand{\geldt}{\tx{ geldt }}
\newcommand{\dan}{\tx{ dan }}
\newcommand{\als}{\tx{ als }}
\newcommand{\dus}{\tx{ dus }}

\begin{document}

{\Large \textbf{Caleidoscoop}}

\bigskip

\textbf{Jasper Vos} \hfill \textbf{Inleverset A} \hfill \today \\
Studentnr: \emph{s2911159}

\rule{\textwidth}{2pt}

\bigskip

\begin{enumerate}[label=\arabic*]
    \item
          \begin{enumerate}[label=\alph*)]
              \item
                    Zij \(a, b, c \in \Q \) en \(\sim\) een equivalentie-relatie waarbij \(a - b \in \Z\).

                    Bewijs dat \(\sim\) een equivalentie-relatie is:
                    \begin{itemize}
                        \item \(\sim\)\emph{ is reflexief:}

                              \(a - a = 0 \en 0 \in \Z \) en dus is \(\sim\) reflexief.
                        \item \(\sim\)\emph{ is symmetrisch:}

                              Als \(a - b \in \Z \dan -1(a - b) = b - a \in \Z\), en dus \(b \sim a\) waaruit volgt dat \(\sim\) symmetrisch is.
                        \item \(\sim\)\emph{ is transitief:}

                              \(a-b \in \Z \en b-c \in \Z\) dan \((a-b) + (b-c) \in \Z \dus a -\cn{b} + \cn{b} - c = a-c \in \Z \)
                              hieruit volgt \(a \sim c\) en dus is \(\sim\) een transitieve relatie.
                    \end{itemize}
                    Voor \(\sim\) geldt dat hij reflexief, symmetrisch en transitief is, en daarmee
                    is \(\sim\) een equivalentie-relatie.

                    Als we nu gaan kijken naar \(\Q/_\sim\),
                    dan kunnen we elke equivalentie-klasse
                    \(\ol{q}\) kunnen schrijven als:
                    \[\ol{q} = \ol{\f{1}{k}} = \{(\f{k(i) + 1}{k}) : i \in \Z\}\]
                    Hieruit volgt dus dat we oneindig equivalentie-klassen hebben want
                    we kunnen een bijectie opstellen vanuit \(f:\Z \rightarrow (0, 1]\) met \(f(k) = \f{1}{k}\), en dus
                    zijn het aantal equivalentie-klassen aftelbaar oneindig.
                    Daarnaast heeft elke equivalentie-klasse oneindig elementen omdat:
                    \[\f{k(i) + 1}{k} = i + \f{1}{k}\]
                    We kunnen dit zien als een strikt stijgende lijn,
                    en dus moet elke equivalentie-klasse oneindig aantal elementen bevatten.
              \item
                    Ik denk niet dat dit kan. Ik stel voor dat het wel kan, en probeer een tegenspraak te herleiden.
                    \begin{proof}
                        Stel dat er een Quotiëntruimte bestaat waarbij $|\q{Q}| = n$, en $|\ol{q}| = m$, waarbij $\ol{q} \in \q{Q}$,
                        We weten dat $(\q{Q})$ partities vormen in \(\Q\).
                        Dit betekent dus dat $\Q$ partities $\ol{q}$ moet vormen waarbij
                        elk element van $\Q$ opgedeeld wordt, echter
                        geldt voor $|\ol{q}| = m$ en $|\q{Q}| = n$, en dus zijn
                        er hoogstens $n\cdot m$ aantal elementen. Dit luidt tot een
                        tegenspraak want $n \cdot m < \infty = |\Q|$.
                    \end{proof}
          \end{enumerate}
\end{enumerate}


\end{document}