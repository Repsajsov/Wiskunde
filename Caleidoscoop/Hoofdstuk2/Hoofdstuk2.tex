\documentclass{article}
\usepackage[utf8]{inputenc}
\usepackage[dutch]{babel}
\usepackage{amsmath, amssymb, amsfonts, amsthm}
\usepackage[margin=2cm]{geometry}
\usepackage{cancel}
\usepackage{enumitem}

\setlength{\parindent}{0pt}
\setcounter{section}{1}
\begin{document}

\begin{center}
    \Large \textbf{Caleidoscoop Hoofdstuk 2}
\end{center}

\rule{\textwidth}{2pt}

\bigskip

\section{Volledige Inductie}

\subsection{Bewijs met volledige inductie}
\begin{enumerate}[label=\alph*)]
    \item $\forall n \in \mathbb{Z}_{\geq 1} : 3^{2n + 1} + 2^{n-1}$ is een $7$-voud.
        \begin{proof}
            \textbf{Basis:} Voor $n=1$:
            \[
                3^{2(1)+1} + 2^{(1)-1} = 3^3+2^0 = 28 = 7\cdot4
            \]
            
            \textbf{Inductiestap:}
            Stel de uitspraak is waar voor $n=k$, en bewijs voor $n=k+1$. Gebruik $3^{2k-1}+2^{k-1}=7p$ voor een zeker $p \in \mathbb{Z}$.    
            \begin{align*}
                3^{2(k+1)+1} + 2^{(k+1)-1} &= 3^2 \cdot 3^{2k+1} + 2 \cdot 2^{k-1} \\
                &= 9 \cdot (7p-2^{k-1}) + 2 \cdot 2^{k-1} \quad (\text{Vervang }3^{2n+1} = 7p-2^{n-1} )\\
                &= 9 \cdot 7p-9 \cdot 2^{k-1} + 2 \cdot 2^{k-1} \\
                &= 9 \cdot 7p+2^{k-1}(-9+2) \\
                &= 7 \cdot 9p+7 \cdot 2^{k-1} \\
                &= 7(9p+2^{k-1})
            \end{align*}
            De uitspraak geldt dus ook voor $n=k+1$ en daarmee is het bewijs voltooid.
        \end{proof}
    \item Iedere $n \in \mathbb{Z}_{>1}$ is deelbaar door een priemgetal.
        \begin{proof}
            \textbf{Basis:} Voor $n=2$ dan $2 | 2$ en $2$ is priem.
            \textbf{Inductiestap:} Stel de bewering geldt voor alle $2 \leq k < n$, en bewijs voor $n$.
            Als $n$ priem is dan $p = n$, dus $p | n$. Als $n$ niet priem dan nemen we $a, b \in \mathbb{Z}$ zodanig dat $n=ab \wedge 1 < a, b < n$.
            Echter aangezien $a < n$ geldt $\exists p \in \mathbb{P} : p | a$ en uit $  a | n$ volgt $p | n$.
        \end{proof}
    \item Voor alle $x \in \mathbb{R}_{>1}$ geldt: $\forall n \in \mathbb{Z}: (1+x)^n \geq (1+nx)$. 
        \begin{proof}
            \textbf{Basis:} Voor $n=1$, dan $(1+x)^1 = 1+(1)x \implies 1 + x = 1+ x$. 
            \textbf{Inductiestap:} Stel de uitspraak is waar voor $n=k$, bewijs voor $n=k+1$.
            Vermedigvuldig beide kanten met $(1+x)$ dan:
                \begin{align*}
                    (1+x)^k(x+1) &\geq (1 + kx)(x+1) \\
                    &\geq (1 + x + kx^2 + kx) \\
                    &> (1+kx +x) \\  
                    &= (1+(k+1)x)
                \end{align*}
                Hieruit volgt dus $(1+x)^{k+1} \geq 1+(k+1)x$. En dat betekent dat de uitspraak juist is.
        \end{proof}
    \item $\forall n \in \mathbb{Z}_{\geq 4}: n! > 2^n.$
    \begin{proof}
        
    \end{proof}

    
\end{enumerate}



\end{document}