
\documentclass{article}
\usepackage[utf8]{inputenc}
\usepackage[dutch]{babel}
\usepackage{amsmath, amssymb, amsfonts, amsthm}
\usepackage[margin=2cm]{geometry}
\usepackage{cancel}
\usepackage{enumitem}


\setlength{\parindent}{0pt}
\setcounter{section}{3}

\newcommand{\Zg}{\mathbb{Z}_{> 0}}
\newcommand{\Z}{\mathbb{Z}}
\newcommand{\Q}{\mathbb{Q}}
\newcommand{\q}{/_\sim}
\newcommand{\ol}[1]{\overline{#1}}
\newcommand{\tx}[1]{\text{#1}}
\newcommand{\en}{\text{ en }}
\newcommand{\f}[2]{\frac{#1}{#2}}
\newcommand{\lm}[1]{\lim_{n \rightarrow \infty} #1}

\begin{document}

\begin{center}
    \Large \textbf{Caleidoscoop Hoofdstuk 3}
\end{center}

\rule{\textwidth}{2pt}

\bigskip

\section{Reele getallen}

\subsection{}
\begin{enumerate}[label=\alph*)]
    \item
          \begin{proof}
              We kunnen de reële
              getallen zien als equivalentie-klassen van
              de verzameling Cauchy-rijen, waarbij:
              \[a_n \sim b_n \iff \lm{a_n} - \lm{b_n} = 0\]
              Zij $a_n, b_n$ Cauchy-rijen met:
              \[a_n = \{\f{1}{3}, \f{1}{3}, \f{1}{3}, \dots \}\]
              \[b_n = \{0, \f{3}{10}, \f{33}{100}, \dots \}\]

              Laten we nu de stelling bekijken:
              \[\forall \epsilon \in \Q_{>0} \
                  \exists N \in \Zg \tx{ met }
                  i \geq N\ \tx{ zodanig dat }
                  |a_i - b_i| < \epsilon\]
              We moeten dus voor alle $\epsilon>0$
              een bepaalde $N$ vinden
              zodat het verschil tussen de
              rijen arbitrair klein kan worden,
              en dus zeker kleiner dan $\epsilon$.

              Bekijk $|a_i - b_i|$, dan:
              \begin{align*}
                  |a_i - b_i| & = \left|\f{1}{3}
                  - \f{3(10^i - 1)}{9 \cdot 10^i}\right|   \\
                              & = \left|\f{1}{3}
                  - \f{10^i - 1}{3 \cdot 10^i}
                  \right|                                  \\
                              & = \left|\f{10^i -
                  (10^i - 1)}{3\cdot 10^i}\right|          \\
                              & = \left|\f{1}{3\cdot 10^i}
                  \right|                                  \\
                              & =  \f{1}{3 \cdot 10^i}
                  \quad (\tx{Merk op }
                  \f{1}{3\cdot 10^i } > 0 )
              \end{align*}
              Nu laten we dit kleiner worden dan
              $\epsilon$ en dus:
              \begin{align*}
                  \f{1}{3 \cdot 10^i} < \epsilon \\
                  \f{1}{3 \epsilon} < 10^i
              \end{align*}
              Dus neem nu $10^N > 10^i$,
              dan bestaat er dus voor alle $\epsilon > 0$
              een $N$ waarvoor de rijen arbitrair dichtbij
              elkaar liggen.
              Dit betekent dus dat $a_n \sim b_n$
              en dus zit $b_n$ in de
              equivalentie-klassen van $\f{1}{3}$,
              ofwel $0.33333333\dots = \f{1}{3}$.
          \end{proof}
          \subsection{}
          \subsection{}
          Neem een willekeurig gekozen $s in S$ dan geldt $s \leq \sup(S)$
          en ook $s \geq \inf(S)$, Dit betekent dus:
          \[\inf(S) \leq x \leq \sup(S)\]
          En dus is elk element $s \in S$ gelijk aan het supremum en infimum, waaruit volgt:
          \[S = \{sup(S)\}\].
\end{enumerate}

\end{document}