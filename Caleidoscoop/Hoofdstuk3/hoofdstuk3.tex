\documentclass{article}
\usepackage[utf8]{inputenc}
\usepackage[dutch]{babel}
\usepackage{amsmath, amssymb, amsfonts, amsthm}
\usepackage[margin=2cm]{geometry}
\usepackage{cancel}
\usepackage{enumitem}


\setlength{\parindent}{0pt}
\setcounter{section}{2}

\newcommand{\Zg}{\mathbb{Z}_{\geq 0}}
\newcommand{\Z}{\mathbb{Z}}
\newcommand{\q}{/_\sim}
\newcommand{\ol}[1]{\overline{#1}}
\newcommand{\tx}[1]{\text{#1}}

\begin{document}

\begin{center}
	\Large \textbf{Caleidoscoop Hoofdstuk 3}
\end{center}

\rule{\textwidth}{2pt}

\bigskip

\section{Equivalentierelaties}

\subsection*{Opgave 3.1}
\begin{enumerate}[label=\alph*)]
	\item
	      Stel de volgende equivalentierelatie $\mathcal{R}$ op, waarbij $a \sim b \Longleftrightarrow a = b$.
	      \begin{enumerate}[label=\arabic*]
		      \item \emph{Reflexief:} Bekijk $a \mathcal{R} a \implies a = a$
		      \item \emph{Symmetrie:} Bekijk $(a \mathcal{R} b \implies b \mathcal{R} a) \implies a = b \implies b = a$
		      \item \emph{Transitiviteit:} Bekijk $((a \mathcal{R} b \wedge b \mathcal{R} c) \implies a \mathcal{R}c) \implies (a = b \wedge b = c) \implies a = c$
	      \end{enumerate}
	\item
	      Neem de volgende equivalentierelatie $\mathcal{R}$ op, waarbij $a \sim b \Longleftrightarrow a \mod 42 = b \mod 42$.
	      \begin{enumerate}[label=\arabic*]
		      \item \emph{Reflexief:} $a \mod 42 = a \mod 42$
		      \item \emph{Symmetrie:} $(a \mod 42 = b \mod 42) \implies b \mod 42 = a \mod 42$
		      \item \emph{Transitiviteit:} $(a \mod 42 = b \mod 42 \wedge b \mod 42 = c \mod 42) \Longleftrightarrow a \mod 42 = c \mod 42$
	      \end{enumerate}
	\item
	      \begin{proof}X
		      $\mathbf{Aanname:}$ Ik stel dat $A$ een verzameling is waarbij $A \neq \emptyset$, en $A/_\sim = \emptyset$.
		      Aangezien $A \neq \emptyset$ bestaat er een $a \in A$, maar als we een equivalentierelatie hebben, dan volgt vanuit reflexiviteit dat $a \sim a$.
		      Als $a \sim a$ dan moet er een equivalentieklasse $\overline{a} = \{b \in A : b \sim a\}$ bestaan waarbij $a \in \overline{a}$, maar $\overline{a} \in A/_\sim$.
		      Dit is een tegenspraak want we stelde dat $A/_\sim = \emptyset$, en dus kan $A/_\sim$ niet leeg zijn.
	      \end{proof}
\end{enumerate}
\subsection*{Opgave 3.2}
\begin{enumerate}[label=\alph*)]
	\item $X$ wordt gepartioneerd in $X/_\sim$, omdat $|X/_\sim| = \infty =$ zit in elke equivalentieklasse minstens 1 represetant die in $X$ moet liggen. Dit betekent dus dat $|X| \geq |X/_\sim| = \infty$.
	\item
	      \begin{itemize}
		      \item \emph{Geval 1:} $(|X/_\sim|) = (n \wedge |X| = \infty)$:
		            Neem $X=\Z$ met $x \sim y$ als $x \equiv y \mod n$, dan
		            heeft $|X\q|$ precies $n$ elmenten namelijk: $\underbrace{\{ \ol{0}, \ol{1}, \ol{2}, \dots ,\ol{n-1} \}}_{n\ \text{elementen}}$.
		            Hieruit volgt dus dat $|X| = \infty$, en $|X\q| = n$.
		      \item \emph{Geval 2:} $(|X\q|  = n) \wedge (|X| = n)$:
		            Laat $X = \Z_{k}$ en maak een equivalentie relatie waarbij $x \sim y \Longleftrightarrow x = y$.
		            Dan heeft onder reflexiviteit iedere $x \in X$ een equivalentieklasse,
		            namelijk: $X\q = \{\ol{0}, \ol{1}, \dots , \ol{k-1} \}$.
		            Dit betekent dus dat $|X| = k$ en $|X\q| = k$.
	      \end{itemize}
	\item
	      Dan moet $X = \emptyset$,
	      \begin{proof}
		      Stel dat $|X| = n$ en $|X\q| = 0$ dan geldt $\forall x \in
			      X$ dat $x \in \ol{x}$,
		      maar dit kan niet want $|X\q| = 0$, en dus
		      moet $|X| = 0$.
	      \end{proof}
\end{enumerate}
\subsection*{Opgave 3.3}
\begin{enumerate}[label=\alph*)]
	\item
	      \begin{enumerate}[label=\arabic*]
		      \item \emph{Reflexief:} $a-a = 0$ en $0 \in W$,
		            dus reflexief.
		            $\quad (\because 0 \in W)$
		      \item \emph{Symmetrie:} als $a-b \in W$
		            dan $(-1)(a - b) \in W
			            \Longleftrightarrow b - a\in W $
		            $\quad (\because v \in W \implies \lambda v \in W)$
		      \item \emph{Transitiviteit:} $a - b + b - c  =
			            a - c \in W$
		            $\quad (\because v, w \in W \implies v + w \in W)$
	      \end{enumerate}
	\item
	      Neem een $a \in V$ dan geldt voor alle $b \in V$,
	      dat hij equivalent is aan $a$, en dus heeft de
	      $V\q$ slechts één equivalentieklasse.
	\item
	      Neem een willekeurige $\ol{a} \in V\q$,
	      dan moet $\ol{a}$ zichzelf bevatten,
	      want $a \sim b \Leftrightarrow a-b \in \{0\} = W$, en
	      dus $a = b$. Dit betekent dat elk element in $V$
	      een eigen equivalentieklasse heeft met zichzelf.
\end{enumerate}
\subsection*{Opgave 3.4}
Ik heb er geen kunnen vinden als we vanuit $\Z$ dit proberen op te lossen.
Als we vanuit $\Zg$ starten dan kan ik het wel oplossen.

Laat $x \sim y$ met $x, y \in \Zg$ als $x$ en $y$ op hetzelfde niveau $n$ liggen in Pascal's driehoek.
\begin{enumerate}[label=\arabic*]
	\item \emph{Reflexief:} $x$ ligt op hetzelfde niveau als $x$ en dus reflexief.
	\item \emph{Symmetrie:} $x$ en $y$ op hetzelfde niveau betekent $y$ en $x$
	      op hetzelfde niveau en dus geldt symmetrie.
	\item \emph{Transitiviteit:} Als $x$ en $y$ op hetzelfde niveau liggen en $y$ en $z$ ook.
	      Dan moet $x$ ook op hetzelfde niveau liggen als $z$.
\end{enumerate}
Tot slot heb ik nog een idee om alsnog met $\Z$ dit op te lossen.
We moeten eerst een functie $f$ opstellen met $f: \Z \rightarrow \Zg$ waarbij
\[
	f = \begin{cases}
		2x       & \tx{als } x \geq 0 \\
		|2x + 1| & \tx{als } x < 0
	\end{cases}
\]
Dit zorgt ervoor dat we gewoon met $\Zg$ verder kunnen werken en dan
dezelfde equivalentierelatie kunnen opstellen als hierboven.
Ik weet niet of dit goed is...
\subsection*{Opgave 3.5}
\begin{enumerate}[label=\alph*)]
	\item
	      \begin{proof}
		      \begin{enumerate}[label=\arabic*]
			      \item \emph{Reflexief:} Als $p = p$ dan moet $p' = p'$.
			      \item \emph{Symmetrie:} Als $p \sim q$ dan $p' = q'
				            \implies q' = p' \implies q \sim p$
			      \item \emph{Transitiviteit:} Als $p \sim q \wedge q \sim r$
			            dan $p' = q'$ en $q' = r'$ waardoor $p' = r' \implies p \sim r$.
		      \end{enumerate}
	      \end{proof}
	\item
	      Laat $p, q \in \Z[X]$ waarbij $p \neq q$ en $\ol{p} = \ol{q}$,
	      dan geldt $f(\ol{p}) \neq f(\ol{q})$
	      maar dit is een tegenspraak want $\ol{p} = \ol{q}$.
	      Dit betekent dat dit geen functie is, Aangezien
	      voor elk argument hebben we een unieke waarde moeten hebben.
	\item
	      Laat $P = p(x) + c$ en $Q = q(x) + d$ met
	      $P \neq Q$ en $P' = Q'$, dan:
	      \begin{align*}
		      g(\ol{P}) & = p(1) + c - (p(0) + c) \\
		                & = p(1) - p(0)           \\ \\
		      g(\ol{Q}) & = q(1) + c - (q(0) + c) \\
		                & = q(1) - q(0)
	      \end{align*}
	      Het verschil tussen $P$ en $Q$ was de constante.
	      De constante verdwijnt door de functie $g$ en dus is
	      deze wel goed gedefinieerd.
\end{enumerate}
\subsection*{Opgave 3.6}
\subsubsection*{Correct gedefinieerd}
Neem $f: X\q \rightarrow \Zg$ met $f(\ol{x}) := g(x)$, waarbij $g: X -> \Zg$ met $g(x) = x \mod 2$.
Beschouw $X \in \Zg$ en stel de equivalentierelatie $(x \sim y)$ als $(x \mod 6) = (y \mod 6)$ op.
Als we nu willekeurige $x, y \in \ol{x}$ representanten zouden selecteren dan $f(x) = f(y)$, 
aangezien voor elk element in $\ol{a}$ geldt dat het in dezelfde restklasse valt. Ofwel
voor $\ol{0}$ kunnen we elk element schrijven als $2(x) + 0$, \ $\ol{1} \implies 2(x) + 1, \ \ol{2} \implies 2(x) + 2$, etc.
Voorbeeld:
\begin{align*}
	\ol{0} &= \{0, 6, 12, 18, 24, \dots \} \\
	&= \{(2\cdot0), (2\cdot3), (2\cdot4), \dots \} \\ \\ 
	\ol{1} &= \{1, 7, 13, 19, \dots \} \\
	&= \{(2\cdot 0 + 1), (2\cdot3 + 1), (2 \cdot 6 + 1), \dots \} \\ \\
	\ol{\vdots} &= \{ \dots \dots \dots \}
\end{align*}
\subsubsection*{Incorrect gedefinieerd}
Neem $f: X\q \rightarrow \Zg$ met $f(\ol{x}) := g(x)$ waarbij $g: X \rightarrow \Zg$ met $g(x) = x \mod 2$.
Beschouw $X \in \Zg$ en stel de equivalentierelatie $x \sim y$ als $(x \mod 3) = (y \mod 3)$ op. 
Als we nu willekeurige $x, y \in \ol{x}$ representanten zouden selecteren dan $f(x) \neq f(y)$, want  
neem bijvoorbeeld $3, 6 \in \ol{0}$, dan $f(3) = 3 \mod 2 = \boxed{1}$ en $f(6) = 6 \mod 2 = \boxed{0}$. 
Hieruit volgt dus dat $f$ geen functie is aangezien een argument meer dan één waarde kan vertegenwoordigen. 

\subsection*{Opgave 3.7}
Neem $(a,b)\cdot(c,d) = (ac + bd,  $








\end{document}