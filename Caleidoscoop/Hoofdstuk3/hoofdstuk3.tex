\documentclass{article}
\usepackage[utf8]{inputenc}
\usepackage[dutch]{babel}
\usepackage{amsmath, amssymb, amsfonts, amsthm}
\usepackage[margin=2cm]{geometry}
\usepackage{cancel}
\usepackage{enumitem}


\setlength{\parindent}{0pt}
\setcounter{section}{2}

\newcommand{\Zg}{\mathbb{Z}_{\geq 0}}
\newcommand{\Z}{\mathbb{Z}}
\newcommand{\q}{/_\sim}
\newcommand{\ol}[1]{\overline{#1}}

\begin{document}

\begin{center}
	\Large \textbf{Caleidoscoop Hoofdstuk 3}
\end{center}

\rule{\textwidth}{2pt}

\bigskip

\section{Equivalentierelaties}

\subsection*{Opgave 3.1}
\begin{enumerate}[label=\alph*)]
	\item
	      Stel de volgende equivalentierelatie $\mathcal{R}$ op, waarbij $a \sim b \Longleftrightarrow a = b$.
	      \begin{enumerate}[label=\arabic*]
		      \item \emph{Reflexief:} Bekijk $a \mathcal{R} a \implies a = a$
		      \item \emph{Symmetrie:} Bekijk $(a \mathcal{R} b \implies b \mathcal{R} a) \implies a = b \implies b = a$
		      \item \emph{Transitiviteit:} Bekijk $((a \mathcal{R} b \wedge b \mathcal{R} c) \implies a \mathcal{R}c) \implies (a = b \wedge b = c) \implies a = c$
	      \end{enumerate}
	\item
	      Neem de volgende equivalentierelatie $\mathcal{R}$ op, waarbij $a \sim b \Longleftrightarrow a \mod 42 = b \mod 42$.
	      \begin{enumerate}[label=\arabic*]
		      \item \emph{Reflexief:} $a \mod 42 = a \mod 42$
		      \item \emph{Symmetrie:} $(a \mod 42 = b \mod 42) \implies b \mod 42 = a \mod 42$
		      \item \emph{Transitiviteit:} $(a \mod 42 = b \mod 42 \wedge b \mod 42 = c \mod 42) \Longleftrightarrow a \mod 42 = c \mod 42$
	      \end{enumerate}
	\item
	      \begin{proof}X
		      $\mathbf{Aanname:}$ Ik stel dat $A$ een verzameling is waarbij $A \neq \emptyset$, en $A/_\sim = \emptyset$.
		      Aangezien $A \neq \emptyset$ bestaat er een $a \in A$, maar als we een equivalentierelatie hebben, dan volgt vanuit reflexiviteit dat $a \sim a$.
		      Als $a \sim a$ dan moet er een equivalentieklasse $\overline{a} = \{b \in A : b \sim a\}$ bestaan waarbij $a \in \overline{a}$, maar $\overline{a} \in A/_\sim$.
		      Dit is een tegenspraak want we stelde dat $A/_\sim = \emptyset$, en dus kan $A/_\sim$ niet leeg zijn.
	      \end{proof}
\end{enumerate}
\subsection*{Opgave 3.2}
\begin{enumerate}[label=\alph*)]
	\item $X$ wordt gepartioneerd in $X/_\sim$, omdat $|X/_\sim| = \infty =$ zit in elke equivalentieklasse minstens 1 represetant die in $X$ moet liggen. Dit betekent dus dat $|X| \geq |X/_\sim| = \infty$.
	\item
	      \begin{itemize}
		      \item \emph{Geval 1:} $(|X/_\sim|) = (n \wedge |X| = \infty)$:
		            Neem $X=\Z$ met $x \sim y$ als $x \equiv y \mod n$, dan
		            heeft $|X\q|$ precies $n$ elmenten namelijk: $\underbrace{\{ \ol{0}, \ol{1}, \ol{2}, \dots ,\ol{n-1} \}}_{n\ \text{elementen}}$.
		            Hieruit volgt dus dat $|X| = \infty$, en $|X\q| = n$.
		      \item \emph{Geval 2:} $(|X\q|  = n) \wedge (|X| = n)$:
		            Laat $X = \Z_{k}$ en maak een equivalentie relatie waarbij $x \sim y \Longleftrightarrow x = y$.
		            Dan heeft onder reflexiviteit iedere $x \in X$ een equivalentieklasse,
		            namelijk: $X\q = \{\ol{0}, \ol{1}, \dots , \ol{k-1} \}$.
		            Dit betekent dus dat $|X| = k$ en $|X\q| = k$.
	      \end{itemize}
	\item
	      Dan moet $X = \emptyset$,
	      \begin{proof}
		      Stel dat $|X| = n$ en $|X\q| = 0$ dan geldt $\forall x \in
			      X$ dat $x \in \ol{x}$,
		      maar dit kan niet want $|X\q| = 0$, en dus
		      moet $|X| = 0$.
	      \end{proof}
\end{enumerate}
\subsection*{Opgave 3.3}




\end{document}