\documentclass{article}
\usepackage[utf8]{inputenc}
\usepackage[dutch]{babel}
\usepackage{amsmath, amssymb, amsfonts, amsthm}
\usepackage[margin=2cm]{geometry}
\usepackage{cancel}
\usepackage{enumitem}

\setlength{\parindent}{0pt}

\newcommand{\f}[2]{\frac{#1}{#2}}
\newcommand{\s}[1]{\sqrt{#1}}
\renewcommand{\l}{\left}
\renewcommand{\r}{\right}

\begin{document}

{\Large \textbf{Analyse I Huiswerk}}

\bigskip

\textbf{Jasper Vos} \hfill \textbf{Huiswerkset 3} \hfill \today \\
Studentnr: \emph{s2911159}

\rule{\textwidth}{2pt}

\bigskip

\begin{enumerate}[label=\arabic*)]
    \item
          \begin{enumerate}[label=\alph*)]
              \item Voor $x \in (0, \infty)$ geldt:
                    \[f(x) = \f{x}{1+\s{x}}\]
                    De functie is goed gedefinieerd want
                    $1+\s{x} \neq 0 $ als $x \in (0, \infty)$,
                    daarnaast geldt voor iedere $x$ dat het
                    een inwendig punt is.

                    Laten we nu $f'$
                    berekenen op $(0, \infty)$:

                    \begin{align*}
                        f'(x) & = \f{d}{dx}\l(\f{x}{1+\s{x}}\r)                                                   \\
                              & = \l(\f{(1+\s{x})-x(\f{1}{2}x^{-\f{1}{2}})}{(1+\s{x})^2}\r)
                        \quad (\text{Gebruik de Quotiëntregel
                        })                                                                                        \\
                              & = \l(\f{1+\s{x}-\f{1}{2}x^{\f{1}{2}}}{(1+\s{x})^2}\r) \quad (\text{Vereenvoudig}) \\
                              & = \boxed{\l(\f{1+\f{1}{2}\s{x}}{(1+\s{x})^2}\r)} \quad (\text{Vereenvoudig})
                    \end{align*}
                    Dus $f'(x) = \l(\f{1+\f{1}{2}\s{x}}{(1+\s{x})^2}\r)$ voor alle $x \in (0, \infty)$.
              \item Laten we eerst kijken of de functie continu is in $x = -2$.

                    Bekijk of de limieten op $x = -2$ overeenkomen:
                    \begin{itemize}
                        \item Voor als $x \leq -2$ dan $\lim_{x\uparrow-2} (\f{1}{x+1}) = \f{1}{-2 + 1}=\boxed{-1}$
                        \item Voor als $x > -2$ dan $\lim_{x\downarrow -2} (x^2 + x) = (-2)^2 + (-2) = \boxed{2}$
                    \end{itemize}
                    En dus is $f$ niet continu op $x = -2$, en dus zeker niet differentieerbaar.
              \item Om te bepalen of $f$ differentieerbaar is in
                    het punt $x=0$, moet allereest gelden dat $f$ continu is op $x=0$.

                    Bekijk eerst of $x=0$ continu is:
                    \begin{itemize}
                        \item Voor als $x < 0$ dan $\lim_{x \uparrow 0} (x^2 + x) = \boxed{0}$
                        \item Voor als $x \geq 0$ dan $\lim_{x \downarrow 0} {\f{x}{1+\s{x}}} = \f{0}{1 + \s{0}}=\boxed{0}$
                    \end{itemize}
                    Dit betekent dat $f$ continu is op $x=0$, dus aan de voorwaarde continuiteit is voldaan.
                    Nu kijken we of de afgeleiden gelijk zijn rond $x=0$.
                    \begin{itemize}
                        \item Neem $x \geq 0$ hebben we vanuit onderdeel a) namelijk:
                              $f'(x) = \l(\f{1+\f{1}{2}\s{x}}{(1+\s{x})^2}\r)$
                              en dus: \[f'(0)= \l(\f{1+\f{1}{2}\s{0}}{(1+\s{0})^2}\r) =
                                  \boxed{1}\]
                        \item Neem $x < 0 $, dan moeten we eerst even de afgeleide berekenen:
                              $f'(x) = \f{d}{dx}(x^2 + x) = 2x + 1$.
                              Bereken het limiet vanuit wat $0$ benadert vanuit $x < 0$, dan:
                              \[\lim_{x \uparrow 0}(2x + 1) = 2(0) + 1 = \boxed{1}\]
                    \end{itemize}
                    Hieruit volgt dus dat $f$ differentieerbaar is op $x = 0$.
              \item We hebben laten zien in vraag c) dat $f$ differentieerbaar is op $x=0$, en dan moet $f$ per definitie continu zijn.
          \end{enumerate}
    \item Gebruik de kettingregel meerdere keren:
          \begin{align*}
              \phi'(x) & = \f{d}{dx}\l(\sin\l(\sqrt{2 + \cos(x^2 + x + 1)}\r)\r)                                               \\
                       & = \cos\l(\s{2 + \cos(x^2 + x + 1)}\r) \cdot \f{1}{2\s{2 + \cos(x^2 + x + 1)}}
              \cdot \l(-\sin(x^2 + x + 1)\r) \cdot (2x + 1)                                                                    \\
                       & = \boxed{\f{(\cos(\s{2 + \cos(x^2 + x + 1)}))(-\sin(x^2 + x + 1))(2x+1)}{2\s{2 + \cos(x^2 + x + 1)}}}
          \end{align*}
    \item
          \begin{enumerate}[label=\alph*)]
              \item
                    \begin{align*}
                        \lim_{x \rightarrow \f{\pi}{4}}\f{1-\tan^2(x)}{\cos(2x)} & =
                        \lim_{x \rightarrow \f{\pi}{4}}\f{1-\f{\sin^2(x)}{\cos^2(x)}}{\cos(2x)}    \quad (\text{Herschrijf $\tan^2(x) = \f{\sin^2(x)}{\cos^2(x)}$})                                                                      \\
                                                                                 & =\lim_{x \rightarrow \f{\pi}{4}}\f{\f{\cos^2(x) - \sin^2(x)}{\cos^2(x)}}{\cos(2x)}                                                                    \\
                                                                                 & = \lim_{x \rightarrow \f{\pi}{4}}\f{\cos^2(x) - \sin^2(x)}{\cos^2(x)\cos(2x)}                                                                         \\
                                                                                 & = \lim_{x \rightarrow \f{\pi}{4}}\f{\cancel{\cos^2(x) - \sin^2(x)}}{\cos^2(x)\cancel{(\cos^2(x)-\sin^2(x))}} \quad (\cos(2x) = \cos^2(x) - \sin^2(x)) \\
                                                                                 & = \lim_{x \rightarrow \f{\pi}{4}}\f{1}{\cos^2(x)} = \f{1}{\cos^2(\f{\pi}{4})} = \boxed{2}
                    \end{align*}
              \item
                    \begin{align*}
                        \lim_{x \rightarrow 1} \f{\theta -1 + \sin(\theta^2 - 1)}{\theta^2-1} & = \lim_{x \rightarrow 1} \f{\theta - 1}{\theta^2 - 1} + \lim_{x \rightarrow 1}\f{\sin(\theta^2 - 1)}{\theta^2 - 1} \quad (\text{Gebruik rekenregels limieten})                                                               \\
                                                                                              & = \lim_{x \rightarrow 1} \f{\cancel{\theta - 1}}{\cancel{(\theta - 1)}(\theta + 1)} + \lim_{x \rightarrow 1}\f{\sin(\theta^2 -1 )}{\theta^2 - 1} \quad (\text{Gebruik $(a^2-b^2) = (a-b)(a+b)$})                             \\
                                                                                              & = \underbrace{\lim_{x \rightarrow 1} \f{1}{\theta + 1}}_{\text{Hieruit volgt }\f{1}{2}} + \underbrace{\lim_{x \rightarrow 1} \f{\sin(\theta^2 - 1)}{\theta^2 - 1}}_{\text{Standaardlimiet gelijk aan } 1} = \boxed{\f{3}{2}}
                    \end{align*}
          \end{enumerate}
\end{enumerate}

\end{document}