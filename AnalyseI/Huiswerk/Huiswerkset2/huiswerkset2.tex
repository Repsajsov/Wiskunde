\documentclass{article}
\usepackage[utf8]{inputenc}
\usepackage[dutch]{babel}
\usepackage{amsmath, amssymb, amsfonts, amsthm}
\usepackage[margin=2cm]{geometry}
\usepackage{cancel}
\usepackage{enumitem}

\setlength{\parindent}{0pt}

\begin{document}

{\Large \textbf{Analyse I Huiswerk}}

\bigskip

\textbf{Jasper Vos} \hfill \textbf{Huiswerkset 2} \hfill \today \\
Studentnr: \emph{s2911159} 

\rule{\textwidth}{2pt}

\bigskip

\begin{enumerate}
    \item
    \begin{enumerate}[label=\alph*)]
        \item
        \item
            \begin{align*}
                \lim_{x \rightarrow -\infty} \sqrt{x^2+x+4} + x &= \lim_{t \rightarrow \infty} \sqrt{t^2-t+4}-t \quad (\text{Vervang $x$ met $t$ waarbij $t=-x$})\\
                &= \lim_{t \rightarrow \infty} \frac{(\sqrt{t^2-t+4}-t)(\sqrt{t^2-t+4}+t)}{\sqrt{t^2-t+4}+t} \quad (\text{Gebruik de worteltruc})\\ 
                &= \lim_{t \rightarrow \infty} \frac{\cancel{t^2}-t+4-\cancel{t^2}}{\sqrt{t^2-t+4}+t} \\
                &= \lim_{t \rightarrow \infty} \frac{\frac{-t}{t}+\frac{4}{t}}{\sqrt{\frac{t^2}{t^2}-\frac{t}{t^2}+\frac{4}{t^2}} + \frac{t}{t}} \quad (\text{Deel de teller en noemer door $t$})\\
                &= \frac{-1 + 0}{\sqrt{1+0+0}+1} = \boxed{-\frac{1}{2}}
            \end{align*}
        \item
            \begin{align*}
                \lim_{x \rightarrow \infty} \frac{2x^3-2x^2-4x}{(x^2-1)(x-2)} &= \lim_{x \rightarrow \infty} \frac{2x^3-2x^2-4x}{x^3-2x^2-x+2} \quad (\text{Werk de haakjes weg}) \\
                &= \lim_{x \rightarrow \infty} \frac{\frac{2x^3}{x^3}-\frac{2x^2}{x^3}-\frac{4x}{x^3}}{\frac{x^3}{x^3}-\frac{2x^2}{x^3}-\frac{x}{x^3}+\frac{2}{x^3}} \quad (\text{Deel de teller en noemer door $x^3$})\\
                &= \frac{2-0-0}{1-0-0+0} = \boxed{2}
            \end{align*}
        \item
            Als $f(x)$ continu is op $x=-1$ moeten zowel het linker als rechterlimiet gelijk aan elkaar zijn. 
            \begin{align*}
                \lim_{x \uparrow -1} \sqrt{x^2+x+4}+x &= \sqrt{(-1)^2-1+4}-1 \\
                &= \sqrt{4} - 1 = 1 \\
                &\Leftrightarrow \\
                \lim_{x \downarrow -1} \frac{2x^3-2x^2-4x}{(x^2-1)(x-2)} &= \lim_{x \downarrow -1} \frac{2x(x^2-x-2)}{(x-1)(x+1)(x-2)} \\
                &= \lim_{x \downarrow -1} \frac{2x\cancel{(x+1)}\cancel{(x-2)}}{(x-1)\cancel{(x+1)}\cancel{(x-2)}} \\
                &= \frac{2(-1)}{(-1)-1} = 1
            \end{align*}
            Aangezien $\lim_{x \uparrow -1} f(x) = \lim_{x \downarrow -1} f(x)$ moet $f(x)$ continu zijn op het punt $x=-1$.
        \item         
            Gebruik wat we bij de vorige vraag hebben bereikt namelijk $\lim_{x \rightarrow 1} \frac{2x}{x-1}$ en werk verder uit.
            Vul in voor $x \rightarrow 1$ dan: 
            \[\lim_{x \rightarrow 1} \frac{2x}{x-1} = \frac{2}{0}\] 
            Dit betekent dat we een verticale asymptoot hebben op $x=1$.
            Vul nu in voor $x \rightarrow 2$ dan: 
            \[\lim_{x \rightarrow 2} \frac{2x}{x-1} = \frac{4}{1} = 4\] 
        \item
            Gebruik weer het vorige resultaat dan: $\lim_{x \uparrow 2} \frac{2x}{x-1} = 4 \Leftrightarrow \lim_{x \downarrow 2} \frac{2x}{x-1} = 4$. Als we dus $c=4$ nemen is $g$ continu op $x=2$.
            $g$ is continu op $(-\infty, 1) \cup (1, \infty)$ aangezien $1$ geen element is op $(-\infty, 1) \cup (1, \infty)$, en voor alle overige elementen het wel gedefinieerd is.
    \end{enumerate}
    \item 
        Zij $f(x) = x^2, g(x) = x^2 \sin(\frac{1}{x}+50x^2-\cos(x))$ en $h(x)=-x^2$. We weten dat:
        \[x^2 \geq x^2 \sin(\frac{1}{x}+50x^2-\cos(x)) \geq -x^2\] 
        Aangezien $1 \geq \sin(\frac{1}{x}+50x^2-\cos(x)) \geq -1$. Hieruit volgt dat we de tussenwaardestelling kunnen gebruiken.
        \[\lim_{x \rightarrow 0} x^2 = 0 \Leftrightarrow \lim_{x \rightarrow 0} -x^2 = 0\]
        Omdat $\lim_{x \rightarrow 0} f(x)=0 \geq \lim_{x \rightarrow 0} g(x) \geq \lim_{x \rightarrow 0} h(x)=0$, moet $\lim_{x \rightarrow 0} g(x) = x^2\sin(\frac{1}{x}+50x^2-\cos(x)) = 0$.
        
\end{enumerate}

\end{document}