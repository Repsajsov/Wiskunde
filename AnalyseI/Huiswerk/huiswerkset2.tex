\documentclass{article}
\usepackage[utf8]{inputenc}
\usepackage[dutch]{babel}
\usepackage{amsmath, amssymb, amsfonts, amsthm}
\usepackage[margin=2cm]{geometry}
\usepackage{cancel}
\usepackage{enumitem}

\setlength{\parindent}{0pt}

\begin{document}

{\Large \textbf{Analyse I Huiswerk}}

\bigskip

\textbf{Jasper Vos} \hfill \textbf{Huiswerkset 2} \hfill \today \\
Studentnr: \emph{s2911159} 

\rule{\textwidth}{2pt}

\bigskip

\begin{enumerate}
    \item
    \begin{enumerate}[label=\alph*)]
        \item 
        We kijken eerst voor $x \in (-\infty, -1 ]$. 
        Voor dit domein geldt de functie $f(x) = \sqrt{x^2 +x + 4} + x$. 
        De wortelfunctie is geldig als hij groter of gelijk is aan 0, dus $x^2 + x + 4 \geq 0$.
        Aangezien $x^2 + x + 4$ een dalparabool is moet $x^2 + x + 4 > 4 > 0$ zijn.
        Hieruit volgt dat $\sqrt{x^2 + x + 4}$ goed gedefinieerd zijn voor $x \geq -1$, en dus 
        ook voor $\sqrt{x^2 + x + 4} + x$ aangezien $x$ gedefinieerd is voor elke $x$. 

        Nu moeten we nog even controleren of de functie continue is voor 
        elke $x\leq-1$. We hebben al laten zien dat de wortelfunctie goed gedefinieerd is en ook dus continu en de samenstelling van deze wortelfunctie en $x$ is ook continu omdat $x$ ook continu is.
    \end{enumerate}
\end{enumerate}

\end{document}