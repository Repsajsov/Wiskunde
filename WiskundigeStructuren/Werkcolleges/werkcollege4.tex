\documentclass{article}
\usepackage[utf8]{inputenc}
\usepackage[dutch]{babel}
\usepackage{amsmath, amssymb, amsfonts, amsthm}
\usepackage[margin=2cm]{geometry}
\usepackage{cancel}
\usepackage{enumitem}


\setlength{\parindent}{0pt}
\setcounter{section}{2}

\newcommand{\Zg}{\mathbb{Z}_{\geq 0}}
\newcommand{\Z}{\mathbb{Z}}
\newcommand{\q}{/_\sim}
\newcommand{\ol}[1]{\overline{#1}}
\newcommand{\tx}[1]{\text{#1}}

\begin{document}

\begin{center}
    \Large \textbf{Wiskundige structuren}
\end{center}

\rule{\textwidth}{2pt}

\bigskip

\section*{Opgave 1}
\subsection*{b)}
\begin{align*}
\end{align*}
\subsection*{c)}
\begin{align*}
\end{align*}
\section*{Opgave 3}
\begin{align*}
    ac       & = bc                                              \\
    ac - bc  & = bc - bc \quad (\tx{Gebruik additieve inverse} ) \\
    c(a - b) & = 0 \quad (\tx{Distributieve eigenschap})         \\
    \Downarrow                                                   \\
    a        & = b \quad (\because c \neq 0)
\end{align*}


\section*{Opgave 12} MOET P(A) P(OMEGA)
\subsection*{d)}





\end{document}