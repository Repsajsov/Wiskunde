\documentclass{article}
\usepackage[utf8]{inputenc}
\usepackage[dutch]{babel}
\usepackage{amsmath, amssymb, amsfonts, amsthm}
\usepackage[margin=2cm]{geometry}
\usepackage{cancel}
\usepackage{enumitem}
\usepackage{hyperref}
\usepackage{stmaryrd}

\newtheorem{lemma}{Lemma}
\providecommand*{\lemmaautorefname}{Lemma}

\setlength{\parindent}{0pt}
\setcounter{section}{2}

\newcommand{\Zg}{\mathbb{Z}_{\geq 0}}
\newcommand{\Z}{\mathbb{Z}}
\newcommand{\q}{/_\sim}
\newcommand{\ol}[1]{\overline{#1}}
\newcommand{\tx}[1]{\text{#1}}
\newcommand{\zdd}{\text{ zodanig dat }}
\newcommand{\geldt}{\text{ geldt }}
\newcommand{\Zij}{\text{Zij }}
\newcommand{\Laat}{\text{Laat }}
\newcommand{\een}{\text{één }}
\newcommand{\dan}{\text{ dan }}
\newcommand{\en}{\text{ en }}
\newcommand{\of}{\text{ of }}

\begin{document}

\begin{center}
    \Large \textbf{Wiskundige structuren}
\end{center}

\rule{\textwidth}{2pt}

\bigskip

\section*{Opgave 1}
\subsection*{a)}
Te bewijzen:
\[-(a+b) = (-a) + (-b)\]
\begin{proof}
    Allereerst moeten we bewijzen dat er slechts \een $x \in \Z \zdd y \in \Z \geldt x + y = 0$

    Neem $x + y_1 = 0 \en x + y_2 = 0$ dan:
    \begin{align*}
        y_1 & = y_1 + 0 \quad (0 \tx{ is het neutrale element})  \\
            & = y_1 + (x + y_2)                                  \\
            & = (y_1 + x) + y_2 \quad ( \Z \tx{ is associatief}) \\
            & = 0 + y_2                                          \\
            & = y_2
    \end{align*}
    Dit betekent dus dat elk element een unieke inverse heeft, en dus:
    \begin{align*}
        a + (-a) + b + (-b)              & = 0                                                          \\
        a + b + (-a) + (-b)              & = 0 \quad (\Z \tx{ is commutatief})                          \\
        (a + b) + (-a) + (-b)            & = 0                                                          \\
        (a + b) + -(a + b) + (-a) + (-b) & = -(a + b) \quad (-(a + b) \tx{ is de inverse van } (a + b)) \\
        0 + (-a) + (-b)                  & = -(a + b)                                                   \\
        (-a) + (-b)                      & = -(a + b)
    \end{align*}
    Hieruit volgt dus dat:
    \[-(a+b) = (-a) + (-b)\]
\end{proof}

\subsection*{b)}
Te bewijzen:
\[ -0 = 0 \]
\begin{proof}
    Uit opgave a hebben we bewezen dat elke element een unieke inverse heeft, en dus:
    \begin{align*}
        -0 & = -0 + 0  \quad (\Z \tx{ heeft $0$ als neutrale element}) \\
           & = 0 + (-0) \quad (\Z \tx{ is commutatief})                \\
           & = 0
    \end{align*}
\end{proof}

\subsection*{c)}
Te bewijzen:
\[(-ab) = (-a)b \]
\begin{lemma}\label{lem:uniciteit}
    Voor alle $x \in \Z \ \exists ! y \in \Z \zdd x + y = 0$
\end{lemma}
\begin{lemma}\label{lem:nulmult}
    Voor alle $x \in \Z \geldt x(0) = 0$
    \begin{proof}
        Laat $x = y(0)$ dan:
        \begin{align*}
            x + x        & = y(0) + y(0)                                       \\
            x + x        & = y(0 + 0) \quad (\tx{Gebruik distributie in }\Z)   \\
            x + x        & = y(0)                                              \\
            x + x        & = x                                                 \\
            x + x + (-x) & = x + (-x) \quad (\tx{Voeg de inverse van $x$ toe}) \\
            x + 0        & = 0 \quad (\tx{$0$ is het neutrale element})        \\
            x            & = 0
        \end{align*}
        Hieruit volgt dus dat $y(0) = 0$.
    \end{proof}
\end{lemma}
\begin{proof}
    Veronderstel dat:
    \begin{align*}
        ab + (-a)b & = b(a + (-a)) \quad (\tx{Gebruik distributie in }\Z)                        \\
                   & = b(0) \quad (\tx{Vanuit \autoref{lem:uniciteit} en \autoref{lem:nulmult}}) \\
                   & = 0
    \end{align*}
    Door \autoref{lem:uniciteit} bestaat er precies \een additieve inverse en dus
    $(-ab) = (-a)b$.
\end{proof}


\subsection*{d)}
Te bewijzen:
\[(-a)\cdot(-b) = a \cdot b\]
\begin{proof}
    Gebruik \autoref{lem:uniciteit} zodat je kan schrijven:
    \begin{align*}
        (-a)\cdot(-b) + -(a\cdot b) & = (-a)((-b) + b) \quad (\tx{Volgens ditributiviteit van $\Z$}) \\
                                    & = (-a)(0)      \quad (\tx{Gebruik \autoref{lem:nulmult}})      \\
                                    & = 0
    \end{align*}
    Volgens \autoref{lem:uniciteit} bestaat er slechts \een inverse en dus moet
    $(-a)\cdot(-b) = (a\cdot b)$
\end{proof}

\section*{Opgave 2}
\begin{proof}
    Stel dat $1 = 0$, en neem $a \in \Z$ waarbij $a \neq 1 \en a \neq 0$ dan:
    \begin{align*}
        a & = a \cdot 1 \quad (\tx{Het neutrale element in vermedigvuldiging}) \\
          & = a \cdot 0 \quad (\tx{Vanuit \autoref{lem:nulmult}})              \\
          & = 0
    \end{align*}
    Hieruit volgt dus $\forall a \in \Z: a = 0$, en dus $\Z = \{0\}$, maar
    volgens axioma $\Z9$ is $\Z$ niet eindig, en dus tegenspraak. $\lightning$
\end{proof}

\section*{Opgave 3}
\begin{proof}
    Vanuit \autoref{lem:uniciteit} stellen we:
    \[ac -(bc) = 0\]
    \begin{align*}
        ac + -(bc) & = ac + (-b)c \quad (\tx{Vanuit Opgave 1(c)})                                   \\
                   & = c(a + (-b))                                 \quad(\tx{distributie van $\Z$}) \\
    \end{align*}
    Merk op dat $c \neq 0$, en dan is er slechts \een oplossing mogelijk:
    \begin{align*}
        (a + (-b))   & = 0 \\
        a + (-b) + b & = b \\
        a + 0        & = b \\
        a            & = b
    \end{align*}
\end{proof}

\section*{Opgave 12} MOET P(A) P(OMEGA)





\end{document}