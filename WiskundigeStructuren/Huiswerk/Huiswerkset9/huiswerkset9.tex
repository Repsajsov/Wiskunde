\documentclass{article}
\usepackage[utf8]{inputenc}
\usepackage[dutch]{babel}
\usepackage{amsmath, amssymb, amsfonts, amsthm}
\usepackage[margin=2cm]{geometry}
\usepackage{cancel}
\usepackage{enumitem}

\setlength{\parindent}{0pt}

\newtheorem{lemma}{Lemma}

\newcommand{\N}{\mathbb{N}}
\newcommand{\Q}{\mathbb{Q}}
\newcommand{\f}[2]{\frac{#1}{#2}}
\newcommand{\tx}[1]{\text{#1}}
\newcommand{\cn}[1]{\cancel{#1}}
\newcommand{\Z}{\mathbb{Z}}
\newcommand{\R}{\mathbb{R}}
\newcommand{\Rho}{\mathcal{P}}
\newcommand{\en}{\tx{ en }}
\newcommand{\of}{\tx{ of }}
\newcommand{\geldt}{\tx{ geldt }}
\newcommand{\dan}{\tx{ dan }}
\newcommand{\als}{\tx{ als }}
\newcommand{\zdd}{\tx{ zodanig dat }}
\newcommand{\met}{\tx{ met }}

\newcommand{\paren}[1]{\left( #1 \right)}
\newcommand{\set}[1]{\left\{ #1 \right\}}
\newcommand{\bracket}[1]{\left[ #1 \right]}
\newcommand{\abs}[1]{\left| #1 \right|}
\newcommand{\norm}[1]{\left\| #1 \right\|}

\begin{document}

{\Large \textbf{Wiskundige Structuren}}

\bigskip

\textbf{Jasper Vos} \hfill \textbf{Huiswerkset 9} \hfill \today \\
Studentnr: \emph{s2911159}

\rule{\textwidth}{2pt}

\bigskip

\section*{Opgave 1}
\begin{proof}
    Te bewijzen:
    \[ \forall \epsilon > 0 \ \exists \delta > 0 \ \zdd x, y \in D \geldt |x - y| < \delta \implies \abs{\f{1}{x^2} - \f{1}{y^2}} < \epsilon\]
    Uitschrijven geeft:
    \[\abs{\f{1}{x^2} - \f{1}{y^2}} = \abs{\f{x^2 - y^2}{(xy)^2}} = \abs{\f{(x-y)(x+y)}{(xy)^2}}\]
    Merk op dat $|x - y| < \delta$, alleen zitten we met die $\abs{\f{x + y}{(xy)^2}}$, we moeten een $M$ vinden zodanig dat $M \geq \f{x + y}{(xy)^2}$.
    Herschrijf:
    \[\f{x}{(xy)^2} + \f{y}{(xy)^2} = \f{1}{xy^2} + \f{1}{x^2y} \leq 1 + 1 = 2 \tx{, Voor alle } x \in [1, \infty) \]
    We hebben $M = 2$ gevonden en dus:
    \[\abs{\f{(x-y)(x+y)}{(xy)^2}} < M \cdot \delta  = 2\delta\]
    Kies $\delta = \f{\epsilon}{2}$, dan hebben we:
    \[\boxed{\abs{\f{1}{x^2} - \f{1}{y^2}} < 2\delta = \cn{2}\cdot\f{\epsilon}{\cn{2}} = \epsilon}\]
    En dus is $f$ uniform continue op $[0, \infty)$.
\end{proof}

\section*{Opgave 2}
\begin{proof}
    Laat $g(x) = x$ en bekijk $x = 0$ dan:
    \[g(0) = 0 \en f(0) = e^{-\f{5}{2}\sqrt{0}} = e^0 = 1, \tx{ en dus } f(0) > g(0)\]
    Bekijk nu $x = 2$:
    \[g(2) = 2 \en f(2) = e^{-\f{5}{2}\sqrt{2}} = \f{1}{e^{\f{5}{2}\sqrt{2}}} < 1, \tx{ en dus } f(2) < g(2)\]
    Omdat $f$ en $g$ beide continu zijn, en omdat $f(0) > g(0)$ en ook $g(2) > f(2)$ moet er volgens de tussenwaardestelling een $x \in [0, 2]$ bestaan waarvoor:
    \[\boxed{f(x) = g(x) = x}\].
\end{proof}

\section*{Opgave 3}
\begin{proof}
    Merk op dat $f$ een vorm heeft als:
    \[f(x) = a_1x^{n_1} + a_2x^{n_2} + \dots \]
    Waarbij $a_i > 0$, en $n_1 > n_i$ waarvoor $i > 1$. 
    
    We kijken dus alleen naar de eerste term om het gedrag van de functie te bepalen.
    Als $n_1$ even is geldt dus dat $f$ altijd een dalparabool is, en als $n_1$ oneven geldt dat $f$ stijgend is als $x > 0$.
    Dit betekent dus dat $f$ naar boven onbegrensd is. 
    
    Volgens de tussenwaardestelling moet er een $c$ bestaan zodanig dat $f(c) = 0$, voor $n_1$ even geeft dit twee oplossingen en voor $n_1$ oneven geeft dit een enkele oplossing.
\end{proof}

\section*{Opgave 4}
\begin{proof}
    Bewijs vanuit het ongerijmde.

    Laat $f$ continu zijn waarbij $x, y$ bestaan met $f(x) \neq f(y)$.
    Kies nu een irrationaal getal $p$ waarvoor geldt $f(x) < p < f(y)$. Volgens de tussenwaardestelling moet er dan een $c \in [0, 1]$ bestaan waarvoor $f(c) = p$.
    Merk op dat $f(c) \in \Q$ volgens definitie van $f$ echter geldt ook dat $f(c) = p$ en $p \notin \Q$, dus hebben we een tegenspraak.

    Als $f$ continu is moet het dus constant zijn,
    en elke constante functie is per definitie continu.
\end{proof}
\end{document}
