\documentclass{article}
\usepackage[utf8]{inputenc}
\usepackage[dutch]{babel}
\usepackage{amsmath, amssymb, amsfonts, amsthm}
\usepackage[margin=2cm]{geometry}
\usepackage{cancel}
\usepackage{enumitem}

\setlength{\parindent}{0pt}

\newtheorem{lemma}{Lemma}

\newcommand{\N}{\mathbb{N}}
\newcommand{\f}[2]{\frac{#1}{#2}}
\newcommand{\tx}[1]{\text{#1}}
\newcommand{\cn}[1]{\cancel{#1}}
\newcommand{\Z}{\mathbb{Z}}
\newcommand{\R}{\mathbb{R}}
\newcommand{\Rho}{\mathcal{P}}
\newcommand{\en}{\tx{ en }}
\newcommand{\of}{\tx{ of }}
\newcommand{\geldt}{\tx{ geldt }}
\newcommand{\dan}{\tx{ dan }}
\newcommand{\als}{\tx{ als }}
\newcommand{\zdd}{\tx{ zodanig dat }}

\begin{document}

{\Large \textbf{Wiskundige Structuren}}

\bigskip

\textbf{Jasper Vos} \hfill \textbf{Huiswerkset 6} \hfill \today \\
Studentnr: \emph{s2911159}

\rule{\textwidth}{2pt}

\bigskip

\section*{Opgave 1}

\begin{enumerate}[label=\alph*)]
    \item
          \begin{proof}
              ~\\
              \begin{itemize}
                  \item \emph{Basisstap: $n = 1$}

                        Te bewijzen: $a_1 > a_0$. We weten $a_0 = \log_2(12)$, en $a_1 = a_{0 + 1} = \log_2(a_0 +12) = \log_2(\log_2(12) + 12)$.

                        Nu moeten we laten zien dat:
                        \[\log_2(\log_2(12) + 12) > \log_2(12)\]
                        Merk op dat $3 < \log_2(12)$ als we dit gebruiken dan:
                        \[\log_2(\log_2(12) + 12) > \log_2(3 + 12) > \log_2(12)\]
                        We kunnen hier dus aflezen dat $\log_2(15) > \log_2(12)$ omdat $\log_2$ strikt stijgend is, en omdat
                        $\log_2(15) < \log_2(\log_2(12) + 12)$. Via transitiviteit van ordening $>$ geldt:

                        \[\log_2(\log_2(12) + 12) > \log_2(12) \iff \boxed{a_1 > a_0}\]

                  \item \emph{Inductie-hypothese: }

                        We nemen aan dat de stelling voor $n = k$ geldt ofwel dat $a_k > a_{k-1}$.
                        Vanuit de inductie-hypothese bouwen we naar de volgende termen.
                        \begin{align*}
                            a_k              & > a_{k-1}              \\
                            a_k + 12         & > a_{k-1} + 12         \\
                            \log_2(a_k + 12) & > \log_2(a_{k-1} + 12)
                        \end{align*}
                        Merk op dat dit overeen komt met de definitie van $a_{k+1}$ en $a_k$, en dus:
                        \[ \log_2(a_k + 12) > \log_2(a_{k-1} + 12) \iff \boxed{a_{k + 1} > a_k} \]
                        Hieruit volgt dus dat de stelling voor elke $n \in \Z_{\geq 0}$ geldt, en daarmee is $a_n$ strikt stijgend.
              \end{itemize}
          \end{proof}
    \item
          \begin{proof}
              ~\\
              \begin{itemize}
                  \item \emph{Basisstap: $n = 1$}

                        Te bewijzen $a_1 < 4$. We weten dat $a_1 = a_{0 + 1} = \log_2(a_0 + 12) = \log_2(\log_2(12) + 12)$.
                        Merk op dat $a_0 = \log_2(12) < \log_2(16) = 4$, en dus is $a_0 < 4$ waaruit volgt:
                        \[\log_2(a_0 + 12) < \log_2(4 + 12) \iff \boxed{a_1 < 4} \]
                        $4$ is dus een bovengrens van $a_1$.

                  \item \emph{Inductie-hypothese: }

                        We nemen aan dat de stelling voor $n = k$ geldt ofwel dat $a_k < 4$.
                        We gebruiken dezelfde truuk als bij de basisstap.
                        Merk op dat $a_k < 4 = \log_2(16)$, weten we dus dat:
                        \begin{align*}
                            a_{k+1} & = \log_2(a_k + 12)                              \\
                                    & < \log_2(4 + 12) = 4 \iff \boxed{a_{k + 1} < 4}
                        \end{align*}
                        Dus $4$ is een bovengrens van $a_{k+1}$.

                        Hieruit volgt dat voor alle $n \in \Z_{\geq 0}$ dat $4$ een bovengrens is van $a_n$.

              \end{itemize}
          \end{proof}
          \pagebreak
    \item Merk op dat $a_n$ van boven begrensd is omdat voor alle $n$ geldt dat $4$ een bovengrens is (\emph{vraag 1b}) en dat $a_n$ strikt stijgend is (\emph{vraag 1a}).
          Dit betekent dat $a_n$ convergent moet zijn volgens de monotone convergentiestelling.
          Merk op dat $\lim_{n \rightarrow \infty} a_{n+1} = \lim_{n \rightarrow \infty} \log_2(a_n +12)$ als we dit verder uitwerken:
          \begin{align*}
              \lim_{n \rightarrow \infty} a_{n+1} & = \lim_{n \rightarrow \infty} \log_2(a_n +12)     \\
                                                  & = \log_2((\lim_{n \rightarrow \infty} a_n) + 12 )
          \end{align*}
          Laat $\lim_{n \rightarrow \infty} a_{n+1} = L$, en dus ook $L = \lim_{n \rightarrow \infty} a_n$.
          \begin{align*}
              L   & = log_2(L + 12) \\
              2^L & = L + 12
          \end{align*}
          Bekijk de eerste resultaten van $2^L$, en neem $L = 4$.
          \[2^4 = 4 + 12 \iff 16 = 16\]
          Hieruit volgt dat $4$ het limiet is.
\end{enumerate}

\section*{Opgave 2}

\begin{enumerate}[label=\alph*)]
    \item
    \begin{proof}
        ~\\

        Kies een willekeurige $M \in \R$, omdat $a_n$ niet begrensd is
          bestaat er een $N \in \N$ zodanig dat $a_n > M$. Merk ook op dat $a_n$ strikt stijgend is en dus
          voor alle $n \geq N$ geldt dan $a_n \geq a_N$, en dus:
          \[\forall M \in \R \ \exists N \in \N \tx{ met } n \geq N \zdd a_n > M\]
          Dit is de definitie van een divergente rij, en dus $\lim_{n \rightarrow \infty} a_n = \infty$.
    \end{proof}
    \item 
    \begin{proof}
        ~\\

    Neem $r = 1 + h$ met $h > 0$ dan:
    \begin{align*}
        r^n &= (1 + h)^n \\
        &\geq 1 + nh
    \end{align*}
    Laten we nu een willekeurige $M$ nemen, en vind een
    $N$ met voor alle $n \geq N$ zodanig dat
    $1 + nh > M$. Kies \[\boxed{N > \f{M - 1}{h}}\] 
    Laten we het controleren: $r^n \geq 1 + Nh > 1 + \f{M - 1}{h} \cdot h = 1 + M - 1 = M$, en dus moet $r^n$ een divergente rij zijn waarbij $\lim_{n \rightarrow \infty} r^n = \infty$.
    \end{proof}
    \item
    \begin{proof}
        ~\\

        Merk op dat $b_n$ geen divergente rij is want $\lim_{n \rightarrow \infty} b_n \neq \infty$. Dit betekent dat er een $M$ bestaat waarbij $b_n < M$, en hieruit volgt dat $b_n$ begrensd is.
        Omdat $b_n$ stijgend en begrensd is moet volgens de monotone convergentiestelling gelden dat $b_n$ convergent is.
    \end{proof}
\end{enumerate}

\end{document}