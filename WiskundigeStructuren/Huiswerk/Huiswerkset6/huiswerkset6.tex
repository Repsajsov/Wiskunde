\documentclass{article}
\usepackage[utf8]{inputenc}
\usepackage[dutch]{babel}
\usepackage{amsmath, amssymb, amsfonts, amsthm}
\usepackage[margin=2cm]{geometry}
\usepackage{cancel}
\usepackage{enumitem}

\setlength{\parindent}{0pt}

\newtheorem{lemma}{Lemma}

\newcommand{\N}{\mathbb{N}}
\newcommand{\f}[2]{\frac{#1}{#2}}
\newcommand{\tx}[1]{\text{#1}}
\newcommand{\cn}[1]{\cancel{#1}}
\newcommand{\Z}{\mathbb{Z}}
\newcommand{\R}{\mathbb{R}}
\newcommand{\Rho}{\mathcal{P}}
\newcommand{\en}{\tx{ en }}
\newcommand{\of}{\tx{ of }}
\newcommand{\geldt}{\tx{ geldt }}
\newcommand{\dan}{\tx{ dan }}
\newcommand{\als}{\tx{ als }}
\newcommand{\zdd}{\tx{ zodanig dat }}

\begin{document}

{\Large \textbf{Wiskundige Structuren}}

\bigskip

\textbf{Jasper Vos} \hfill \textbf{Huiswerkset 6} \hfill \today \\
Studentnr: \emph{s2911159}

\rule{\textwidth}{2pt}

\bigskip

\section*{Opgave 1}

\begin{enumerate}[label=\alph*)]
    \item
    \begin{proof}
    ~\\
    \begin{itemize}
        \item \emph{Basisstap: $n = 1$}

        Te bewijzen: $a_1 > a_0$. We weten $a_0 = \log_2(12)$, en $a_1 = a_{0 + 1} = \log_2(a_0 +12) = \log_2(\log_2(12) + 12)$.
        
        Nu moeten we laten zien dat: 
        \[\log_2(\log_2(12) + 12) > \log_2(12)\]
        Merk op dat $3 < \log_2(12)$ als we dit gebruiken dan:
        \[\log_2(\log_2(12) + 12) > \log_2(3 + 12) > \log_2(12)\]
        We kunnen hier dus aflezen dat $\log_2(15) > \log_2(12)$ omdat $\log_2$ strikt stijgend is, en omdat
        $\log_2(15) < \log_2(\log_2(12) + 12)$. Via transitiviteit van ordening $>$ geldt:

        \[\log_2(\log_2(12) + 12) > \log_2(12) \iff \boxed{a_1 > a_0}\]

        \item \emph{Inductie-hypothese: }

        We nemen aan dat de stelling voor $n = k$ geldt ofwel dat $a_k > a_{k-1}$.
        Vanuit de inductie-hypothese bouwen we naar de volgende termen. 
        \begin{align*}
            a_k &> a_{k-1} \\
            a_k + 12 &> a_{k-1} + 12 \\
            \log_2(a_k + 12) &> \log_2(a_{k-1} + 12) 
        \end{align*}
        Merk op dat dit overeen komt met de definitie van $a_{k+1}$ en $a_k$, en dus:
        \[ \log_2(a_k + 12) > \log_2(a_{k-1} + 12) \iff \boxed{a_{k + 1} > a_k} \]
        Hieruit volgt dus dat de stelling voor elke $n \in \Z_{\geq 0}$ geldt, en daarmee is $a_n$ strikt stijgend.
    \end{itemize}
    \end{proof}
    \item 
    \begin{proof}
        ~\\
        \begin{itemize}
        \item \emph{Basisstap: $n = 1$}

        Te bewijzen $a_1 < 4$. We weten dat $a_1 = a_{0 + 1} = \log_2(a_0 + 12) = \log_2(\log_2(12) + 12)$.
        Merk op dat $a_0 = \log_2(12) < \log_2(16) = 4$, en dus is $a_0 < 4$ waaruit volgt:
        \[\log_2(a_0 + 12) < \log_2(4 + 12) \iff \boxed{a_1 < 4} \]
        $4$ is dus een bovengrens van $a_1$.

        \item \emph{Inductie-hypothese: }

        We nemen aan dat de stelling voor $n = k$ geldt ofwel dat $a_k < 4$.
        We gebruiken dezelfde truuk als bij de basisstap.
        Merk op dat $a_k < 4 = \log_2(16)$, weten we dus dat: 
        \begin{align*}
        a_{k+1} &= \log_2(a_k + 12) \\ 
        &< \log_2(4 + 12) = 4 \iff \boxed{a_{k + 1} < 4}
        \end{align*}
        Dus $4$ is een bovengrens van $a_{k+1}$.

        Hieruit volgt dat voor alle $n \in \Z_{\geq 0}$ dat $4$ een bovengrens is van $a_n$.

        \end{itemize}
    \end{proof}
    \pagebreak
    \item Merk op dat $a_n$ van boven begrensd is omdat voor alle $n$ geldt dat $4$ een bovengrens is (\emph{vraag 1b}) en dat $a_n$ strikt stijgend is (\emph{vraag 1a}).
    Dit betekent dat $a_n$ convergent moet zijn volgens de monotone convergentiestelling. De limiet van deze rij is dus het $\sup(a_n)$.
    \begin{proof}   
    \emph{Claim: $\sup(a_n) = 4$ }
        \begin{itemize}
            \item Te bewijzen: $4$ is een bovengrens. 

            Vanuit \emph{vraag 1b} geldt dat $4$ een bovengrens is.
            \item Te bewijzen: $4$ is de kleinste bovengrens is:


        \end{itemize}
    \end{proof}

\end{enumerate}

\section*{Opgave 2} 

\begin{enumerate}[label=\alph*)]
    \item 
\end{enumerate}

\end{document}