\documentclass{article}
\usepackage[utf8]{inputenc}
\usepackage[dutch]{babel}
\usepackage{amsmath, amssymb, amsfonts, amsthm}
\usepackage[margin=2cm]{geometry}
\usepackage{cancel}
\usepackage{enumitem}

\setlength{\parindent}{0pt}

\newtheorem{lemma}{Lemma}

\newcommand{\N}{\mathbb{N}}
\newcommand{\f}[2]{\frac{#1}{#2}}
\newcommand{\tx}[1]{\text{#1}}
\newcommand{\cn}[1]{\cancel{#1}}
\newcommand{\Z}{\mathbb{Z}}
\newcommand{\R}{\mathbb{R}}
\newcommand{\Rho}{\mathcal{P}}
\newcommand{\en}{\tx{ en }}
\newcommand{\of}{\tx{ of }}
\newcommand{\geldt}{\tx{ geldt }}
\newcommand{\dan}{\tx{ dan }}
\newcommand{\als}{\tx{ als }}
\newcommand{\zdd}{\tx{ zodanig dat }}

\begin{document}

{\Large \textbf{Wiskundige Structuren}}

\bigskip

\textbf{Jasper Vos} \hfill \textbf{Huiswerkset 7} \hfill \today \\
Studentnr: \emph{s2911159}

\rule{\textwidth}{2pt}

\bigskip

\section*{Opgave 1}

We gebruiken volledige inductie om te bewijzen dat $I$ niet leeg is voor alle $n \in \N$.
\begin{proof}
    ~\\
    \begin{itemize}
        \item \emph{Basisstap: $n=0, \ n = 1$}

              Voor $n=0$ is triviaal want $\mathcal{I}_0 := \bigcap_{0\geq 0} I_0 = I_0$, en $I_0$ is een niet-lege begrensde interval, dus $\mathcal{I}$ niet leeg.

              Voor $n=1$ geldt $\mathcal{I}_1 := \bigcap_{1\geq i \geq0} I_i = I_0 \cap I_1$.

              Laat $x\in I_1$ dan $x\in I_0$ want $I_1 \subset I_0$ en $I_1$ niet-leeg.
              Hieruit volgt dus $x \in I_1 \cap I_0$, en dus $I_1 \cap I_0 \neq \emptyset$ waaruit volgt dat $\mathcal{I}$ niet-leeg is.
        \item \emph{Inductie-hypothese:}

              Neem nu $n=k$ en beschouw $\mathcal{I}_k$ niet-leeg. Laat zien dat voor $n = k+1$, geldt dat $\mathcal{I}_{k+1}$ niet-leeg is.

              Voor $n=k$ geldt $\mathcal{I}_k+1 := \bigcap_{k+1 \geq i \geq 0} = I_{k+1} \cap I_k \cap I_{k-1} \cap \dots \cap I_0 = I_{k+1} \cap \mathcal{I}_k $.

              Laat $x \in I_{k+1}$ dan $x \in I_k$ want $I_{k+1} \subset I_k \subset \mathcal{I}_k$, en dus $x \in I_{k+1} \cap \mathcal{I}_k$. Hieruit volgt dus dat $I_{k+1} \cap \mathcal{I}_k \neq \emptyset$, en dus $\mathcal{I}_{k+1}$ niet-leeg.
        \item \emph{Conclusie:}

              $\mathcal{I}$ is niet-leeg voor $n=0, \ n=1$, en verder voor elke $n = k$ en zijn opvolgers. Hieruit volgt dus dat $\mathcal{I}$ leeg is voor elke $n \in \N$.
    \end{itemize}
\end{proof}
\section*{Opgave 2}
\begin{enumerate}[label=\alph*)]
    \item
          Elke begrensde rij heeft een convergente deelrij (Bolzano-Weierstrass). We kijken dus eerst of de rij begrensd is, merk op:
          \[-1 \leq \sin(\f{\pi}{2}n) \leq 1 \iff |\sin(\f{\pi}{2}n)| \leq 1\]
          Hieruit volgt dus dat $\sin(\f{\pi}{2}n)$, begrensd is.

          Laten we nu een deelrij zoeken, bekijk $\sin(\f{\pi}{2}n)$:
          \[\sin(\f{\pi}{2}n) = (0, 1, 0, -1, 0, 1, \dots)\]

          Neem $\sin(n2\pi) = (0, 0, 0, \dots)$ dan hebben we een constante rij, en constante rijen zijn convergent.
    \item
          Voor alle $M \in \R_{>0}$ geldt dat er een $n > M$ volgens de Archimedische eigenschap is.
          Dit betekent dus dat de rij onbegrensd is, en dus geen convergente deelrijen kan bevatten (Bolzano-Weierstrass).
\end{enumerate}

\end{document}