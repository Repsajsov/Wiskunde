\documentclass{article}
\usepackage[utf8]{inputenc}
\usepackage[dutch]{babel}
\usepackage{amsmath, amssymb, amsfonts, amsthm}
\usepackage[margin=2cm]{geometry}
\usepackage{cancel}
\usepackage{enumitem}

\setlength{\parindent}{0pt}

\newtheorem{lemma}{Lemma}

\newcommand{\N}{\mathbb{N}}
\newcommand{\f}[2]{\frac{#1}{#2}}
\newcommand{\tx}[1]{\text{#1}}
\newcommand{\cn}[1]{\cancel{#1}}
\newcommand{\Z}{\mathbb{Z}}
\newcommand{\R}{\mathbb{R}}
\newcommand{\Rho}{\mathcal{P}}
\newcommand{\en}{\tx{ en }}
\newcommand{\of}{\tx{ of }}
\newcommand{\geldt}{\tx{ geldt }}
\newcommand{\dan}{\tx{ dan }}
\newcommand{\als}{\tx{ als }}
\newcommand{\zdd}{\tx{ zodanig dat }}

\begin{document}

{\Large \textbf{Wiskundige Structuren}}

\bigskip

\textbf{Jasper Vos} \hfill \textbf{Huiswerkset 5} \hfill \today \\
Studentnr: \emph{s2911159}

\rule{\textwidth}{2pt}

\bigskip

\section*{Opgave 1}

\begin{enumerate}[label=\alph*)]
    \item
    \begin{proof}
    Te bewijzen: $a \cdot 0 = 0$:
    \begin{align*}
        a\cdot 0 &= a \cdot 0 + 0 &\quad (0 \tx{ is neutraal voor optelling in } \R)\\
        &= a\cdot 0 + (a + (-a)) &\quad (\forall x \in \R \ \exists (-x) \in \R \zdd x + (-x) = 0) \\
        &= a \cdot 0 + a\cdot (1) + (-a) &\quad (\tx{Optelling is associatief, en $1$ is neutraal met vermenigvuldiging in }\R) \\
        &= a \cdot ( 0 + 1) + (-a) &\quad (\tx{Gebruik distributieve eigenschap in }\R) \\
        &= a \cdot (1) + (-a) &\quad (0\tx{ is neutraal met optelling in } \R) \\
        &= a + (-a) &\quad (1 \tx{ is neutraal met vermenigvuldiging in } \R) \\
        &= 0 &\quad ((-a)\tx{ is de inverse van $a$, en dus } a + (-a) = 0 )
    \end{align*}
    \end{proof}
    \item 
    \begin{proof}
    Te bewijzen: $a \neq 0$ en $b \neq 0$ dan $a \cdot b \neq 0$: 
    
    Bewijs uit het ongerijmde waarbij we
    stellen dat $a \neq 0$ en $b \neq 0$ dan $a \cdot b = 0$:
    \begin{align*}
        ab &= 0 \\
        a^{-1}ab &= a^{-1}0 &\quad (\tx{Vermenigvuldig beide kanten met $a^{-1}$}) \\
        (a^{-1}a)b &= 0  &\quad (\tx{Uit resultaat a) geldt }\forall x \in \R \ x\cdot 0 = 0) \\ 
        (1)b &= 0 &\quad (\forall x \in R \ \exists x^{-1} \in R \zdd x\cdot x^{-1} = 1)\\
        b^{-1}b &= b^{-1}0 &\quad (\tx{Vermenigvuldig beide kanten met $b^{-1}$}) \\
        (b^{-1}b) &= 0 &\quad (\tx{Uit resultaat a) geldt }\forall x \in \R \ x\cdot 0 = 0) \\ 
        1 &= 0 &\quad (\forall x \in R \ \exists x^{-1} \in R \zdd x\cdot x^{-1} = 1)
    \end{align*}
    Tegenspraak want $1 \neq 0$, en hieruit volgt als $a \neq 0$ en $b \neq 0$ dan $a\cdot b \neq 0$.
    \end{proof}
    \item
    \begin{proof}
        Te bewijzen: $(a + b)(a - b) = a^2 - b^2$:
        \begin{lemma}[Identiteit: $-b^2 = b(-b)$]
        \end{lemma}
        \begin{proof}
            Voor elk element bestaat een inverse, veronderstel dat $b^2$ de inverse van $b(-b)$ is, dan:
            \begin{align*}
                b(-b) + b^2 &= b(b + (-b)) &\quad (\tx{Gebruik distributieve eigenschap in } \R) \\
                &= b(0) &\quad (\tx{De inverse van $b$ is $(-b)$, en dus $b + (-b) = 0$}) \\
                &= 0 &\quad (\tx{Uit resultaat a) geldt }\forall x \in \R \ x\cdot 0 = 0) 
            \end{align*}
            En dus geldt dat $-b^2 = b(-b)$.
        \end{proof}
        \begin{align*}
            (a + b)(a-b) &= (a + b)(a+(-b)) &\quad (\tx{Definitie: $a-b = a + (-b)$})\\
            &= a^2 + a(-b) + b(a) + b(-b)  \\
            &= a^2 + a(b + (-b)) + b(-b) &\quad (\tx{Gebruik tweemaal distributieve eigenschap in } \R)\\
            &= a^2 + a(0) + b(-b) &\quad (\tx{De inverse van $b$ is $(-b)$, en dus $b + (-b) = 0$}) \\
            &= a^2 + 0 + (b(-b)) &\quad (\tx{Uit resultaat a) geldt }\forall x \in \R \ x\cdot 0 = 0)\\
            &= a^2 + (b(-b)) &\quad (0 \tx{ is neutraal met optelling in }\R)\\
            &= a^2 + (-b^2)  &\quad (\tx{Gebruik \emph{Lemma 1} waarbij: $-b^2 = b(-b)$}) \\
            &= a^2 - b^2 &\quad (\tx{Definitie: $a-b = a + (-b)$})\\
        \end{align*}
    \end{proof}
\end{enumerate}

\section*{Opgave 2} 

\subsection*{Infimum:}
Veronderstel dat $\inf(A) = 0$:
\begin{proof}
\begin{enumerate}
    \item Te bewijzen $0$ is een ondergrens.

    $a_i = \f{1}{n+1}$, en voor alle $n$ geldt $\f{1}{n+1} > 0$, dus is $0$ een ondergrens van $A$.
    
    \item Te bewijzen $0$ is de grootste ondergrens.

    Gebruik de definitie: $\forall \epsilon > 0$ bestaat er een $a \in A \zdd a > 0 + \epsilon$. 
    \begin{align*}
        a &< 0 + \epsilon \\
    \f{1}{n + 1} &< 0 + \epsilon \\
    \f{1}{n + 1} &< \epsilon \\
    \f{n + 1}{n + 1} &< \epsilon(n + 1) \\
    \f{1}{\epsilon} &< (n + 1) \\
    \f{1}{\epsilon} - 1 &< n
    \end{align*}
    Dus voor elke $\epsilon > 0$ bestaat er een element kleiner dan $\epsilon$.
\end{enumerate}
    Beide voorwaarden gelden en dus: \[\inf(A) = 0\].
\end{proof}

\subsection*{Supremum:}
Veronderstel dat $\sup(A) = 1$:
\begin{proof}
\begin{enumerate}
    \item Te bewijzen $1$ is een bovengrens.

    $a_i = \f{1}{n+1}$, merk op dat $\f{1}{n+1}$ dalend is naarmate $n$ groter wordt.
    Volgens het welordeningsprincipe op $\N$ is $n=0$ het kleinste element en dus $\f{1}{0 + 1} = 1$.
    Hieruit volgt dus dat voor alle $n\in \N $ dat $1 \geq \f{1}{n + 1}$, en dus is $1$ een bovengrens.
    \item Te bewijzen $1$ is de kleinste bovengrens.

    Merk op dat $1 \in A$, en een bovengrens is. Dit betekent dat $1$ de kleinste bovengrens moet zijn. 
\end{enumerate}
Beide voorwaarden zijn voldaan en dus:
    \[\sup(A) = 1\]
\end{proof}
\subsection*{Maximum/minimum:}
$A$ heeft geen minimum want $\inf(A) = 0 \notin A$, maar $A$ heeft wel een maximum omdat $\sup(A) = 1$ en $1 \in A$. 

\section*{Opgave 3}
De definitie van een divergente rij:
\[\forall M > 0 \ \exists N \in \N \zdd \forall n \geq N : a_n > M \]
\begin{proof}
We moeten dus een $N$ vinden die altijd een groter rij-element geeft voor elke $M > 0$.
\begin{align*}
    a_n &= \f{n^2 + \sin(n)}{n + 1} \\
    &\geq \f{n^2 - 1}{n + 1} 
    = \f{\cn{(n+1)}(n-1)}{\cn{n+1}}  
    = n-1 \quad (\tx{Merk op dat $-1 \leq \sin(n) \leq 1$})
\end{align*}
We willen dat $n - 1 > M$, en dus nemen we $n > M + 1$. Kies $N = \lceil M + 1 \rceil$.
Daarmee geldt voor alle $n \geq N$ dat $a_n > M$.  
\end{proof}

\end{document}