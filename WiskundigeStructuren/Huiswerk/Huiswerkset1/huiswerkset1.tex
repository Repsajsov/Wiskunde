\documentclass{article}
\usepackage[utf8]{inputenc}
\usepackage[dutch]{babel}
\usepackage{amsmath, amssymb, amsfonts, amsthm}
\usepackage[margin=2cm]{geometry}
\usepackage{cancel}
\usepackage{enumitem}

\setlength{\parindent}{0pt}

\begin{document}

{\Large \textbf{Wiskundige Structuren Huiswerk}}

\bigskip

\textbf{Jasper Vos} \hfill \textbf{Huiswerkset 1} \hfill \today \\
Studentnr: \emph{s2911159} 

\rule{\textwidth}{2pt}

\bigskip

\section*{Opgave 1}
    \begin{enumerate}[label=\alph*)]
        \item 
            \begin{proof}
                Laten we aannemen dat de vergelijking niet klopt door te stellen:
                \[(A \cap B) \cap (A \setminus B) \neq \emptyset\]
                Neem een $x \in (A \cap B) \cap (A \setminus B)$ en werk verder uit:
                \begin{align*}
                    x \in (A \cap B) \cap (A \setminus B) &\implies (x \in A \wedge x \in B) \wedge (x \in A \wedge x \notin B) \\
                    &\implies x \in A \wedge (x \in B \wedge x \notin B) \\
                    &\implies x \in A \wedge x \in \emptyset \\
                    &\implies A \cap \emptyset = \emptyset
                \end{align*}
                Dit is een tegenspraak dus $(A \cap B) \cap (A \setminus B) = \emptyset$.
            \end{proof}
        \item
            \begin{proof}
                Net zoals bij de vorige vraag stellen we dat de vegelijking niet klopt:
                \[(A \cap B) \cup (A \setminus B) \neq A\]
                \begin{align*}
                    x \in (A \cap B) \cup (A \setminus B) &\implies (x \in A \wedge x \in B) \vee (x \in A \wedge x \notin B) \\
                    &\implies x \in A \wedge (x \in B \vee x \notin B) \quad (p \wedge q) \vee (p \wedge r) \Leftrightarrow p \wedge (q \vee r) \\
                    &\implies x \in A \wedge x \in U \\
                    &\implies A \cap U = A
                \end{align*}
                Dit is een tegenspraak dus $(A \cap B) \cup (A \setminus B) = A$.
            \end{proof}
    \end{enumerate}
\section*{Opgave 2}
    \begin{enumerate}[label=\alph*)]
        \item 
            We kijken eerst of $f$ injectief is:
            \begin{proof}
                Voor alle $n_1, n_2 \in A : f(n_1) = f(n_2) \implies (n_1)=(n_2)$, en dus: 
                \begin{align*}
                    (n_1)^2 &= (n_2)^2 \\ 
                    \sqrt{(n_1)^2} &= \sqrt{(n_2)^2} \\
                    n_1 &= n_2 
                \end{align*}
                Aangezien $n_1, n_2 \geq 0$ is er een unieke $n$ voor een bepaalde $m \in B$, en dus is $f$ injectief.
            \end{proof}
            Nu gaan we kijken of $f$ surjectief is:
            \begin{proof}
                $\forall m \in B \ \exists n \in A$ zodanig dat $f(n) = m$.
                \begin{align*}
                    f(n) &= m \\ 
                    n^2 &= m \\ 
                    n &= \sqrt{m}
                \end{align*}
                Hieruit volgt $f(\sqrt{m}) = (\sqrt{m})^2 = m$, en aangezien elk element $m$ dus bereikt kan worden voor een bepaalde $x$ is $f$ surjectief.
            \end{proof}
            $f$ is dus zowel injectief als surjectief.
        \item
            \begin{proof}
                Eerst bewijzen we of $g$ injectief is:
                \[\forall x_1, x_2 \in A : g(x_1) = g(x_2) \implies x_1 = x_2\]
                Hieruit volgt: 
                \begin{align*}
                    g(x_1) &= g(x_2) \\
                    (x_1)^3 + 5 &= (x_2)^3 + 5 \\ 
                    (x_1)^3 &= (x_2)^3 \\
                    x_1 &= x_2 
                \end{align*}
                $g$ is dus injectief, nu bewijzen we surjectiviteit:
                \[\forall y \in B \ \exists x \in A : g(x) = y\]
                Dus:
                \begin{align*}
                    g(x) &= y \\
                    x^3 + 5 &= y \\
                    x^3 &= y - 5 \\
                    x &= \sqrt[3]{y-5}
                \end{align*}
                Nu gaan we het controleren:
                \[g(\sqrt[3]{y-5})=(\sqrt[3]{y-5})^3 + 5 = y-5 + 5 = y\]
                Dus $g$ is ook surjectief, en aangezien $g$ zowel injectief als surjectief is geldt dat $g$ een bijectie is en een inverse $g^{-1}$ heeft.
                Nu gaan we de inverse berekenen:
                \begin{align*}
                    g(y) &= x \\
                    y^3 + 5 &= x \\  
                    y^3 &= x - 5 \\
                    y &= \sqrt[3]{x-5} = g^{-1}(x)
                \end{align*}

            \end{proof}
    \end{enumerate}
\end{document}