
\documentclass{article}
\usepackage[utf8]{inputenc}
\usepackage[dutch]{babel}
\usepackage{amsmath, amssymb, amsfonts, amsthm}
\usepackage[margin=2cm]{geometry}
\usepackage{cancel}
\usepackage{enumitem}

\setlength{\parindent}{0pt}

\newtheorem{lemma}{Lemma}

\newcommand{\N}{\mathbb{N}}
\newcommand{\Q}{\mathbb{Q}}
\newcommand{\f}[2]{\frac{#1}{#2}}
\newcommand{\tx}[1]{\text{#1}}
\newcommand{\cn}[1]{\cancel{#1}}
\newcommand{\Z}{\mathbb{Z}}
\newcommand{\R}{\mathbb{R}}
\newcommand{\Rho}{\mathcal{P}}
\newcommand{\en}{\tx{ en }}
\newcommand{\of}{\tx{ of }}
\newcommand{\geldt}{\tx{ geldt }}
\newcommand{\dan}{\tx{ dan }}
\newcommand{\als}{\tx{ als }}
\newcommand{\zdd}{\tx{ zodanig dat }}
\newcommand{\met}{\tx{ met }}

\begin{document}

{\Large \textbf{Wiskundige Structuren}}

\bigskip

\textbf{Jasper Vos} \hfill \textbf{Huiswerkset 8} \hfill \today \\
Studentnr: \emph{s2911159}

\rule{\textwidth}{2pt}

\bigskip

\section*{Opgave 1}
\begin{enumerate}[label=\alph*)]
    \item 
\begin{proof}
Om aan te tonen dat $0$ een verdichtingspunt is moeten we kijken of er oneindig aantal elementen rond $0$ zitten.
Kies $\epsilon > 0$, en dan bekijken we het interval $(0, \ \epsilon)$: 
\[0 < \f{1729}{n + 1} < \epsilon\]
Als we dit herschrijven kunnen we een $N$ (met de archimedische eigenschap) vinden waarbij $\f{1729}{N + 1} < \epsilon$, ofwel:
\[
    \f{1729}{n + 1} < \epsilon \iff \f{1729}{\epsilon} < n + 1 \iff \f{1729}{\epsilon} - 1 < n\\
\]
We kiezen dus $N = \lceil\f{1729}{\epsilon} - 1\rceil$, dan geldt dus voor alle $n > N$ dat het kleiner dan epsilon is en dus bestaan er oneindig punten rond $0$, en daarmee is $0
$ een verdichtingspunt.
\end{proof}
\item
\begin{proof}
    Gebruik definitie om te verifiëren dat $1$ het limiet is. Definitie:
    \[ \forall \epsilon > 0 \ \exists \delta > 0 \en \forall x \in D\ \zdd |x-0|=|x| < \delta \implies \left|\f{1}{1+x} - 1\right| < \epsilon\]
    Herschrijf de termen:
\[
        \left|\f{1}{1+x} - 1\right| = \left|\f{1-1-x}{1+x}\right| = \left|\f{-x}{1+x}\right| = \left|\f{x}{1+x}\right| \\
\]
Merk op dat $x > 0$ vanuit \emph{Opgave 1a}, dan:
\[\left|\f{x}{x + 1}\right| = \f{x}{1 + x} < \f{x}{1} = x \leq |x| < \delta\]
Als we dus $\delta = \epsilon$ kiezen dan krijgen we:
\[
        \left|\f{1}{1+x} - 1\right| < \delta = \epsilon \\
\]

De definitie houdt stand als we stellen dat $L = 1$, en dus klopt $\lim_{x \rightarrow 0} f(x) = 1$.
\end{proof}
\item 
\begin{proof}
Laat $\epsilon > 0$, dan moet er een $N \in \N$ bestaan met $\forall n\geq N$ zodanig dat: 
\[\left|\f{1}{1 + \f{1729}{n+1}}- 1\right| < \epsilon\]
Herschrijf de termen:
\[  \left|\f{-1729}{n+1+1729}\right| < \epsilon \iff 1729 < \epsilon (n + 1730) \iff \f{1729 - 1730\epsilon}{\epsilon} < n\]
Kies volgens de archimedische eigenschap $N = \lceil\f{1729-1730\epsilon}{\epsilon}\rceil$, dan geldt voor alle $n \geq N$ dat je arbitrair dichtbij $1$ kan komen en dus is de rij convergent en het limiet $1$.
\end{proof}
\end{enumerate}
\section*{Opgave 2}
\begin{proof}
    
Gebruik definitie \emph{i)} uit het boek:
\[\forall \epsilon > 0 \ \exists \delta > 0 \met \forall x \in \R \zdd |x-c| < \delta \implies |f(x) - f(0)| < \epsilon \]
Herschrijf:
\[\left|x^4\cdot\cos(\f{1}{x})-f(0)\right| = \left|x^4\cdot\cos(\f{1}{x})\right| \leq |x^4| = |x|^4 < \delta^4\]
Kies $\delta = \sqrt[4]{\epsilon}$, dan:
\[\left|x^4\cdot\cos(\f{1}{x})-f(0)\right| < \delta^4 = (\sqrt[4]{\epsilon })^4 = \epsilon\]
Hieruit volgt dus dat voor alle $\epsilon$ een $\delta$ kunnen vinden en daarmee is de functie continue op $x=0$.
\end{proof}
\section*{Opgave 3}
Voor alle $x, y \in \R$, geldt dat er een $q \in \Q$ bestaat zodanig dat $x < q< y$. 
Dit betekent dus dat wel altijd een gat tussen $0$ en $1$ hebben en dus kan dit nooit continue zijn.
Om dit verder formeel te laten zien kunnen we een $\epsilon =\f{1}{2}$ kiezen, en gevallen afgaan.

\begin{itemize}
    \item Stel dat $c \in \R \setminus \Q$, dan bestaat er in het interval $(c-\delta , c + \delta)$ een $x \in Q$ (omdat we tussen elke twee reële getallen een rationaal getal kunnen proppen) en dus:
    \[ |1-0| = 1 \nless \f{1}{2}\] 
    \item Stel dat $c \in \Q$, dan bestaat vice versa ook een reëel getal tussen twee rationale getallen en dus:
    \[|0-1| = 1\nless \f{1}{2}\]

\end{itemize}

\section*{Opgave 4} 
Ga per term af of hij continue is dan is de som van alle continue termen ook continue.
\begin{itemize}
    \item 
Merk op dat $|x|$ continue is, en dus is $|x||x||x| = |x|^3= |x^3|$ ook continue. 
\item
We kunnen $\f{1}{1 + x^2}$ als twee continue functies zien namelijk de constante functie $f(x) = 1 \en g(x) = 1 + x^2$. Constante functies zijn altijd continue en $g(x)$ is ook continue want het is een polynoom.
Als twee functies continue zijn dan is $\f{f(x)}{g(x)}$ ook continue als $g(x) \neq 0$, en dit geldt want $g(x) > 0$ voor alle $x$. 
\item De laatste term $9x^8$ is een polynoom en dus continue.
\end{itemize}
Hierbij zijn alle termen continue en dus is de som ook continue.
\end{document}
