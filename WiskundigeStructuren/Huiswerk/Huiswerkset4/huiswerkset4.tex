\documentclass{article}
\usepackage[utf8]{inputenc}
\usepackage[dutch]{babel}
\usepackage{amsmath, amssymb, amsfonts, amsthm}
\usepackage[margin=2cm]{geometry}
\usepackage{cancel}
\usepackage{enumitem}

\setlength{\parindent}{0pt}

\begin{document}

{\Large \textbf{Wiskundige Structuren}}

\bigskip

\textbf{Jasper Vos} \hfill \textbf{Huiswerkset 4} \hfill \today \\
Studentnr: \emph{s2911159}

\rule{\textwidth}{2pt}

\bigskip

\newtheorem{lemma}{Lemma}

\newcommand{\N}{\mathbb{N}}
\newcommand{\f}[2]{\frac{#1}{#2}}
\newcommand{\tx}[1]{\text{#1}}
\newcommand{\cn}[1]{\cancel{#1}}
\newcommand{\Z}{\mathbb{Z}}
\newcommand{\R}{\mathbb{R}}
\newcommand{\Rho}{\mathcal{P}}
\newcommand{\en}{\tx{ en }}
\newcommand{\of}{\tx{ of }}
\newcommand{\geldt}{\tx{ geldt }}
\newcommand{\dan}{\tx{ dan }}
\newcommand{\als}{\tx{ als }}
\newcommand{\zdd}{\tx{ zodanig dat }}

\section*{Opgave 1}
\begin{enumerate}[label=\alph*)]
    \item We hebben eerst een hulpstelling nodig voor het unieke inverse van elk element in $\Z$.
    \begin{lemma}[Unieke inverse]
        \(\forall x \in \Z \ \exists ! y \in \Z \zdd x + y = 0 \)
        \begin{proof}
            Neem \(x, y, y' \in Z \) en laat \( x + y = 0 \en x + y' = 0\) dan:
            \begin{align*}
                x + y &= x + y' \\
                y &= y' \quad (\tx{Schrapwet})
            \end{align*}
            Hieruit volgt dus dat er een unieke inverse is.
        \end{proof}
    \end{lemma}
    Nu beginnen we het bewijs waarom $(-1)a = -a$:
    \begin{align*}
        (-1)a &= 0 + (-1)a \quad (0 \tx{ is neutraal in optelling})\\
        &= a + (-a) + (-1)a \quad (a + (-a) = 0 \ \tx{lemma unieke inverse})\\ 
        &= a + (-1)a + (-a) \quad (\tx{Optelling is commutatief}) \\
        &= (1)a + (-1)a + (-a) \quad (1 \tx{ is neutraal in vermenigvuldiging}) \\
        &= (1 + (-1))a + (-a) \quad (\tx{Ditstributieve eigenschap}) \\
        &= (0)a + (-a) \\
        &= 0 + (-a) \\
        &= -a
    \end{align*}

    \item
    \item
\end{enumerate}

\section*{Opgave 2}
\begin{enumerate}[label=\alph*)]
    \item 
    \item
    \item
\end{enumerate}

\end{document}