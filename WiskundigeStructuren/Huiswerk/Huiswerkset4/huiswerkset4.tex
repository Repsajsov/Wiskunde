\documentclass{article}
\usepackage[utf8]{inputenc}
\usepackage[dutch]{babel}
\usepackage{amsmath, amssymb, amsfonts, amsthm}
\usepackage[margin=2cm]{geometry}
\usepackage{cancel}
\usepackage{enumitem}

\setlength{\parindent}{0pt}

\begin{document}

{\Large \textbf{Wiskundige Structuren}}

\bigskip

\textbf{Jasper Vos} \hfill \textbf{Huiswerkset 4} \hfill \today \\
Studentnr: \emph{s2911159}

\rule{\textwidth}{2pt}

\bigskip

\newtheorem{lemma}{Lemma}

\newcommand{\N}{\mathbb{N}}
\newcommand{\f}[2]{\frac{#1}{#2}}
\newcommand{\tx}[1]{\text{#1}}
\newcommand{\cn}[1]{\cancel{#1}}
\newcommand{\Z}{\mathbb{Z}}
\newcommand{\R}{\mathbb{R}}
\newcommand{\Rho}{\mathcal{P}}
\newcommand{\en}{\tx{ en }}
\newcommand{\of}{\tx{ of }}
\newcommand{\geldt}{\tx{ geldt }}
\newcommand{\dan}{\tx{ dan }}
\newcommand{\als}{\tx{ als }}
\newcommand{\zdd}{\tx{ zodanig dat }}

\section*{Opgave 1}
\begin{enumerate}[label=\alph*)]
    \item We hebben eerst een aantal hulpstellingen nodig.
          \begin{lemma}[Schrapwet]\label{lem:schrapwet}
              Zij $x, y, z \in \Z$ dan geldt $x + z = y + z \implies x = y$.
              \begin{proof}
                  Neem $x, y, z \in \Z$ dan:
                  \begin{align*}
                      x + z        & = y + z                                       \\
                      x + z + (-z) & = y + z + (-z) \quad (\tx{Additieve inverse}) \\
                      x + 0        & = y + 0 \quad (\tx{a + (-a) = 0})             \\
                      x            & = y \quad (0 \tx{ is neutraal in optelling})
                  \end{align*}
              \end{proof}

          \end{lemma}
          \begin{lemma}[Unieke inverse]\label{lem:unieke-inverse}
              \(\forall x \in \Z \ \exists ! y \in \Z \zdd x + y = 0 \).
              \begin{proof}
                  Neem \(x, y, y' \in Z \) en laat \( x + y = 0 \en x + y' = 0\) dan:
                  \begin{align*}
                      x + y & = x + y'                                        \\
                      y     & = y' \quad (\tx{zie Lemma \ref{lem:schrapwet}})
                  \end{align*}
                  Hieruit volgt dus dat er een unieke inverse is.
              \end{proof}
          \end{lemma}
          \begin{lemma}[vermenigvuldiging met $0$]\label{lem:vermenigvuldiging-met-nul}
              \(\forall x \in \Z \ x(0) = 0 \).
              \begin{proof}
                  Neem $x \in \Z$ dan:
                  \begin{align*}
                      x(0) & = x(0) + 0 \quad (0 \tx{ is neutraal in optelling})              \\
                           & = x(0) + (x + (-x)) \quad (\tx{Merk op } x + (-x) = 0)           \\
                           & = x(0) + x + (-x) \quad (\tx{Associativiteit in $\Z$})           \\
                           & = x(0 + 1) + (-x) \quad (\tx{Ditstributieve eigenschap in $\Z$}) \\
                           & = x(1) + (-x) \quad (0 \tx{ is neutraal in optelling})           \\
                           & = x + (-x) \quad (1 \tx{ is neutraal in vermenigvuldiging})      \\
                           & = 0 \quad (\tx{Merk op } x + (-x) = 0)
                  \end{align*}
              \end{proof}
          \end{lemma}
          Nu beginnen we het bewijs waarom $(-1)a = -a$:
          \begin{align*}
              (-1)a & = 0 + (-1)a \quad (0 \tx{ is neutraal in optelling})                              \\
                    & = a + (-a) + (-1)a \quad (a + (-a) = 0 \ \tx{zie Lemma \ref{lem:unieke-inverse}}) \\
                    & = a + (-1)a + (-a) \quad (\tx{Optelling is commutatief})                          \\
                    & = (1)a + (-1)a + (-a) \quad (1 \tx{ is neutraal in vermenigvuldiging})            \\
                    & = (1 + (-1))a + (-a) \quad (\tx{Ditstributieve eigenschap})                       \\
                    & = (0)a + (-a) \quad (\tx{Merk op }1 + (-1) = 0)                                   \\
                    & = 0 + (-a)     \quad (\tx{zie Lemma \ref{lem:vermenigvuldiging-met-nul}})         \\
                    & = -a
          \end{align*}

    \item
          We moeten eigenlijk laten zien wat het inverse is van $0$.
          Gebruik Lemma \ref{lem:unieke-inverse} waarbij we zeggen dat elk element een unieke inverse heeft.
          Stel voor $0 + a = 0$ dan:
          \begin{align*}
              0 + a & = 0                                          \\
              a     & = 0 \quad (0 \tx{ is neutraal in optelling}) \\
          \end{align*}
          Hieruit volgt dus dat $0$ een inverse heeft namelijk zichzelf, en dus $0 = -0$.
    \item
          Bewijs met het ongerijmde.

          Stel $0 = 1$ en gebruik het feit dat elk element
          een unieke inverse heeft (zie Lemma \ref{lem:unieke-inverse}). Bekijk de linker en rechterkant van de vergelijking en bepaal hun inverse.

          Voor $0$:
          \[0 + 0 = 0 \tx{ en dus is } 0 \tx{ de inverse.}\]
          Voor $1$:
          \[1 + (-1) = 0 \tx{ en dus is } (-1) \tx{ de inverse.}\]

          Echter omdat elk element in $\Z$ een unieke inverse heeft is dit een tegenspraak,
          en dus $1 \neq 0$.
\end{enumerate}

\pagebreak
\section*{Opgave 2}
\begin{enumerate}[label=\alph*)]
    \item
          We moeten laten zien dat $\sim_n$ reflexief, symmetrisch en transitief is.
          \begin{enumerate}[label=\arabic*.]
              \item \emph{Reflexiviteit:} Te bewijzen $x \sim_n x$.

                    Zij $x \sim_n x$ dan:
                    \[x \sim_n n \iff (x + x = n + 1) \vee (x = x)\]
                    Voor alle $x$ geldt dat $x = x$ dus is $\sim_n$ reflexief.
              \item \emph{Symmetrie:} Te bewijzen $x \sim_n y$ dan $y \sim_n x$.

                    Beredeneer vanuit $x \sim_n y$ dan:
                    \begin{align*}
                        x  \sim_n y & \iff x + y = n + 1 \vee x = y                                      \\
                                    & \iff y + x = n + 1 \vee y = x \quad (\tx{Gebruik commutativiteit}) \\
                                    & \iff y \sim_n x
                    \end{align*}
                    Hieruit volgt dat $\sim_n$ symmetrisch is.
              \item \emph{Transitiviteit:}
                    Te bewijzen als $x \sim_n y$ en $y \sim_n z$ dan $x \sim_n z$.
                    \[x \sim_n y \iff (x + y = n + 1) \vee (x = y)\]
                    \begin{center}
                        en
                    \end{center}
                    \[y \sim_n z \iff (y + z = n + 1) \vee (y = z)\]
                    \begin{center}
                        Gebruik substitutie
                    \end{center}
                    \begin{align*}
                        (x + y = y + z)           & \vee (x = y = z)                            \\
                        (x + \cn{y} = \cn{y} + z) & \vee (x = z) \quad (\tx{Gebruik Schrapwet}) \\
                        (x + z)                   & \vee (x = z) \iff x \sim_n z
                    \end{align*}
                    Dan volgt dat $\sim_n$ transitief is.
          \end{enumerate}
          Vanwege reflexiviteit, symmetrie en transitiviteit geldt dat $\sim_n$ een equivalentie-relatie is.
    \item
    \item
          \begin{align*}
              1 + 2 + 3 + \dots + n \\
              n + (n-1) + (n-2) + \dots + 1
          \end{align*}
          Tel beide rijen op dan:
          \[\underbrace{(n+1) + (n+1) + \dots + (n+1)}_{n \tx{ termen}} = n(n+1)\]
          Delen door twee omdat we beide rijen opgeteld hebben dus:
          \[\boxed{\sum_{i=1}^{n} i = \f{n(n+1)}{2}}\]
\end{enumerate}

\end{document}