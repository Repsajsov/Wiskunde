\documentclass{article}
\usepackage[utf8]{inputenc}
\usepackage[dutch]{babel}
\usepackage{amsmath, amssymb, amsfonts, amsthm}
\usepackage[margin=2cm]{geometry}
\usepackage{cancel}
\usepackage{enumitem}

\setlength{\parindent}{0pt}

\begin{document}

{\Large \textbf{Wiskundige Structuren}}

\bigskip

\textbf{Jasper Vos} \hfill \textbf{Huiswerkset 4} \hfill \today \\
Studentnr: \emph{s2911159}

\rule{\textwidth}{2pt}

\bigskip

\newtheorem{lemma}{Lemma}

\newcommand{\N}{\mathbb{N}}
\newcommand{\f}[2]{\frac{#1}{#2}}
\newcommand{\tx}[1]{\text{#1}}
\newcommand{\cn}[1]{\cancel{#1}}
\newcommand{\Z}{\mathbb{Z}}
\newcommand{\R}{\mathbb{R}}
\newcommand{\Rho}{\mathcal{P}}
\newcommand{\en}{\tx{ en }}
\newcommand{\of}{\tx{ of }}
\newcommand{\geldt}{\tx{ geldt }}
\newcommand{\dan}{\tx{ dan }}
\newcommand{\als}{\tx{ als }}
\newcommand{\zdd}{\tx{ zodanig dat }}

\section*{Opgave 1}
\begin{enumerate}[label=\alph*)]
    \item We hebben eerst een hulpstelling nodig voor het unieke inverse van elk element in $\Z$.
          \begin{lemma}[Unieke inverse]
              \(\forall x \in \Z \ \exists ! y \in \Z \zdd x + y = 0 \)
              \begin{proof}
                  Neem \(x, y, y' \in Z \) en laat \( x + y = 0 \en x + y' = 0\) dan:
                  \begin{align*}
                      x + y & = x + y'                    \\
                      y     & = y' \quad (\tx{Schrapwet})
                  \end{align*}
                  Hieruit volgt dus dat er een unieke inverse is.
              \end{proof}
          \end{lemma}
          Nu beginnen we het bewijs waarom $(-1)a = -a$:
          \begin{align*}
              (-1)a & = 0 + (-1)a \quad (0 \tx{ is neutraal in optelling})                   \\
                    & = a + (-a) + (-1)a \quad (a + (-a) = 0 \ \tx{lemma unieke inverse})    \\
                    & = a + (-1)a + (-a) \quad (\tx{Optelling is commutatief})               \\
                    & = (1)a + (-1)a + (-a) \quad (1 \tx{ is neutraal in vermenigvuldiging}) \\
                    & = (1 + (-1))a + (-a) \quad (\tx{Ditstributieve eigenschap})            \\
                    & = (0)a + (-a)                                                          \\
                    & = 0 + (-a)                                                             \\
                    & = -a
          \end{align*}

    \item
          We moeten eigenlijk laten zien wat het inverse is van $0$.
          Gebruik lemma waarbij we dus weten dat unieke inverse is.
          Stel voor $0 + a = 0$ dan:
          \begin{align*}
              0 + a & = 0 \\
              a     & = 0 \\
          \end{align*}
          Dus $0$ is het inverse van $0$ en dus $0 = -0$.
    \item
          Bewijs met het ongerijmde, stel $0 = 1$ dan en gebruikt het feit dat elk element
          een unieke inverse heeft. Bekijk linkerkant $ 0 + 0 = 0$ en rechterkant $1 + (-1) = 0$.
          maar $0 \neq -1$ dit is een tegenspraak want als 0 = 1 dan zouden ze een unieke inverse moeten hebben.
\end{enumerate}

\pagebreak
\section*{Opgave 2}
\begin{enumerate}[label=\alph*)]
    \item
          \begin{enumerate}[label=\arabic*.]
              \item \emph{Reflexiviteit:} Te bewijzen $x \sim_n x$.

                    Zij $x \sim_n x$ dan:
                    \[x \sim_n n \iff (x + x = n + 1) \vee (x = x)\]
                    Voor alle $x$ geldt dat $x = x$ dus is $\sim_n$ reflexief.
              \item \emph{Symmetrie:} Te bewijzen $x \sim_n y$ dan $y \sim_n x$.

                    Beredeneer vanuit $x \sim_n y$ dan:
                    \begin{align*}
                        x  \sim_n y & \iff x + y = n + 1 \vee x = y                                      \\
                                    & \iff y + x = n + 1 \vee y = x \quad (\tx{Gebruik commutativiteit}) \\
                                    & \iff y \sim_n x
                    \end{align*}
                    Hieruit volgt dat $\sim_n$ symmetrisch is.
              \item \emph{Transitiviteit:}
                    Te bewijzen als $x \sim_n y$ en $y \sim_n z$ dan $x \sim_n z$.
                    \[x \sim_n y \iff (x + y = n + 1) \vee (x = y)\]
                    \begin{center}
                        en
                    \end{center}
                    \[y \sim_n z \iff (y + z = n + 1) \vee (y = z)\]
                    \begin{center}
                        Gebruik substitutie
                    \end{center}
                    \begin{align*}
                        (x + y = y + z)           & \vee (x = y = z)                            \\
                        (x + \cn{y} = \cn{y} + z) & \vee (x = z) \quad (\tx{Gebruik Schrapwet}) \\
                        (x + z)                   & \vee (x = z) \iff x \sim_n z
                    \end{align*}
                    Dan volgt dat $\sim_n$ transitief is.
          \end{enumerate}
          Vanwege reflexiviteit, symmetrie en transitiviteit geldt dat $\sim_n$ een equivalentie-relatie is.

    \item
    \item
\end{enumerate}

\end{document}