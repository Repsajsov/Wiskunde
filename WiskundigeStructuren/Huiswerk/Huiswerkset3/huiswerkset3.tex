\documentclass{article}
\usepackage[utf8]{inputenc}
\usepackage[dutch]{babel}
\usepackage{amsmath, amssymb, amsfonts, amsthm}
\usepackage[margin=2cm]{geometry}
\usepackage{cancel}
\usepackage{enumitem}

\setlength{\parindent}{0pt}

\begin{document}

{\Large \textbf{Wiskundige Structuren Huiswerk}}

\bigskip

\textbf{Jasper Vos} \hfill \textbf{Huiswerkset 3} \hfill \today \\
Studentnr: \emph{s2911159}

\rule{\textwidth}{2pt}

\bigskip

\newcommand{\N}{\mathbb{N}}
\newcommand{\f}[2]{\frac{#1}{#2}}
\newcommand{\tx}[1]{\text{#1}}
\newcommand{\cn}[1]{\cancel{#1}}
\newcommand{\R}{\mathbb{R}}
\newcommand{\Rho}{\mathcal{P}}
\newcommand{\en}{\tx{ en }}
\newcommand{\of}{\tx{ of }}
\newcommand{\geldt}{\tx{ geldt }}
\newcommand{\dan}{\tx{ dan }}
\newcommand{\als}{\tx{ als }}

\section*{Opgave 1}
\begin{proof}
	We bewijzen voor zowel $n = 0 \en n = 1$,
	omdat er vaak dubbelzinnigheid is over 
	$0 \in \N \of 0 \notin \N$.
	\begin{enumerate}
		\item \emph{Basisstap: $n = 0, \ n = 1$}

		      Neem $|A| = |\emptyset|$ = 0,
		      dan en slechts dan als
		      $|\Rho(A)| |\{\emptyset\}| = 2^0 = 1$.
		      Dus de uitspraak geldt voor $n = 0$.


		      Neem $|A| = |\{a\}| = 1$
		      dan en slechts dan als
		      $|\Rho(A)| = |\{\emptyset, \{a\}\}| = 2^1 = 2$.
		      Dus de uistpraak geldt voor $n = 1$.
		\item \emph{Inductiehypothese:}

		      Neem aan dat de stelling geldt voor
		      $0 \leq k < n$, dan geldt dus:
		      $|A| = k \en |\Rho(A)| = 2^k$.

		      Laat $k = n-1$ en $B = A \cup \{b\}$ waarbij $b \notin A$
		      Dan kunnen we de machtsverzameling opstellen
		      voor $\Rho(B)$ waarbij we $\Rho(B)$ partioneren met $\Rho(B) = P_1 \cup P_2$ in het geval 
			  $b \in \Rho(B) \of b \notin \Rho(B)$:
		      \[P_1 = \{V \in \Rho(B): b \notin V)\} = \Rho(A)\]
		      \[ P_2 = \{V \in \Rho(B): b \in V \} \]
			  Nu moeten we nog bewijzen dat $P_1 \cup P_2 = \Rho(B)$ met: 
			  \begin{align*}
					& P_1 \cup P_2 \subset \Rho(B) \\
					\tx{\emph{Geval $V \in P_1$}:}& \\
					& \implies V \in \Rho(B) \en b \notin V \quad (\tx{De definitie van $P_1$}) \\
					& \implies V \in \Rho(B) \\
					\tx{\emph{Geval $V \in P_2$}:}& \\
					& \implies V \in \Rho(B) \en b \in V \quad (\tx{De definitie van $P_2$}) \\
					& \implies V \in \Rho(B) \\ \\
					&\Rho(B) \subset P_1 \cup P_2 \\
					\tx{\emph{Geval $b \notin V$ met $V\in \Rho(B)$}:}& \\
					& \implies V \in \Rho(B) \en b \notin V \quad (\tx{Volgens definitie van $P_1$}) \\
					& \implies V \in P_1 \\
					& \implies V \in P_1 \cup P_2 \\
					\tx{\emph{Geval $b \in V$ met $V\in \Rho(B)$}:}& \\
					& \implies V \in \Rho(B) \en b \in V \quad (\tx{Volgens definitie van $P_2$}) \\
					& \implies V \in P_2 \\
					& \implies V \in P_2 \cup P_1
			  \end{align*}
			  Dus $P_1 \cup P_2 = \Rho(B)$, we kunnen nu een functie $f: P_1 \rightarrow P_2$ met $f(V) = V \cup \{b\}$, 
			  als we bewijzen dat $f$ een bijectie is dan $\Rho(A) = |P_1| = |p_2|$.

			  Voor alle $V, W \in P_1 \als f(V) = f(W) \dan V \cup \{b\} = W \cup \{b\} \implies V = W$, en 
			  dus is $f$ injectief. 
			  
			  Voor alle $W \in P_2$ geldt dat er een $V \in P_1$ bestaat zodanig dat $f(V) = W$.
			  laat $V = W \slash \{b\}$ dan $f(W \slash \{b\}) = W\slash\{b\} \cup \{b\} = W$, en dus is
			  $f$ surjectief.

			  We hebben dus een bijectie tussen $P_1 \en P_2$ en dus $|P_1| = |P_2|$.
		      Als we nu alles optellen krijgen we:
		      \begin{align*}
			      |\Rho(B)| & = |P_1 \cup P_2|    \quad (P_1, P_2 \tx{ partioneren } \Rho(B))         \\
			                & = |\Rho(A)| + |\Rho(A)| \quad (|P_1| = |P_2| \tx{ omdat we $P_2$ kunnen construreren uit $P_1$})\\
			                & = 2^k + 2^k             \\
			                & = 2(2^k)                \\
			                & = 2^{k+1}               \\
		      \end{align*}
		\item \emph{Uitspraak waar voor alle $n$:}

		      We stellen dat de uitspraak waar is voor alle
		      $n$ en gaan dit bewijzen door te stellen dat
		      dit niet zo is door vervolgens een tegenspraak
		      te vinden.

		      Vanuit de welordening van $\N$ is er een kleinste
		      element $n_0 \in \N$.
		      We zeggen dat er een kleinste $n_0$ moet bestaan
		      waarvoor de uispraak niet waar is, maar
		      we hebben al bewezen dat voor $0 \leq k \leq n$
		      de uitspraak waar is. Dit is dus een tegenspraak
		      en daarom geldt voor alle $n \in N$ dat de uitspraak waar is.
	\end{enumerate}
\end{proof}
\section*{Opgave 2}
\begin{proof}
Als $f$ een inverse heeft geldt: \[f^{-1}(a) = b \Leftrightarrow f(b) = a\] 
Neem $a \in A$
en laat $f^{-1}(a) = b$, en $f(b) = a$. 
Vervolgens stellen we op dat $f(b) = f(f^{-1}(a)) = a$,
echter hebben we per definitie van $f$ dat $f(f(a)) = a$, en dus moet $f = f^{-1}$, omdat $f$ injectief is
kan $f(f(a)) = f(f^{-1}(a))$ alleen als $f(a) = f^{-1}(a)$.
\end{proof}

\section*{Opgave 3}
\begin{proof}
	Volledige inductie laten we eerst beginnen met $n=0$ aangezien de formule impliceert dat $0 \in \N$.
	\begin{enumerate}
		\item \emph{Basisstap: $n = 0$}
		Voor de linkerkant:
			\[\sum_{i = 0}^{0}3i(i + 1) = 3(0)(0+1) = \boxed{0}\]
		en de rechterkant:	
			\[0(0+1)(0+2) = \boxed{0} \]
			Dus de stelling klopt als $n = 0$.
		\item \emph{Inductiehypothese:}
			Neem aan dat de stelling klopt voor $0 \leq k \leq n$ dus: 
			\[ \sum_{i = 0}^{k} 3i(i +1) = k(k+1)(n+2) \]
			Laat nu $k = n-1$ dan, en bewijs voor $k + 1 = n$:
			\begin{align*}
				\sum_{i= 0}^{k} 3i(i + 1) + 3(k+1)(k + 2) &= k(k+1)(k+2) + 3(k+1)(k+2) \quad (\tx{Substitutie}) \\ 
				&= (k + 3)(k+1)(k+2) \quad (\tx{Distributie}) \\
				&= (k + 1)(k+2)(k+3) \quad (\tx{Commutativiteit}) \\
				&= (k + 1)((k+1) + 1)((k+1)+ 2)
			\end{align*}
		\item \emph{Uitspraak waar voor alle $n$:}
		Stel dat de stelling niet geldt voor alle $n \in \N$ dan bestaat er een kleinste $n_0$ waarbij de stelling niet waar moet zijn,
		echter geldt voor $0 \leq k \leq n$ dat de stelling klopt, en dus is dit een tegenspraak.
		
		De stelling is dus waar voor alle $n \in \N$.
	\end{enumerate}
\end{proof}

\end{document}