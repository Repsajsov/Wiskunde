\documentclass{article}
\usepackage[utf8]{inputenc}
\usepackage[dutch]{babel}
\usepackage{amsmath, amssymb, amsfonts, amsthm}
\usepackage[margin=2cm]{geometry}
\usepackage{cancel}
\usepackage{enumitem}

\setlength{\parindent}{0pt}

\begin{document}

{\Large \textbf{Wiskundige Structuren Huiswerk}}

\bigskip

\textbf{Jasper Vos} \hfill \textbf{Huiswerkset 3} \hfill \today \\
Studentnr: \emph{s2911159}

\rule{\textwidth}{2pt}

\bigskip

\newcommand{\N}{\mathbb{N}}
\newcommand{\f}[2]{\frac{#1}{#2}}
\newcommand{\tx}[1]{\text{#1}}
\newcommand{\cn}[1]{\cancel{#1}}
\newcommand{\R}{\mathbb{R}}
\newcommand{\Rho}{\mathcal{P}}
\newcommand{\en}{\tx{ en }}
\newcommand{\of}{\tx{ of }}

\section*{Opgave 1}
\begin{proof}
	Ik bewijs voor zowel $n = 0 \en n = 1$,
	omdat er vaak dubbelzinnigheid is over
	$0 \in \N \of 0 \notin \N$.
	\begin{enumerate}
		\item \emph{Basisstap: $n = 0, \ n = 1$}

		      Neem $|A| = |\emptyset|$ = 0,
		      dan en slechts dan als
		      $|\Rho(A)| |\{\emptyset\}| = 2^0 = 1$.
		      Dus de uitspraak geldt voor $n = 0$.


		      Neem $|A| = |\{a\}| = 1$
		      dan en slechts dan als
		      $|\Rho(A)| = |\{\emptyset, \{a\}\}| = 2^1 = 2$.
		      Dus de uistpraak geldt voor $n = 1$.
		\item \emph{Inductiehypothese:}
		      Neem aan dat de stelling geldt voor
		      $0 \leq k < n$, dan geldt dus:
		      $|A| = k \en |\Rho(A)| = 2^k$.

		      Laat $k = n-1$, en $B = A \cup \{b\}$,
		      Dan kunnen we de machtsverzameling opstellen
		      voor $\Rho(B)$ waarbij:
		      \[C = \{V \in \Rho(B): \{b\} \notin V)\} = \Rho(A) \]
		      en:
		      \[ D = \{V \in \Rho(B): \{b\} \in V \} \]
		      Hieruit volgt $C \cup D = \Rho(B)$. Merk op
		      dat $D = \{V \cup \{b\} : V \in \Rho(A)\}$, en dus:
		      \[|D| = |\Rho(A)|\]
		      Als we nu alles optellen krijgen we:
		      \begin{align*}
			      |\Rho(B)| & = |C| + |D|             \\
			                & = |\Rho(A)| + |\Rho(A)| \\
			                & = 2^k + 2^k             \\
			                & = 2(2^k)                \\
			                & = 2^{k+1}               \\
		      \end{align*}
		\item \emph{Uitspraak waar voor alle $n$:}
		      We stellen dat de uitspraak waar is voor alle
		      $n$ en gaan dit bewijzen door te stellen dat
		      dit niet zo is door vervolgens een tegenspraak
		      te vinden.

		      Vanuit de welordening van $\N$ is er een kleinste
		      element $n_0 \in \N$.
		      We zeggen dat er een kleinste $n_0$ moet bestaan
		      waarvoor de uispraak niet waar is, maar
		      we hebben al bewezen dat voor $0 \leq k \leq n$
		      de uitspraak waar is. Dit is dus een tegenspraak
		      en daarom geldt voor alle $n \in N$ dat de uitspraak waar is.
	\end{enumerate}
\end{proof}
\section*{Opgave 2}
Voor alle $a \in A : f(f(a)) = a$,
dan betekent dus ook omdat $f$ een bijectie is
dat $ff = \tx{id}_A$


\end{document}