\documentclass{article}
\usepackage[utf8]{inputenc}
\usepackage[dutch]{babel}
\usepackage{amsmath, amssymb, amsfonts, amsthm}
\usepackage[margin=2cm]{geometry}
\usepackage{cancel}
\usepackage{enumitem}

\setlength{\parindent}{0pt}

\begin{document}

{\Large \textbf{Wiskundige Structuren Huiswerk}}

\bigskip

\textbf{Jasper Vos} \hfill \textbf{Huiswerkset 2} \hfill \today \\
Studentnr: \emph{s2911159}

\rule{\textwidth}{2pt}

\bigskip

\newcommand{\N}{\mathbb{N}}
\newcommand{\f}[2]{\frac{#1}{#2}}
\newcommand{\tx}[1]{\text{#1}}
\newcommand{\cn}[1]{\cancel{#1}}
\newcommand{\R}{\mathbb{R}}

\section*{Opgave 1}
\begin{proof}

    Als $A$ en $B$ beide aftelbaar oneindig zijn dan
    bestaat er een bijectie vanuit $\N$.
    Zij $g: \N \rightarrow A$ en $h: \N \rightarrow B$.
    Laten we nu een functie $f$ opstellen
    waarbij we $A$ en $B$ vlechten in $f$.
    Zij $f:\N \rightarrow A \cup B$: waarvoor geldt:

    \[
        f(n) = \begin{cases}
            a_{\f{n}{2}}   & \tx{als $n$ even is}   \\
            b_{\f{n+1}{2}} & \tx{als $n$ oneven is} \\
        \end{cases}
    \]

    \subsection*{Injectiviteit:}
    Zij $n_1, n_2 \in \N \tx{ te bewijzen } f(n_1) = f(n_2) \implies n_1 = n_2$
    \subsubsection*{Geval 1: $n_1, n_2$ beide even:}
    \begin{align*}
        f(n_1)         & = f(n_2)                                               \\
        a_{\f{n_1}{2}} & = a_{\f{n_2}{2}}                                       \\
        n_1            & = n_2 \quad (\because g, h \tx{ beide injectief zijn})
    \end{align*}
    \subsubsection*{Geval 2: $n_1, n_2$ beide oneven:}
    \begin{align*}
        f(n_1)           & = f(n_2)                                               \\
        b_{\f{n_1+1}{2}} & = b_{\f{n_2+1}{2}}                                     \\
        n_1              & = n_2 \quad (\because g, h \tx{ beide injectief zijn})
    \end{align*}
    \subsubsection*{Geval 3: $n_1$ even en $n_2$ oneven:}
    \begin{align*}
        f(n_1)         & = f(n_2)              \\
        a_{\f{n_1}{2}} & \neq b_{\f{n_2+1}{2}} \\
    \end{align*}
    Dit kan niet omdat aan de linkerkant
    een element van $A$ staat en aan de rechterkant
    een element van $B$, en gegeven was dat $A \cap B = \emptyset$.
    Geval 3 is dus niet relevant en de overige zijn correct dus
    $f$ is injectief.

    \subsection*{Surjectiviteit:}
    Te bewijzen $\forall m \in A \cup B$ geldt dat er een $n \in \N$ bestaat zodanig
    dat $f(n) = m$
    \subsubsection*{Geval 1: Als $m \in A$:}
    Dan bestaat er een $a_k \in A$ zodanig dat $a_k = m$, we kiezen $n = 2k$ dan:

    \[f(2k) = a_{\f{2k}{2}} = a_k = m\]

    \subsubsection*{Geval 2: Als $m \in B$:}
    Dan bestaat er een $b_k \in B$ zodanig dat $b_k = m$, we kiezen $n = 2k-1$ dan:

    \[f(2k-1) = b_{\f{(2k-1) + 1}{2}} = b_k = m\]

    $f$ is surjectief en injectief waaruit volgt dat $f$ bijectief is.
    Hieruit volgt dus dat $f: \N \rightarrow A \cup B$ aftelbaar oneindig is.

\end{proof}
\section*{Opgave 2}
\begin{proof}
    Om bijectiviteit te bewijzen moet $f$ zowel injectief als surjectief zijn.
    \subsection*{Injectiviteit:}
    Te bewijzen $\forall (n_1, x_1), \ (n_2, x_2) \in \N_{>0}
        \rightarrow \R$ dan $f(n_1, x_1) = f(n_2, x_2) \implies n_1 = n_2, \tx{ en } x_1 = x_2$.
    \begin{align*}
        (98n_1x_1, n_1) & = (98n_2x_2, n_2)                                                          \\ \\
        n_1             & = n_2 \quad (\tx{Rechtercomponent})                                        \\ \\
        98n_1x_1        & = 98n_2x_2                                                                 \\
        n_1x_1          & = n_2x_2                                                                   \\
        x_1             & = x_2 \quad (\tx{Geschrapt vanuit het resultaat van het rechtercomponent})
    \end{align*}
    Voor zowel het rechter als linkercomponent geldt dus dat $f$ injectief is.
    \subsection*{Surjectiviteit:}
    Te bewijzen $\forall (y, m) \in \R \times \N_{>0}$
    bestaat er een $(n, x) \in \N_{>0} \times \R$
    zodanig dat $f(n, x) = (y, m)$.
    neem $(n, x) = (m, \f{y}{98m})$, dan:
    \[f(m, \f{y}{98m}) = (\cn{98m}(\f{y}{\cn{98m}}), m) = (y, m)\]
    Hieruit volgt dus dat $f$ surjectief is. We hebben
    al eerder bewezen dat $f$ injectief is en dus is $f$
    bijectief.
\end{proof}
\subsection*{Inverse:}
Tot slot stellen we de inverse $f^{-1} = g$ op.

\[g: \R \times \N_{>0} \rightarrow \N_{>0} \times \R \tx{ met } g(x, n) = (n, \f{x}{98n}) \]


\end{document}