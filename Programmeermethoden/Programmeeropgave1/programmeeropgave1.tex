%
% Stel je wilt het C++-programma iets.cc mooi printen,
% en wellicht er nog wat begeleidende tekst bij schrijven.
%

\documentclass[10pt]{article}

\usepackage[utf8]{inputenc}

\parindent=0pt

\usepackage{fullpage}

\frenchspacing

\usepackage{microtype}

\usepackage[english,dutch]{babel}

\usepackage{listings}
% Er zijn talloze parameters ...
\lstset{language=C++, showstringspaces=false, basicstyle=\small,
  numbers=left, numberstyle=\tiny, numberfirstline=false, breaklines=true,
  stepnumber=1, tabsize=8, 
  commentstyle=\ttfamily, identifierstyle=\ttfamily,
  stringstyle=\itshape}

\title{Programmeeropgave 1}
\author{Jasper Vos}

\begin{document}

\selectlanguage{dutch}

\maketitle

\section{Uitleg}
Dit verslag heb ik vooral gebruik gemaakt van "if" condities, cout, cin en variabelen.

\section{Tijd}
Ik heb ieder werkcollege eraan gewerkt plus in de tussenuren op donderdag. 
Ik kom dan op een aantal van ongeveer $6\cdot\frac{5}{3} \cdot  = 10$ uur.

\section*{Code}
En dit is het programma:

\lstinputlisting{programmeeropgave1.cc}

\end{document}

%
% Stel je wilt het C++-programma iets.cc mooi printen,
% en wellicht er nog wat begeleidende tekst bij schrijven.
%

\documentclass[10pt]{article}

\usepackage[utf8]{inputenc}

\parindent=0pt

\usepackage{fullpage}

\frenchspacing

\usepackage{microtype}

\usepackage[english,dutch]{babel}

\usepackage{listings}
% Er zijn talloze parameters ...
\lstset{language=C++, showstringspaces=false, basicstyle=\small,
  numbers=left, numberstyle=\tiny, numberfirstline=false, breaklines=true,
  stepnumber=1, tabsize=8, 
  commentstyle=\ttfamily, identifierstyle=\ttfamily,
  stringstyle=\itshape}

\title{Mooi printen}
\author{Walter Kosters}

\begin{document}

\selectlanguage{dutch}

\maketitle

\section{Uitleg}
Tijd voor een verslag.
Hoe print je daarbij een C$\stackrel{++}{}$ programma mooi?
Bijvoorbeeld met \LaTeX-package \verb+listings+.
Let op de talloze opties, bijvoorbeeld voor de tab-grootte.

\section{Tijd}
Er is hier veel tijd aan besteed.

\section*{Code}
En dit is het programma:

\lstinputlisting{iets.cc}

\end{document}
