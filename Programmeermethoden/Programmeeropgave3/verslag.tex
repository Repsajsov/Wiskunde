\documentclass[10pt]{article}
\usepackage{graphicx}
\usepackage[utf8]{inputenc}

\parindent=0pt

\usepackage{fullpage}

\frenchspacing

\usepackage{microtype}

\usepackage[english,dutch]{babel}

\usepackage{listings}
\lstset{language=C++, showstringspaces=false, basicstyle=\small,
  numbers=left, numberstyle=\tiny, numberfirstline=false, breaklines=true,
  stepnumber=1, tabsize=8, 
  commentstyle=\ttfamily, identifierstyle=\ttfamily,
  stringstyle=\itshape}

\title{LightsOut!}
\author{Vincent van der Velden \emph{s4494059}, Jasper Vos \emph{s2911159}}

\begin{document}

\selectlanguage{dutch}

\maketitle

\section{Uitleg}
\subsection*{Hoe het spel werkt}
LightsOut is een spel waarbij we een rooster hebben met een aantal lampen die 
aan of uit staan. Het is de bedoeling om elke lamp uit te krijgen, 
waarbij de speler een lamp kan 
"togglen". Elke "toggle" zorgt er echter voor dat de desbetreffende 
lamp wordt omgedraaid en de $4$ directe buren ook.

\subsection*{Structuur van ons programma}
Het programma wat we hebben gemaakt heeft drie verschillende menu's:

\subsubsection*{Parameter menu:}
Instellen van hoogte, breedte, karakters, proportie lampen aan,
torus-modus, pen-modus, en animatie.

\subsubsection*{Puzzel menu:}
Verschillende manieren om puzzels op te lossen, 
\begin{itemize}
    \item \emph{volg()} Lost automatisch op behalve de onderste rij.
    \item \emph{losOp()} Lost volledige puzzel op door volg steeds aan te roepen, en als het geen oplossing bevat wordt er een bepaalde combinatie in de bovenste rij aangeklikt om vervolgens \emph{volg()} weer te proberen. Dit blijft doorgaan tot alle mogelijke lamp combinaties zijn geprobeerd.
    \item \emph{speelOplossing()} Lost een volledige puzzel vanuit de geregistreerde "klikken" van een gegenereerde puzzel. 
    \item \emph{spelen()} Hiermee kun je zelf de puzzel spelen waarbij je de cursor kan bewegen met WASD, en klikken met E.
\end{itemize}

\subsubsection*{Teken menu:}
\begin{itemize}
    \item \emph{maakSchoon()} Zet alle arrays die te maken hebben met de puzzel naar false.
    \item \emph{randomBord()} Zet random lampen aan met een bepaalde proportie die je kan instellen bij het parameter menu.
    \item \emph{toggle()} Hiermee kun je bepaalde lampen aan/uit zetten met de cursor die beweegt met WASD, en klikken met E.
    \item \emph{genereerBord()} Dit zet random zetten waarmee we een puzzel genereren. De zetten worden opgeslagen en kan gebruikt worden voor \emph{speelOplossing}.
\end{itemize}

\begin{figure}[h]
\centering
\includegraphics[width=0.5\textwidth]{preview.png}
\caption{Screenshot van het programma}
\end{figure}

\section{Tijd}

\section*{Code}

\lstinputlisting{programmeeropgave3.cc}

\end{document}
