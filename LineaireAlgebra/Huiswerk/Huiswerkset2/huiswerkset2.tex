\documentclass{article}
\usepackage[utf8]{inputenc}
\usepackage[dutch]{babel}
\usepackage{amsmath, amssymb, amsfonts, amsthm}
\usepackage[margin=2cm]{geometry}
\usepackage{cancel}
\usepackage{enumitem}

% Stof: §§1.4–1.7 van het dictaat, behalve subsecties 1.5.2 en 1.7.2
% Opgaven:  1.4.2(1), 1.4.4, 1.4.5, 1.5.2, 1.5.5, 1.6.1, 1.6.3, 1.6.5, 1.7.3

\setlength{\parindent}{0pt}

\begin{document}

{\Large \textbf{Lineaire Algebra Huiswerk}}

\bigskip

\textbf{Jasper Vos} \hfill \textbf{Huiswerkset 2} \hfill \today \\
Studentnr: \emph{s2911159} 

\rule{\textwidth}{2pt}

\bigskip

\section*{Opdracht 1.4.2(1)}
\begin{proof}
    Schrijf probleem in logische notatie zodat het makkelijker is.
    \begin{align*}
        ||v|| = ||w|| &\Leftrightarrow <v-w, v+w> = 0 \\
        &\Leftrightarrow \\
        (||v|| = ||w|| \implies <v-w, v+w> = 0) \quad &\wedge \quad (<v-w, v+w> = 0 \implies ||v|| = ||w||)
    \end{align*}
    Bewijs $||v|| = ||w|| \implies <v, w> = 0$:
    \begin{align*}
        \boxed{||v|| = ||w||} &\implies ||v|| - ||w|| = 0 \\
        &\implies ||v||^2 - ||w||^2 = 0 \\
        &\implies <v, v> - <w, w> = 0 \\
        &\implies <v, v> + <v, w> - <v, w> - <w, w> = 0 \\
        &\implies v_1^2 + vw_1 - vw_1 - w_1^2 + v_2^2 + vw_2 -vw_2 -w_2^2 + \dots + v_n^2 + vw_n - vw_n -w_n^2 = 0 \\
        &\implies (v_1 + w_1)(v_1 - w_1) + \dots + (v_n + w_n)(v_n - w_n) = 0 \\
        &\implies \boxed{<v+w, v-w> = 0}
    \end{align*}
    Bewijs nu voor $<v, w> = 0 \implies ||v|| = ||w||$:
    \begin{align*}
        \boxed{<v+w, v-w> = 0} &\implies (v_1 + w_1)(v_1 - w_1) + \dots + (v_n + w_n)(v_n - w_n) = 0 \\
        &\implies v_1^2 + vw_1 - vw_1 - w_1^2 + v_2^2 + vw_2 -vw_2 -w_2^2 + \dots + v_n^2 + vw_n - vw_n -w_n^2 = 0 \\
        &\implies <v, v> + <v, w> - <v, w> - <w, w> = 0 \\
        &\implies <v, v> - <w, w> = 0 \\
        &\implies ||v||^2 - ||w||^2 = 0 \\
        &\implies \boxed{||v|| - ||w|| = 0}
    \end{align*}
    Dit voldoet aan $||v|| = ||w|| \Leftrightarrow <v-w, v+w> = 0$ en dus zijn we klaar.
\end{proof}
\section*{Opdracht 1.6.5}
Laten we eerst het vlak $V$ opspannen en daarna de afstand tussen het vlak en $q$ berekenen.
Neem $V = \{ \lambda_1 \lambda_2 \in \mathbb{R} : p_1 + \lambda_1 (p_2-p_1) + \lambda_2 (p_3-p_1) \}$ en vul in:
\[ V = \{ \lambda_1, \lambda_2 \in \mathbb{R} : (1,2,-1) + \lambda_1(0,-2,2) + \lambda_2(-3, 1, 2) \} \]
Bereken het kruisproduct $(p_2 - p_1) \times (p_3 - p_1)$ zodat je een vector krijgt die zowel op $p_2 - p_1$ en $p_3-p-1$ loodrecht staat.
\[n=(-2)(2) - (2)(1), (2)(-3)- (0)(2), (0)(1)-(-2)(-3) = (-6,-6,-6) \]
Tot slot projecteren we de vector $q-p_1$ op de normaalvector $n$.
\[ <(q-p_1)n> = ((2-1)(-6)+(2-2)(-6)+(1-(-1))(-6))-18 \]
Bereken nu de afstand:
\[\frac{18}{\sqrt{36+36+36}} = \boxed{\sqrt{3}}\]



\end{document}