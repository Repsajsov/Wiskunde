
\documentclass{article}
\usepackage[utf8]{inputenc}
\usepackage[dutch]{babel}
\usepackage{amsmath, amssymb, amsfonts, amsthm}
\usepackage[margin=2cm]{geometry}
\usepackage{cancel}
\usepackage{enumitem}


\setlength{\parindent}{0pt}

\newcommand{\R}{\mathbb{R}}

\begin{document}

{\Large \textbf{Lineaire Algebra}}

\bigskip

\textbf{Jasper Vos} \hfill \textbf{Huiswerkset 9} \hfill \today \\
Studentnr: \emph{s2911159}

\rule{\textwidth}{2pt}

\bigskip

\section*{Opgave 8.1.4(1)}
We hebben $\text{rk}(g \circ f) = \dim g(f[U]) \leq \dim(f[U]) = \text{rk}(f)$
Als $g$ injectief is, dan verandert $g$ de dimensie van $f[U]$ niet, en dus:
\[\boxed{\text{rk}(g \circ f) = \text{rk}(f)}\]
\subsection*{Voorbeeld}
Neem de volgende vectorruimtes: $U = \R, \ V = \R^2$ en tot slot $W = \R$.

Stel de afbeeldingen op $f: U \rightarrow V$ met $f(x) = (x, 0)$, en $g: V \rightarrow W$ waarbij $g(x, y) = x$. 
$g$ is niet injectief omdat: \[g(0, 1) = 0 = g(0, 2)\]

Het beeld van $f$ heeft als dimensie de $x$-as want het tweede component is constant $0$, en dus $\dim(f) = 1$. Het beeld van $g \circ f$ geeft ook de $x$-as want $g \circ f = x$, en dus $\dim(g \circ f) = 1$.

Dit betekent dus dat: \[\boxed{\text{rk}(g \circ f) = \text{rk}(f)\text{ waarbij $g$ niet injectief is}}\] 

\section*{Opgave 8.1.4(2)}
We hebben $\text{rk}(g \circ f) \leq \text{rk}(g)$, als $f$ surjectief dan $f[U] = V$, en dus:
\[\boxed{g[f[U]] = g[V] \text{, en dus } \text{rk}(g \circ f) = \text{rk}(g)}\]
\subsection*{Voorbeeld}
Neem de volgende vectorruimtes: $U = \R, \ V = \R^2$ en $W = \R$.
Stel de afbeeldingen op: \[f: U \rightarrow V \text{ met } f(x) = (x, x) \text{ en } g: V \rightarrow W \text{ met } g(x, y) = x + y\]
Merk op dat $f$ niet surjectief want $(1, 0)$ wordt niet bereikt in $V$ vanuit $f$.
Het beeld van $g$ geeft $\dim(g) = 1$, en voor $g \circ f(x) = g(x, x) = x + x = 2x$ en dus $\dim(\text{rk}(g \circ f)) = 1$.
Dit betekent dus: \[\boxed{\text{rk}(g \circ f) = \text{rk}(g) \text{ waarbij $f$ niet surjectief is}} \]

\pagebreak
\section*{Opgave 8.3.2}

Gegeven is $F = \mathbb{F}_2$, dus alle bewerkingen gebeuren modulo $2$.

De deelruimte $U \subseteq \mathbb{F}_2^4$ is opgespannen door:
\[
u_1 = (1,1,1,1), \ u_2 = (1,1,0,0), \text{ en } u_3 = (0,1,1,0)
\]
Een willekeurig element uit $U$ is dus te schrijven als:        
$
(a+b, \ a+b+c, \ a+c, \ a)
\quad \text{met } a,b,c \in \mathbb{F}_2.
$

De deelruimte $V \subseteq \mathbb{F}_2^4$ is opgespannen door:
$
v_1 = (1,1,1,0), \ v_2 = (0,1,1,1)
$
Een willekeurig element uit $V$ is dus:
\[
(p, \ p+q, \ p+q, \ q)
\quad \text{met } p,q \in \mathbb{F}_2.
\]

We zoeken $U \cap V$, dus we stellen:
\[
(a+b, \ a+b+c, \ a+c, \ a)
=
(p, \ p+q, \ p+q, \ q)
\]

Dit geeft het stelsel:
\begin{align*}
a + b &= p \\
a + b + c &= p + q \\
a + c &= p + q \\
a &= q
\end{align*}

Invullen $a = q$ in de derde vergelijking:
$
q + c = p + q \implies c = p
$
, dan wordt de eerste vergelijking:
\[
a + b = p \implies q + b = p \implies b = p + q
\]

Invullen in de tweede vergelijking:
\[
a + b + c = p + q
\implies q + (p+q) + p = p + q
\implies 0 = 0
\]

Nu geldt $p = q$, en dus:
$
p = q = a = c \quad \text{en} \quad b = 0.
$
Dit levert maar één niet-nulvector op:
$
(1,0,0,1)
$

\[
U \cap V = \text{span}\{(1,0,0,1)\}
\]

\[
\boxed{U \cap V = \{0, (1,0,0,1)\} = \text{span}\{(1,0,0,1)\}}
\]
\end{document}