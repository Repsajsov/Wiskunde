\documentclass{article}
\usepackage[utf8]{inputenc}
\usepackage[dutch]{babel}
\usepackage{amsmath, amssymb, amsfonts, amsthm}
\usepackage[margin=2cm]{geometry}
\usepackage{cancel}
\usepackage{enumitem}

\setlength{\parindent}{0pt}

\newcommand{\R}{\mathbb{R}}

\begin{document}

{\Large \textbf{Lineaire Algebra Huiswerk}}

\bigskip

\textbf{Jasper Vos} \hfill \textbf{Huiswerkset 6} \hfill \today \\
Studentnr: \emph{s2911159}

\rule{\textwidth}{2pt}

\bigskip

\section*{Opgave 5.4.3}

Bepaal de matrix $M$ waarvoor $f_M: \R^3 \to \R^3$ reflectie is in het vlak $x + 2y - z = 0$.

Het vlak heeft normale vector $a = (1, 2, -1)$.

\subsection*{Normale vector normaliseren}

Bereken lengte:
\[ \langle a, a \rangle = 1^2 + 2^2 + (-1)^2 = 6 \]

Dus $\|a\| = \sqrt{6}$ en:
\[ \hat{a} = \frac{1}{\sqrt{6}}(1, 2, -1) \]

\subsection*{Reflectie}

In matrixvorm (hier ben ik niet helemaal zeker, maar volgens mij klopt dit):
\[ M = I_3 - 2\hat{a}\hat{a}^{\top} \]

\subsection*{Berekenen van $\hat{a}\hat{a}^{\top}$}

\begin{align*}
    \hat{a}\hat{a}^{\top} & = \frac{1}{6} \begin{pmatrix} 1 \\ 2 \\ -1 \end{pmatrix}
    \begin{pmatrix} 1 & 2 & -1 \end{pmatrix}
    = \frac{1}{6} \begin{pmatrix}
                      1  & 2  & -1 \\
                      2  & 4  & -2 \\
                      -1 & -2 & 1
                  \end{pmatrix}
\end{align*}

\subsection*{Matrix $M$ bepalen}

\begin{align*}
    M & = I_3 - 2\hat{a}\hat{a}^{\top}                                      \\
      & = \begin{pmatrix} 1 & 0 & 0 \\ 0 & 1 & 0 \\ 0 & 0 & 1 \end{pmatrix}
    - \frac{2}{6} \begin{pmatrix}
                      1  & 2  & -1 \\
                      2  & 4  & -2 \\
                      -1 & -2 & 1
                  \end{pmatrix}                                            \\
      & = \begin{pmatrix} 1 & 0 & 0 \\ 0 & 1 & 0 \\ 0 & 0 & 1 \end{pmatrix}
    - \frac{1}{3} \begin{pmatrix}
                      1  & 2  & -1 \\
                      2  & 4  & -2 \\
                      -1 & -2 & 1
                  \end{pmatrix}
\end{align*}

Element voor element:
\begin{align*}
    M_{11} & = 1 - \frac{1}{3} = \frac{2}{3}, \quad M_{12} = 0 - \frac{2}{3} = -\frac{2}{3}, \quad M_{13} = 0 + \frac{1}{3} = \frac{1}{3}  \\
    M_{21} & = 0 - \frac{2}{3} = -\frac{2}{3}, \quad M_{22} = 1 - \frac{4}{3} = -\frac{1}{3}, \quad M_{23} = 0 + \frac{2}{3} = \frac{2}{3} \\
    M_{31} & = 0 + \frac{1}{3} = \frac{1}{3}, \quad M_{32} = 0 + \frac{2}{3} = \frac{2}{3}, \quad M_{33} = 1 - \frac{1}{3} = \frac{2}{3}
\end{align*}
en dus:

\[ \boxed{M = \begin{pmatrix}
            \frac{2}{3}  & -\frac{2}{3} & \frac{1}{3} \\[0.3em]
            -\frac{2}{3} & -\frac{1}{3} & \frac{2}{3} \\[0.3em]
            \frac{1}{3}  & \frac{2}{3}  & \frac{2}{3}
        \end{pmatrix}} \]

\section*{Opgave 5.5.4}

Rotatiematrix $A_{\alpha} = \begin{pmatrix} \cos \alpha & -\sin \alpha \\ \sin \alpha & \cos \alpha \end{pmatrix}$

\textbf{Te bewijzen:}
\begin{align*}
    \cos(\alpha + \beta) & = \cos \alpha \cos \beta - \sin \alpha \sin \beta \\
    \sin(\alpha + \beta) & = \sin \alpha \cos \beta + \cos \alpha \sin \beta
\end{align*}

\begin{proof}
    Als je eerst over hoek $\beta$ roteert en daarna over hoek $\alpha$, krijg je een rotatie over $\alpha + \beta$.
    In matrixvorm betekent dit dat $A_{\alpha} \cdot A_{\beta} = A_{\alpha+\beta}$ moet gelden.

    Bereken $A_{\alpha} \cdot A_{\beta}$:
    \begin{align*}
        A_{\alpha} \cdot A_{\beta} & =
        \begin{pmatrix} \cos \alpha & -\sin \alpha \\ \sin \alpha & \cos \alpha \end{pmatrix}
        \begin{pmatrix} \cos \beta & -\sin \beta \\ \sin \beta & \cos \beta \end{pmatrix}
    \end{align*}

    Bereken de elementen:

    \textbf{(1,1):} $\cos \alpha \cos \beta + (-\sin \alpha)(\sin \beta) = \cos \alpha \cos \beta - \sin \alpha \sin \beta$

    \textbf{(1,2):} $\cos \alpha (-\sin \beta) + (-\sin \alpha)(\cos \beta) = -\cos \alpha \sin \beta - \sin \alpha \cos \beta$

    Dit kan ik ook schrijven als: $-(\sin \alpha \cos \beta + \cos \alpha \sin \beta)$

    \textbf{(2,1):} $\sin \alpha \cos \beta + \cos \alpha \sin \beta$

    \textbf{(2,2):} $\sin \alpha(-\sin \beta) + \cos \alpha \cos \beta = \cos \alpha \cos \beta - \sin \alpha \sin \beta$

    Dus:
    \[ A_{\alpha} \cdot A_{\beta} =
        \begin{pmatrix}
            \cos \alpha \cos \beta - \sin \alpha \sin \beta & -(\sin \alpha \cos \beta + \cos \alpha \sin \beta) \\
            \sin \alpha \cos \beta + \cos \alpha \sin \beta & \cos \alpha \cos \beta - \sin \alpha \sin \beta
        \end{pmatrix} \]

    We weten dat dit gelijk moet zijn aan:
    \[ A_{\alpha + \beta} =
        \begin{pmatrix}
            \cos(\alpha + \beta) & -\sin(\alpha + \beta) \\
            \sin(\alpha + \beta) & \cos(\alpha + \beta)
        \end{pmatrix} \]

    Door de elementen te vergelijken krijgen we:

    Uit (1,1): $\boxed{\cos(\alpha + \beta) = \cos \alpha \cos \beta - \sin \alpha \sin \beta}$

    Uit (2,1): $\boxed{\sin(\alpha + \beta) = \sin \alpha \cos \beta + \cos \alpha \sin \beta}$

    \bigskip

    Element (1,2) en (2,2) geven hetzelfde resultaat, dus onze berekening klopt.

\end{proof}

\end{document}