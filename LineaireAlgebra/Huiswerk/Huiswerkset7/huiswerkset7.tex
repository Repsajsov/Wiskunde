\documentclass{article}
\usepackage[utf8]{inputenc}
\usepackage[dutch]{babel}
\usepackage{amsmath, amssymb, amsfonts, amsthm}
\usepackage[margin=2cm]{geometry}
\usepackage{cancel}
\usepackage{enumitem}

\setlength{\parindent}{0pt}

\newcommand{\R}{\mathbb{R}}

\begin{document}

{\Large \textbf{Lineaire Algebra}}

\bigskip

\textbf{Jasper Vos} \hfill \textbf{Huiswerkset 7} \hfill \today \\
Studentnr: \emph{s2911159}

\rule{\textwidth}{2pt}

\bigskip

\section*{Opgave 6.3.3}
Geef generatoren voor de kernel van matrix $A$ uit Example 6.10.

Uit Example 6.10 hebben we de row echelon form:
\[ A' = \begin{pmatrix}
        0 & 1 & 0 & 2  & 0 & -5 \\
        0 & 0 & 1 & -2 & 0 & 3  \\
        0 & 0 & 0 & 0  & 1 & 1  \\
        0 & 0 & 0 & 0  & 0 & 0
    \end{pmatrix} \]

De pivots staan in kolommen $j_1 = 2, j_2 = 3, j_3 = 5$, dus kolommen zonder pivot zijn $k \in \{1, 4, 6\}$.
We construeren voor elke kolom zonder pivot een generator $w_k$ volgens Propositie 6.19.

\textbf{Generator $w_1$:} Voor kolom 1 nemen we $x = (1, x_2, x_3, 0, x_5, 0)^\top$ met $x_1 = 1$.
Dan moet $A'x = 0$:
\begin{align*}
    \text{Rij 3:} \quad & x_5 = 0 \\
    \text{Rij 2:} \quad & x_3 = 0 \\
    \text{Rij 1:} \quad & x_2 = 0
\end{align*}
Dus $w_1 = (1, 0, 0, 0, 0, 0)^\top$.

\textbf{Generator $w_4$:} Voor kolom 4 nemen we $x = (0, x_2, x_3, 1, x_5, 0)^\top$ met $x_4 = 1$.
Dan moet $A'x = 0$:
\begin{align*}
    \text{Rij 3:} \quad & x_5 = 0                                           \\
    \text{Rij 2:} \quad & x_3 - 2 \cdot 1 = 0 \quad \implies \quad x_3 = 2  \\
    \text{Rij 1:} \quad & x_2 + 2 \cdot 1 = 0 \quad \implies \quad x_2 = -2
\end{align*}
Dus $w_4 = (0, -2, 2, 1, 0, 0)^\top$.

\textbf{Generator $w_6$:} Voor kolom 6 nemen we $x = (0, x_2, x_3, 0, x_5, 1)^\top$ met $x_6 = 1$.
Dan moet $A'x = 0$:
\begin{align*}
    \text{Rij 3:} \quad & x_5 + 1 = 0 \quad \implies \quad x_5 = -1 \\
    \text{Rij 2:} \quad & x_3 + 3 = 0 \quad \implies \quad x_3 = -3 \\
    \text{Rij 1:} \quad & x_2 - 5 = 0 \quad \implies \quad x_2 = 5
\end{align*}
Dus $w_6 = (0, 5, -3, 0, -1, 1)^\top$.

Volgens Propositie 6.3 geldt $\ker A = \ker A'$, dus:
\[ \boxed{\ker A = L(w_1, w_4, w_6)} \]
met
\[ w_1 = \begin{pmatrix} 1 \\ 0 \\ 0 \\ 0 \\ 0 \\ 0 \end{pmatrix}, \quad
    w_4 = \begin{pmatrix} 0 \\ -2 \\ 2 \\ 1 \\ 0 \\ 0 \end{pmatrix}, \quad
    w_6 = \begin{pmatrix} 0 \\ 5 \\ -3 \\ 0 \\ -1 \\ 1 \end{pmatrix} \]

\section*{Opgave 6.3.5}
Zij $A \in \text{Mat}(m \times n, F)$ een matrix en $f_A : F^n \to F^m$ de geassocieerde lineaire afbeelding.

\subsection*{(1) Als $f_A$ injectief is, dan $m \geq n$}
\begin{proof}
    Veronderstel $f_A$ is injectief. Breng $A$ in row echelon form $A'$ via elementaire rijoperaties.
    Volgens Propositie 6.6 is $f_{A'}$ ook injectief.

    Volgens Propositie 6.20 is $f_{A'}$ injectief $\Leftrightarrow$ elke kolom van $A'$ bevat een pivot.
    Dus $A'$ heeft $n$ pivots, en elke pivot staat in een andere rij.
    Hieruit volgt dat $A'$ minstens $n$ rijen heeft, dus $m \geq n$.
\end{proof}

\subsection*{(2) Als $A$ inverteerbaar is, dan $m = n$}
\begin{proof}
    Veronderstel $A$ is inverteerbaar. Dan bestaat $A^{-1} \in \text{Mat}(n \times m, F)$ zodat:
    \begin{align*}
        A \cdot A^{-1} & = I_m \\
        A^{-1} \cdot A & = I_n
    \end{align*}

    Merk op dat $f_A \circ f_{A^{-1}} = f_{AA^{-1}} = f_{I_m} = \text{id}_{F^m}$ surjectief is.
    Dus $f_A$ is surjectief, wat betekent dat $\text{im } f_A = F^m$.

    Ook geldt $f_{A^{-1}} \circ f_A = f_{A^{-1}A} = f_{I_n} = \text{id}_{F^n}$ injectief is.
    Dus $f_A$ is injectief, wat betekent dat $\ker f_A = \{0\}$.

    Omdat $f_A$ injectief is, volgt uit deel (1) dat $m \geq n$.
    Omdat $f_A$ surjectief is, moet $\dim(\text{im } f_A) = m$.

    Breng $A$ in row echelon form $A'$. Dan heeft $A'$ precies $n$ kolommen en elke kolom bevat een pivot (want $f_A$ injectief).
    Dus $A'$ heeft $n$ niet-nul rijen. Maar $\text{im } A' = \text{im } A = F^m$ heeft dimensie $m$.

    Aangezien de $n$ niet-nul rijen van $A'$ de row space opspannen en $\dim(R(A')) = m$, volgt $n \geq m$.

    Combineren we $m \geq n$ en $n \geq m$, dan $m = n$.
\end{proof}

\end{document}