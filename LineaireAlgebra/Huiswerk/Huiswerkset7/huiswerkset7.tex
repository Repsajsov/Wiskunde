\documentclass{article}
\usepackage[utf8]{inputenc}
\usepackage[dutch]{babel}
\usepackage{amsmath, amssymb, amsfonts, amsthm}
\usepackage[margin=2cm]{geometry}
\usepackage{cancel}
\usepackage{enumitem}

\setlength{\parindent}{0pt}

\newcommand{\R}{\mathbb{R}}

\begin{document}

{\Large \textbf{Lineaire Algebra}}

\bigskip

\textbf{Jasper Vos} \hfill \textbf{Huiswerkset 7} \hfill \today \\
Studentnr: \emph{s2911159}

\rule{\textwidth}{2pt}

\bigskip

\section*{Opgave 6.3.3}
Gebruik het resultaat uit het voorbeeld dus:

\[A' = \begin{pmatrix}
    0 & 1 & 0 & 2 & 0 & -5 \\
    0 & 0 & 1 & -2 & 0 & 3 \\
    0 & 0 & 0 & 0 & 1 & 1 \\
    0 & 0 & 0 & 0 & 0 & 0
\end{pmatrix}\]

We gaan kernvectoren opstellen bij de kolommen die geen pivot hebben. Ofwel kolom 1, en 4.
Dan hebben we dus de volgende systemen van vergelijkingen.

\begin{align*}
\end{align*}

\section*{Opgave 6.3.5}


\end{document}