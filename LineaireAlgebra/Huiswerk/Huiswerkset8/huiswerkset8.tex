\documentclass{article}
\usepackage[utf8]{inputenc}
\usepackage[dutch]{babel}
\usepackage{amsmath, amssymb, amsfonts, amsthm}
\usepackage[margin=2cm]{geometry}
\usepackage{cancel}
\usepackage{enumitem}

\setlength{\parindent}{0pt}

\newcommand{\R}{\mathbb{R}}

\begin{document}

{\Large \textbf{Lineaire Algebra}}

\bigskip

\textbf{Jasper Vos} \hfill \textbf{Huiswerkset 8} \hfill \today \\
Studentnr: \emph{s2911159}

\rule{\textwidth}{2pt}

\bigskip

\section*{Opgave 7.1.8}

We kunnen een matrix opstellen waarbij de rijen de functie $f_i$ voorstellen en $a_j$ de kolommen.

\[
    \begin{pmatrix}
        f_0(a_0) & f_0(a_1) & f_0(a_2) & \dots  & f_0(a_n) \\
        f_1(a_0) & f_1(a_1) & f_1(a_2) & \dots  & f_1(a_n) \\
        f_2(a_0) & f_2(a_1) & f_2(a_2) & \dots  & f_2(a_n) \\
        \vdots   & \vdots   & \vdots   & \ddots & \vdots   \\
        f_n(a_0) & f_n(a_1) & f_n(a_2) & \dots  & f_n(a_n)
    \end{pmatrix}
\]
Met de voorwaarde die we hebben gekregen voor elke $f$:
\[
    f_i(a_j) =
    \begin{cases}
        1 & \text{als j $\geq$ i} \\
        0
    \end{cases}
\]
krijgen we:
\[
    \begin{pmatrix}
        1      & 0      & 0      & \dots  & 0      \\
        1      & 1      & 0      & \dots  & 0      \\
        1      & 1      & 1      & \dots  & 0      \\
        \vdots & \vdots & \vdots & \ddots & \vdots \\
        1      & 1      & 1      & 1      & 1
    \end{pmatrix}
\]
Met rij-reductie kunnen we $R_2 = R_2 - R_1$.
\[
    \begin{pmatrix}
        1      & 0      & 0      & \dots  & 0      \\
        0      & 1      & 0      & \dots  & 0      \\
        1      & 1      & 1      & \dots  & 0      \\
        \vdots & \vdots & \vdots & \ddots & \vdots \\
        1      & 1      & 1      & 1      & 1
    \end{pmatrix}
\]
Met rij-reductie kunnen we $R_3 = R_3 - R_2 - R_1$.
\[
    \begin{pmatrix}
        1      & 0      & 0      & \dots  & 0      \\
        0      & 1      & 0      & \dots  & 0      \\
        0      & 0      & 1      & \dots  & 0      \\
        \vdots & \vdots & \vdots & \ddots & \vdots \\
        1      & 1      & 1      & 1      & 1
    \end{pmatrix}
\]
Merk op dat je dit $n$ keer kan doen en dan krijg je de identiteitsmatrix waaruit volgt dat het linear onafhankelijk is.



\end{document}