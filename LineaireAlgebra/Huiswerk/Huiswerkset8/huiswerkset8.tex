\documentclass{article}
\usepackage[utf8]{inputenc}
\usepackage[dutch]{babel}
\usepackage{amsmath, amssymb, amsfonts, amsthm}
\usepackage[margin=2cm]{geometry}
\usepackage{cancel}
\usepackage{enumitem}

\setlength{\parindent}{0pt}

\newcommand{\R}{\mathbb{R}}

\begin{document}

{\Large \textbf{Lineaire Algebra}}

\bigskip

\textbf{Jasper Vos} \hfill \textbf{Huiswerkset 8} \hfill \today \\
Studentnr: \emph{s2911159}

\rule{\textwidth}{2pt}

\bigskip

\section*{Opgave 7.1.8}
\subsection*{Idee}
We moeten laten zien dat $f_0, f_1, \dots f_n$ lineair onafhankelijk zijn.
Dit kunnen we doen door een matrix $A$ te construeren.
Deze matrix $A$ bevat functies als rijen en de punten worden geëvalueerd als kolommen.
Tot slot proberen we matrix $A$ met rij-reductie naar de identiteitsmatrix te krijgen.
Dit toont aan dat de rang $n+1$ is en dus $n+1$ functies lineair onafhankelijk zijn.


\subsection*{Constructie}
\subsubsection*{Stap 1: Matrix opstellen}
Definieer matrix $A$ met $A_{ij} = f_i(a_j)$:
\[
    A =  \begin{pmatrix}
        f_0(a_0) & f_0(a_1) & f_0(a_2) & \dots  & f_0(a_n) \\
        f_1(a_0) & f_1(a_1) & f_1(a_2) & \dots  & f_1(a_n) \\
        f_2(a_0) & f_2(a_1) & f_2(a_2) & \dots  & f_2(a_n) \\
        \vdots   & \vdots   & \vdots   & \ddots & \vdots   \\
        f_n(a_0) & f_n(a_1) & f_n(a_2) & \dots  & f_n(a_n)
    \end{pmatrix}
\]
\subsubsection*{Stap 2: Voorwaarde toepassen}
Gegeven is
$
    f_i(a_j) =
    \begin{cases}
        1 & \text{, als $j \leq i$} \\
        0 & \text{, als $j > i$}
    \end{cases}
$
, Herschrijf de matrix $A$:
\[
     A = \begin{pmatrix}
        1      & 0      & 0      & \dots  & 0      \\
        1      & 1      & 0      & \dots  & 0      \\
        1      & 1      & 1      & \dots  & 0      \\
        \vdots & \vdots & \vdots & \ddots & \vdots \\
        1      & 1      & 1      & 1      & 1
    \end{pmatrix}
\]
\subsubsection*{Stap 3: Rij-reductie}
Met de operatie $R_2 = R_2 - R_1$, vervolgens $R_3 = R_3 - R_2 - R_1$, etc. Kunnen we $I_{n+1}$ construeren.
\[
    \begin{pmatrix}
        1      & 0      & 0      & \dots  & 0      \\
        1      & 1      & 0      & \dots  & 0      \\
        1      & 1      & 1      & \dots  & 0      \\
        \vdots & \vdots & \vdots & \ddots & \vdots \\
        1      & 1      & 1      & 1      & 1
    \end{pmatrix}
\xrightarrow{R_2 = R_2 - R_1}
    \begin{pmatrix}
        1      & 0      & 0      & \dots  & 0      \\
        0      & 1      & 0      & \dots  & 0      \\
        1      & 1      & 1      & \dots  & 0      \\
        \vdots & \vdots & \vdots & \ddots & \vdots \\
        1      & 1      & 1      & 1      & 1
    \end{pmatrix}
    \rightarrow \dots \xrightarrow{R_{n+1} = R_{n+1} - R_n - \dots - R_1} 
    \boxed{
    \begin{pmatrix}
        1      & 0      & 0      & \dots  & 0      \\
        0      & 1      & 0      & \dots  & 0      \\
        0      & 0      & 1      & \dots  & 0      \\
        \vdots & \vdots & \vdots & \ddots & \vdots \\
        0      & 0      &0      & 0      & 1
    \end{pmatrix}
    }
\]

\subsection*{Conclusie}
Doordat we $I_{n+1}$ hebben geconstrueerd betekent dit dat we een matrix $A$ hebben met een rang van $n+1$, en dus  
geldt $f_0, f_1, \dots f_n$ lineair onafhankelijk.

\pagebreak
\section*{Opgave 7.3.3(1)}
\subsection*{Idee} 
We proberen de dimensie te vinden door de basis van $V = \ker(a)$ te vinden.

\subsection*{Basis van $V$}
Neem $v \in V$, dan moet $\langle v, a\rangle = 0$ vanwege $V = a^\bot$.
\[v = \langle(v_1, v_2, v_3, v_4), \ (1, 1, 1, 1)\rangle = 0 \iff  v_1 = -v_2 -v_3 -v_4 \]
De algemene oplossing is:
\[v = v_2(-1, 1, 0, 0) + v_3(-1, 0, 1, 0) + v_4(-1, 0, 0, 1) \]
Met vrije parameters $v_2, \ v_3, \ v_4 \in \R$.

Een basis voor $V$ is dus: 
 \[\boxed{V = L\Big((-1, 1, 0, 0), (-1, 0, 1, 0), (-1, 0, 0, 1)\Big)}\]

\subsection*{Conclusie}
De basis bevat $3$ vectoren, en dus $\dim(V) = 3$.

\section*{Opgave 7.3.3(2)}
\subsection*{Idee}
We checken of $v_1, \ v_2$ lineair onafhankelijk zijn door de vectoren als rijen in een matrix te zetten.
Als de rang van deze matrix gelijk aan $2$ is dan zijn de vectoren lineair onafhankelijk.
\subsection*{Rij-reductie}
Zet $v_1$ als $R_1$ en $v_2$ als $R_2$ en reduceer:
\begin{align*}
\begin{pmatrix}
    2 & -3 & -1 & 2 \\
    -1 & 3 & 2 & -4
\end{pmatrix}
&\xrightarrow{R_2 = R_1 + R_2}
\begin{pmatrix}
    2 & -3 & -1 & 2 \\
    1 & 0 & 1 & -2
\end{pmatrix} \\
&\xrightarrow{R_1 \rightleftarrows R_2}
\begin{pmatrix}
    1 & 0 & 1 & -2 \\
    2 & -3 & -1 & 2
\end{pmatrix}  \\
&\xrightarrow{R_2 = R_2 - 2R_1}
\begin{pmatrix}
    1 & 0 & 1 & -2 \\
    0 & -3 & -3 & 6 
\end{pmatrix} \\
&\xrightarrow{R_2 = -\frac{1}{3}R_2}
\boxed{
\begin{pmatrix}
    1 & 0 & 1 & -2 \\
    0 & 1 & 1 & -2
\end{pmatrix}
}
\end{align*}
\subsection*{Conclusie}
Met rij-reductie zien we dus dat de matrix $2$ pivots heeft en daarmee een rang van $2$. Hieruit volgt 
dat $v_1, \ v_2$ lineair onafhankelijk zijn.
\pagebreak
\section*{Opgave 7.3.3(3)}
\subsection*{Idee}
Aangezien we weten dat $dim(V) = 3$, hebben we $3$ onafhankelijke vectoren nodig voor een basis.
We hebben al $v_1$ en $v_2$ uit \emph{Opgave 7.3.3(2)}, en kiezen
 $v_3 = (-1, 0, 0, 1)$ die we hebben gevonden in \emph{Opgave 7.3.3(1)}.
We stellen de matrix op en reduceren om te verifiëren of de rang gelijk is aan $3$.
\subsection*{Rij-reductie}
We zetten de vectoren $v_1, \ v_2$, en $v_3$ in de matrix en reduceren: 
\begin{align*}
    \begin{pmatrix}
        -1 & 0 & 0 & 1 \\
        2 & -3 & -1 & 2 \\
        1 & 0 &1 & -2 
    \end{pmatrix}
    &\xrightarrow{R_3 = R_3 + R_1}
    \begin{pmatrix}
        -1 & 0 & 0 & 1 \\
        2 & -3 & -1 & 2 \\
        0 & 0 &1 & -1 \\ 
    \end{pmatrix} \\
    &\xrightarrow{R_2 = R_2 + 2R_1}
    \begin{pmatrix}
        -1 & 0 & 0 & 1 \\
        0 & -3 & -1 & 4 \\
        0 & 0 &1 & -1 \\ 
    \end{pmatrix} \\
    &\xrightarrow{R_1 = -R_1, \ R_2 = -\frac{1}{3}R_2}
    \boxed{\begin{pmatrix}
        1 & 0 & 0 & -1 \\
        0 & 1 & \frac{1}{3} & -\frac{4}{3} \\
        0 & 0 &1 & -1 \\ 
    \end{pmatrix} }
\end{align*}
\subsection*{Conclusie}
De matrix heeft $3$ pivots en dus is de rang van de matrix gelijk aan $3$ net zoals $\dim(V) = 3$. De vectoren zijn lineair onafhankelijk en vormen de basis:
\[
\boxed{
V = \Big((-1, 0, 0, 1), \ (2, -3, -1, 2), \ (1, 0, 1, -2) \Big)
}
\]

\end{document}