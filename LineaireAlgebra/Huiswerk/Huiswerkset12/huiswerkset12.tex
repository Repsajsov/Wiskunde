\documentclass{article}
\usepackage[utf8]{inputenc}
\usepackage[dutch]{babel}
\usepackage{amsmath, amssymb, amsfonts, amsthm}
\usepackage[margin=2cm]{geometry}
\usepackage{cancel}
\usepackage{enumitem}

\setlength{\parindent}{0pt}

\newcommand{\R}{\mathbb{R}}
\newcommand{\f}[2]{\frac{#1}{#2}}

\begin{document}

{\Large \textbf{Lineaire Algebra}}

\bigskip

\textbf{Jasper Vos} \hfill \textbf{Huiswerkset 12} \hfill \today \\
Studentnr: \emph{s2911159}

\rule{\textwidth}{2pt}

\bigskip

\section*{Opgave 10.5.1(1)}

\subsection*{Idee}
We bepalen de rang van $C_a$ door naar de determinant te kijken.
Een $3 \times 3$ matrix heeft rang $3$ als de determinant niet nul is.

\subsection*{Berekening determinant}
Bereken $\det(C_a)$ met rijontwikkeling langs de eerste rij:
\begin{align*}
    \det(C_a) & = \det \begin{pmatrix} a & a & 2 \\ 1 & 0 & a \\ -2 & -3 & 1 \end{pmatrix}                                                                                            \\
              & = a \det \begin{pmatrix} 0 & a \\ -3 & 1 \end{pmatrix} - a \det \begin{pmatrix} 1 & a \\ -2 & 1 \end{pmatrix} + 2 \det \begin{pmatrix} 1 & 0 \\ -2 & -3 \end{pmatrix} \\
              & = a(0 + 3a) - a(1 + 2a) + 2(-3 - 0)                                                                                                                                   \\
              & = 3a^2 - a - 2a^2 - 6                                                                                                                                                 \\
              & = a^2 - a - 6                                                                                                                                                         \\
              & = (a-3)(a+2)
\end{align*}

\subsection*{Conclusie}
De determinant is nul als $(a-3)(a+2) = 0$, dus voor $a = 3$ of $a = -2$.

\textbf{Rang van $C_a$:}
\begin{itemize}
    \item Als $a \in \{-2, 3\}$: $\det(C_a) = 0$, dus rang $< 3$. Na rij-reductie blijkt rang $= 2$.
    \item Als $a \notin \{-2, 3\}$: $\det(C_a) \neq 0$, dus $\boxed{\text{rang}(C_a) = 3}$
\end{itemize}

\pagebreak
\section*{Opgave 10.5.1(2)}

\subsection*{Idee}
Gebruik het resultaat uit deel (1). Als $a = 2$, dan is $2 \notin \{-2, 3\}$, dus rang$(C_2) = 3$.
Een $3 \times 3$ matrix met rang $3$ is inverteerbaar.

\subsection*{Bewijs inverteerbaarheid}
Voor $a = 2$:
\[\det(C_2) = (2-3)(2+2) = (-1)(4) = -4 \neq 0\]
Dus $C_2$ is \textbf{inverteerbaar}.

\subsection*{Berekening inverse}
We gebruiken de uitgebreide matrix $[C_2 \ | \ I]$ en brengen deze naar $[I \ | \ C_2^{-1}]$:
\begin{align*}
    \left(\begin{array}{ccc|ccc}
                  2  & 2  & 2 & 1 & 0 & 0 \\
                  1  & 0  & 2 & 0 & 1 & 0 \\
                  -2 & -3 & 1 & 0 & 0 & 1
              \end{array}\right)
     & \xrightarrow{R_1 = \frac{1}{2}R_1}
    \left(\begin{array}{ccc|ccc}
                  1  & 1  & 1 & \frac{1}{2} & 0 & 0 \\
                  1  & 0  & 2 & 0           & 1 & 0 \\
                  -2 & -3 & 1 & 0           & 0 & 1
              \end{array}\right)                      \\
     & \xrightarrow{R_2 = R_2 - R_1}
    \left(\begin{array}{ccc|ccc}
                  1  & 1  & 1 & \frac{1}{2}  & 0 & 0 \\
                  0  & -1 & 1 & -\frac{1}{2} & 1 & 0 \\
                  -2 & -3 & 1 & 0            & 0 & 1
              \end{array}\right)                     \\
     & \xrightarrow{R_3 = R_3 + 2R_1}
    \left(\begin{array}{ccc|ccc}
                  1 & 1  & 1 & \frac{1}{2}  & 0 & 0 \\
                  0 & -1 & 1 & -\frac{1}{2} & 1 & 0 \\
                  0 & -1 & 3 & 1            & 0 & 1
              \end{array}\right)                      \\
     & \xrightarrow{R_2 = -R_2}
    \left(\begin{array}{ccc|ccc}
                  1 & 1  & 1  & \frac{1}{2} & 0  & 0 \\
                  0 & 1  & -1 & \frac{1}{2} & -1 & 0 \\
                  0 & -1 & 3  & 1           & 0  & 1
              \end{array}\right)                     \\
     & \xrightarrow{R_3 = R_3 + R_2}
    \left(\begin{array}{ccc|ccc}
                  1 & 1 & 1  & \frac{1}{2} & 0  & 0 \\
                  0 & 1 & -1 & \frac{1}{2} & -1 & 0 \\
                  0 & 0 & 2  & \frac{3}{2} & -1 & 1
              \end{array}\right)                      \\
     & \xrightarrow{R_3 = \frac{1}{2}R_3}
    \left(\begin{array}{ccc|ccc}
                  1 & 1 & 1  & \frac{1}{2} & 0            & 0           \\
                  0 & 1 & -1 & \frac{1}{2} & -1           & 0           \\
                  0 & 0 & 1  & \frac{3}{4} & -\frac{1}{2} & \frac{1}{2}
              \end{array}\right)  \\
     & \xrightarrow{R_2 = R_2 + R_3}
    \left(\begin{array}{ccc|ccc}
                  1 & 1 & 1 & \frac{1}{2} & 0            & 0           \\
                  0 & 1 & 0 & \frac{5}{4} & -\frac{3}{2} & \frac{1}{2} \\
                  0 & 0 & 1 & \frac{3}{4} & -\frac{1}{2} & \frac{1}{2}
              \end{array}\right)   \\
     & \xrightarrow{R_1 = R_1 - R_2}
    \left(\begin{array}{ccc|ccc}
                  1 & 0 & 1 & -\frac{3}{4} & \frac{3}{2}  & -\frac{1}{2} \\
                  0 & 1 & 0 & \frac{5}{4}  & -\frac{3}{2} & \frac{1}{2}  \\
                  0 & 0 & 1 & \frac{3}{4}  & -\frac{1}{2} & \frac{1}{2}
              \end{array}\right) \\
     & \xrightarrow{R_1 = R_1 - R_3}
    \boxed{
        \left(\begin{array}{ccc|ccc}
                  1 & 0 & 0 & -\frac{3}{2} & 2            & -1          \\
                  0 & 1 & 0 & \frac{5}{4}  & -\frac{3}{2} & \frac{1}{2} \\
                  0 & 0 & 1 & \frac{3}{4}  & -\frac{1}{2} & \frac{1}{2}
              \end{array}\right)
    }
\end{align*}

\subsection*{Antwoord}
\[
    \boxed{
    C_2^{-1} = \begin{pmatrix}
        -\frac{3}{2} & 2            & -1          \\
        \frac{5}{4}  & -\frac{3}{2} & \frac{1}{2} \\
        \frac{3}{4}  & -\frac{1}{2} & \frac{1}{2}
    \end{pmatrix}
    = \frac{1}{4} \begin{pmatrix}
        -6 & 8  & -4 \\
        5  & -6 & 2  \\
        3  & -2 & 2
    \end{pmatrix}
    }
\]

\pagebreak
\section*{Opgave 10.5.1(3)}

\subsection*{Idee}
Het stelsel $C_a x = v_b$ heeft oneindig veel oplossingen als:
\begin{itemize}
    \item De rang van $C_a$ kleiner is dan $3$ (vrije variabelen)
    \item Het stelsel consistent is (geen tegenstrijdigheden)
\end{itemize}

Uit deel (1) weten we dat rang$(C_a) < 3$ precies wanneer $a \in \{-2, 3\}$.
We moeten nu voor beide waarden nagaan welke $b$ consistentie geeft.

\subsection*{Geval $a = -2$}
Voor $a = -2$ wordt het stelsel:
\[
    \begin{pmatrix}
        -2 & -2 & 2  \\
        1  & 0  & -2 \\
        -2 & -3 & 1
    \end{pmatrix}
    \begin{pmatrix} x_1 \\ x_2 \\ x_3 \end{pmatrix}
    =
    \begin{pmatrix} 2 \\ 1 \\ b \end{pmatrix}
\]

Breng de uitgebreide matrix naar rij-echelonvorm:
\begin{align*}
    \left(\begin{array}{ccc|c}
                  -2 & -2 & 2  & 2 \\
                  1  & 0  & -2 & 1 \\
                  -2 & -3 & 1  & b
              \end{array}\right)
     & \xrightarrow{R_1 = -\frac{1}{2}R_1}
    \left(\begin{array}{ccc|c}
                  1  & 1  & -1 & -1 \\
                  1  & 0  & -2 & 1  \\
                  -2 & -3 & 1  & b
              \end{array}\right)                            \\
     & \xrightarrow{R_2 = R_2 - R_1, \ R_3 = R_3 + 2R_1}
    \left(\begin{array}{ccc|c}
                  1 & 1  & -1 & -1  \\
                  0 & -1 & -1 & 2   \\
                  0 & -1 & -1 & b-2
              \end{array}\right)                            \\
     & \xrightarrow{R_3 = R_3 - R_2}
    \left(\begin{array}{ccc|c}
                  1 & 1  & -1 & -1  \\
                  0 & -1 & -1 & 2   \\
                  0 & 0  & 0  & b-4
              \end{array}\right)
\end{align*}

Voor consistentie moet $b - 4 = 0$, dus $\boxed{b = 4}$.

\subsection*{Geval $a = 3$}
Voor $a = 3$ wordt het stelsel:
\[
    \begin{pmatrix}
        3  & 3  & 2 \\
        1  & 0  & 3 \\
        -2 & -3 & 1
    \end{pmatrix}
    \begin{pmatrix} x_1 \\ x_2 \\ x_3 \end{pmatrix}
    =
    \begin{pmatrix} 2 \\ 1 \\ b \end{pmatrix}
\]

Breng de uitgebreide matrix naar rij-echelonvorm:
\begin{align*}
    \left(\begin{array}{ccc|c}
                  3  & 3  & 2 & 2 \\
                  1  & 0  & 3 & 1 \\
                  -2 & -3 & 1 & b
              \end{array}\right)
     & \xrightarrow{R_1 \leftrightarrow R_2}
    \left(\begin{array}{ccc|c}
                  1  & 0  & 3 & 1 \\
                  3  & 3  & 2 & 2 \\
                  -2 & -3 & 1 & b
              \end{array}\right)                             \\
     & \xrightarrow{R_2 = R_2 - 3R_1, \ R_3 = R_3 + 2R_1}
    \left(\begin{array}{ccc|c}
                  1 & 0  & 3  & 1   \\
                  0 & 3  & -7 & -1  \\
                  0 & -3 & 7  & b+2
              \end{array}\right)                             \\
     & \xrightarrow{R_3 = R_3 + R_2}
    \left(\begin{array}{ccc|c}
                  1 & 0 & 3  & 1   \\
                  0 & 3 & -7 & -1  \\
                  0 & 0 & 0  & b+1
              \end{array}\right)
\end{align*}

Voor consistentie moet $b + 1 = 0$, dus $\boxed{b = -1}$.

\subsection*{Antwoord}
Het stelsel $C_a x = v_b$ heeft oneindig veel oplossingen voor:
\[
    \boxed{(a, b) \in \{(-2, 4), \ (3, -1)\}}
\]

\pagebreak
\section*{Opgave 10.5.1(4)}

\subsection*{Idee}
Uit deel (3) hebben we twee paren gevonden: $(a,b) = (-2, 4)$ en $(a,b) = (3, -1)$.
De kleinste waarde van $a$ is $a = -2$, dus we beschrijven de oplossingsruimte voor $(a,b) = (-2, 4)$.

\subsection*{Oplossingsruimte bepalen}
Uit deel (3) hadden we voor $a = -2$ en $b = 4$ de gereduceerde matrix:
\[
    \left(\begin{array}{ccc|c}
            1 & 1  & -1 & -1 \\
            0 & -1 & -1 & 2  \\
            0 & 0  & 0  & 0
        \end{array}\right)
\]

Breng naar volledig gereduceerde rij-echelonvorm:
\begin{align*}
    \left(\begin{array}{ccc|c}
                  1 & 1  & -1 & -1 \\
                  0 & -1 & -1 & 2  \\
                  0 & 0  & 0  & 0
              \end{array}\right)
     & \xrightarrow{R_2 = -R_2}
    \left(\begin{array}{ccc|c}
                  1 & 1 & -1 & -1 \\
                  0 & 1 & 1  & -2 \\
                  0 & 0 & 0  & 0
              \end{array}\right)        \\
     & \xrightarrow{R_1 = R_1 - R_2}
    \left(\begin{array}{ccc|c}
                  1 & 0 & -2 & 1  \\
                  0 & 1 & 1  & -2 \\
                  0 & 0 & 0  & 0
              \end{array}\right)
\end{align*}

Dit geeft het stelsel:
\begin{align*}
    x_1 - 2x_3 & = 1  \\
    x_2 + x_3  & = -2
\end{align*}

De variabele $x_3$ is vrij. Stel $x_3 = t$ met $t \in \R$, dan:
\begin{align*}
    x_1 & = 1 + 2t \\
    x_2 & = -2 - t \\
    x_3 & = t
\end{align*}

\subsection*{Parametrische vorm}
\[
    x = \begin{pmatrix} x_1 \\ x_2 \\ x_3 \end{pmatrix}
    = \begin{pmatrix} 1 + 2t \\ -2 - t \\ t \end{pmatrix}
    = \begin{pmatrix} 1 \\ -2 \\ 0 \end{pmatrix}
    + t \begin{pmatrix} 2 \\ -1 \\ 1 \end{pmatrix}
\]

\subsection*{Antwoord}
De oplossingsruimte voor $(a, b) = (-2, 4)$ is:
\[
    \boxed{
        \left\{ \begin{pmatrix} 1 \\ -2 \\ 0 \end{pmatrix}
        + t \begin{pmatrix} 2 \\ -1 \\ 1 \end{pmatrix}
        \, : \, t \in \R \right\}
    }
\]

Dit is een lijn door het punt $(1, -2, 0)$ in de richting van de vector $(2, -1, 1)$.

\section*{Opgave 11.1.3}

\subsubsection*{Eigenwaarden bepalen}
Los de karakteristieke vergelijking op:
\begin{align*}
    \det(A - \lambda I)                                                & = 0 \\
    \det\begin{pmatrix} 5-\lambda & -4 \\ 8 & -7-\lambda \end{pmatrix} & = 0 \\
    (5-\lambda)(-7-\lambda) - (-4)(8)                                  & = 0 \\
    -35 - 5\lambda + 7\lambda + \lambda^2 + 32                         & = 0 \\
    \lambda^2 + 2\lambda - 3                                           & = 0 \\
    (\lambda + 3)(\lambda - 1)                                         & = 0
\end{align*}

Eigenwaarden: $\boxed{\lambda_1 = -3 \text{ en } \lambda_2 = 1}$

\subsubsection*{Eigenruimte voor $\lambda_1 = -3$}
Los $(A + 3I)v = 0$ op:
\[
    \begin{pmatrix} 8 & -4 \\ 8 & -4 \end{pmatrix}
    \begin{pmatrix} v_1 \\ v_2 \end{pmatrix} = 0
\]

Rijreductie:
\begin{align*}
    \begin{pmatrix} 8 & -4 \\ 8 & -4 \end{pmatrix}
     & \xrightarrow{R_1 = \frac{1}{8}R_1}
    \begin{pmatrix} 1 & -\frac{1}{2} \\ 8 & -4 \end{pmatrix} \\
     & \xrightarrow{R_2 = R_2 - 8R_1}
    \begin{pmatrix} 1 & -\frac{1}{2} \\ 0 & 0 \end{pmatrix}
\end{align*}

Dit geeft $v_1 - \frac{1}{2}v_2 = 0 \implies v_1 = \frac{1}{2}v_2$.

Stel $v_2 = 2t$ met $t \in \R$, dan $v_1 = t$ en:
\[v = t\begin{pmatrix} 1 \\ 2 \end{pmatrix}\]

Basis voor $E_{-3}(A)$: $\boxed{\left\{\begin{pmatrix} 1 \\ 2 \end{pmatrix}\right\}}$

\subsubsection*{Eigenruimte voor $\lambda_2 = 1$}
Los $(A - I)v = 0$ op:
\[
    \begin{pmatrix} 4 & -4 \\ 8 & -8 \end{pmatrix}
    \begin{pmatrix} v_1 \\ v_2 \end{pmatrix} = 0
\]

Rijreductie:
\begin{align*}
    \begin{pmatrix} 4 & -4 \\ 8 & -8 \end{pmatrix}
     & \xrightarrow{R_1 = \frac{1}{4}R_1}
    \begin{pmatrix} 1 & -1 \\ 8 & -8 \end{pmatrix} \\
     & \xrightarrow{R_2 = R_2 - 8R_1}
    \begin{pmatrix} 1 & -1 \\ 0 & 0 \end{pmatrix}
\end{align*}

Dit geeft $v_1 - v_2 = 0 \implies v_1 = v_2$.

Stel $v_2 = t$ met $t \in \R$, dan $v_1 = t$ en:
\[v = t\begin{pmatrix} 1 \\ 1 \end{pmatrix}\]

Basis voor $E_1(A)$: $\boxed{\left\{\begin{pmatrix} 1 \\ 1 \end{pmatrix}\right\}}$

\pagebreak

\subsubsection*{Eigenwaarden bepalen}
Los de karakteristieke vergelijking op:
\begin{align*}
    \det(B - \lambda I)                                                                            & = 0 \\
    \det\begin{pmatrix} 3-\lambda & 2 & 0 \\ -1 & -\lambda & 0 \\ 0 & 0 & -3-\lambda \end{pmatrix} & = 0
\end{align*}

Gebruik cofactorontwikkeling langs de derde kolom:
\begin{align*}
    (-3-\lambda) \det\begin{pmatrix} 3-\lambda & 2 \\ -1 & -\lambda \end{pmatrix} & = 0 \\
    (-3-\lambda)[(3-\lambda)(-\lambda) - (2)(-1)]                                 & = 0 \\
    (-3-\lambda)[-3\lambda + \lambda^2 + 2]                                       & = 0 \\
    (-3-\lambda)(\lambda^2 - 3\lambda + 2)                                        & = 0 \\
    (-3-\lambda)(\lambda - 1)(\lambda - 2)                                        & = 0
\end{align*}

Eigenwaarden: $\boxed{\lambda_1 = -3, \ \lambda_2 = 1, \text{ en } \lambda_3 = 2}$

\subsubsection*{Eigenruimte voor $\lambda_1 = -3$}
Los $(B + 3I)v = 0$ op:
\[
    \begin{pmatrix} 6 & 2 & 0 \\ -1 & 3 & 0 \\ 0 & 0 & 0 \end{pmatrix}
    \begin{pmatrix} v_1 \\ v_2 \\ v_3 \end{pmatrix} = 0
\]

Rijreductie:
\begin{align*}
    \begin{pmatrix} 6 & 2 & 0 \\ -1 & 3 & 0 \\ 0 & 0 & 0 \end{pmatrix}
     & \xrightarrow{R_1 \leftrightarrow R_2}
    \begin{pmatrix} -1 & 3 & 0 \\ 6 & 2 & 0 \\ 0 & 0 & 0 \end{pmatrix}  \\
     & \xrightarrow{R_2 = R_2 + 6R_1}
    \begin{pmatrix} -1 & 3 & 0 \\ 0 & 20 & 0 \\ 0 & 0 & 0 \end{pmatrix} \\
     & \xrightarrow{R_1 = -R_1, \ R_2 = \frac{1}{20}R_2}
    \begin{pmatrix} 1 & -3 & 0 \\ 0 & 1 & 0 \\ 0 & 0 & 0 \end{pmatrix}  \\
     & \xrightarrow{R_1 = R_1 + 3R_2}
    \begin{pmatrix} 1 & 0 & 0 \\ 0 & 1 & 0 \\ 0 & 0 & 0 \end{pmatrix}
\end{align*}

Dit geeft $v_1 = 0$ en $v_2 = 0$, terwijl $v_3$ vrij is.

Stel $v_3 = t$ met $t \in \R$:
\[v = t\begin{pmatrix} 0 \\ 0 \\ 1 \end{pmatrix}\]

Basis voor $E_{-3}(B)$: $\boxed{\left\{\begin{pmatrix} 0 \\ 0 \\ 1 \end{pmatrix}\right\}}$

\subsubsection*{Eigenruimte voor $\lambda_2 = 1$}
Los $(B - I)v = 0$ op:
\[
    \begin{pmatrix} 2 & 2 & 0 \\ -1 & -1 & 0 \\ 0 & 0 & -4 \end{pmatrix}
    \begin{pmatrix} v_1 \\ v_2 \\ v_3 \end{pmatrix} = 0
\]

Rijreductie:
\begin{align*}
    \begin{pmatrix} 2 & 2 & 0 \\ -1 & -1 & 0 \\ 0 & 0 & -4 \end{pmatrix}
     & \xrightarrow{R_1 \leftrightarrow R_2}
    \begin{pmatrix} -1 & -1 & 0 \\ 2 & 2 & 0 \\ 0 & 0 & -4 \end{pmatrix} \\
     & \xrightarrow{R_2 = R_2 + 2R_1}
    \begin{pmatrix} -1 & -1 & 0 \\ 0 & 0 & 0 \\ 0 & 0 & -4 \end{pmatrix} \\
     & \xrightarrow{R_1 = -R_1, \ R_3 = -\frac{1}{4}R_3}
    \begin{pmatrix} 1 & 1 & 0 \\ 0 & 0 & 0 \\ 0 & 0 & 1 \end{pmatrix}    \\
     & \xrightarrow{R_2 \leftrightarrow R_3}
    \begin{pmatrix} 1 & 1 & 0 \\ 0 & 0 & 1 \\ 0 & 0 & 0 \end{pmatrix}
\end{align*}

Dit geeft $v_1 + v_2 = 0 \implies v_1 = -v_2$ en $v_3 = 0$.

Stel $v_2 = t$ met $t \in \R$, dan $v_1 = -t$ en:
\[v = t\begin{pmatrix} -1 \\ 1 \\ 0 \end{pmatrix}\]

Basis voor $E_1(B)$: $\boxed{\left\{\begin{pmatrix} -1 \\ 1 \\ 0 \end{pmatrix}\right\}}$

\subsubsection*{Eigenruimte voor $\lambda_3 = 2$}
Los $(B - 2I)v = 0$ op:
\[
    \begin{pmatrix} 1 & 2 & 0 \\ -1 & -2 & 0 \\ 0 & 0 & -5 \end{pmatrix}
    \begin{pmatrix} v_1 \\ v_2 \\ v_3 \end{pmatrix} = 0
\]

Rijreductie:
\begin{align*}
    \begin{pmatrix} 1 & 2 & 0 \\ -1 & -2 & 0 \\ 0 & 0 & -5 \end{pmatrix}
     & \xrightarrow{R_2 = R_2 + R_1}
    \begin{pmatrix} 1 & 2 & 0 \\ 0 & 0 & 0 \\ 0 & 0 & -5 \end{pmatrix} \\
     & \xrightarrow{R_3 = -\frac{1}{5}R_3}
    \begin{pmatrix} 1 & 2 & 0 \\ 0 & 0 & 0 \\ 0 & 0 & 1 \end{pmatrix}  \\
     & \xrightarrow{R_2 \leftrightarrow R_3}
    \begin{pmatrix} 1 & 2 & 0 \\ 0 & 0 & 1 \\ 0 & 0 & 0 \end{pmatrix}
\end{align*}

Dit geeft $v_1 + 2v_2 = 0 \implies v_1 = -2v_2$ en $v_3 = 0$.

Stel $v_2 = t$ met $t \in \R$, dan $v_1 = -2t$ en:
\[v = t\begin{pmatrix} -2 \\ 1 \\ 0 \end{pmatrix}\]

Basis voor $E_2(B)$: $\boxed{\left\{\begin{pmatrix} -2 \\ 1 \\ 0 \end{pmatrix}\right\}}$
\end{document}