\documentclass{article}
\usepackage[utf8]{inputenc}
\usepackage[dutch]{babel}
\usepackage{amsmath, amssymb, amsfonts, amsthm}
\usepackage[margin=2cm]{geometry}
\usepackage{cancel}
\usepackage{enumitem}


\setlength{\parindent}{0pt}

\newcommand{\R}{\mathbb{R}}

\begin{document}

{\Large \textbf{Lineaire Algebra}}

\bigskip

\textbf{Jasper Vos} \hfill \textbf{Huiswerkset 10} \hfill \today \\
Studentnr: \emph{s2911159}

\rule{\textwidth}{2pt}

\bigskip

\section*{Opgave 8.5.1}

\subsection*{Stelsel 1}
Gegeven stelsel:
\begin{align*}
    2x_1 + 3x_2 - 2x_3 & = 0 \\
    3x_1 + 2x_2 + 2x_3 & = 0 \\
    -x_2 + 2x_3        & = 0
\end{align*}

Matrix en vector:
$$
    A = \begin{pmatrix}
        2 & 3  & -2 \\
        3 & 2  & 2  \\
        0 & -1 & 2
    \end{pmatrix}, \quad
    b = \begin{pmatrix}
        0 \\ 0 \\ 0
    \end{pmatrix}
$$

Dit is een homogeen stelsel. Breng $A$ naar gereduceerde rij-echelonvorm:
$$
    \begin{pmatrix}
        2 & 3  & -2 \\
        3 & 2  & 2  \\
        0 & -1 & 2
    \end{pmatrix}
    \rightsquigarrow
    \begin{pmatrix}
        1 & 0 & 2  \\
        0 & 1 & -2 \\
        0 & 0 & 0
    \end{pmatrix}
$$

Uit de rij-echelonvorm volgt:
$$
    x_1 + 2x_3 = 0 \implies x_1 = -2x_3
$$
$$
    x_2 - 2x_3 = 0 \implies x_2 = 2x_3
$$

Dus de oplossingsverzameling is:
$$
    \left\{ t \begin{pmatrix} -2 \\ 2 \\ 1 \end{pmatrix} : t \in \mathbb{R} \right\}
$$

\subsection*{Stelsel 2}
Gegeven stelsel:
\begin{align*}
    2x_1 + 3x_2 - 2x_3 & = 1  \\
    3x_1 + 2x_2 + 2x_3 & = -1 \\
    -x_2 + 2x_3        & = -1
\end{align*}

Matrix en vector:
$$
    A = \begin{pmatrix}
        2 & 3  & -2 \\
        3 & 2  & 2  \\
        0 & -1 & 2
    \end{pmatrix}, \quad
    b = \begin{pmatrix}
        1 \\ -1 \\ -1
    \end{pmatrix}
$$

Uitgebreide matrix naar rij-echelonvorm:
$$
    \begin{pmatrix}
        2 & 3  & -2 & 1  \\
        3 & 2  & 2  & -1 \\
        0 & -1 & 2  & -1
    \end{pmatrix}
    \rightsquigarrow
    \begin{pmatrix}
        1 & 0 & 2  & 1 \\
        0 & 1 & -2 & 1 \\
        0 & 0 & 0  & 0
    \end{pmatrix}
$$

Neem $x_3 = 0$ dan:
$$
    a = \begin{pmatrix} 1 \\ 1 \\ 0 \end{pmatrix}
$$

Algemene oplossing:
$$
    \left\{ \begin{pmatrix} 1 \\ 1 \\ 0 \end{pmatrix} + t \begin{pmatrix} -2 \\ 2 \\ 1 \end{pmatrix} : t \in \mathbb{R} \right\}
$$

\subsection*{Stelsel 3}
Gegeven stelsel:
\begin{align*}
    2x_1 + 3x_2 - 2x_3 & = 1 \\
    3x_1 + 2x_2 + 2x_3 & = 1 \\
    -x_2 + 2x_3        & = 1
\end{align*}
Matrix en vector:
$$
    A = \begin{pmatrix}
        2 & 3  & -2 \\
        3 & 2  & 2  \\
        0 & -1 & 2
    \end{pmatrix}, \quad
    b = \begin{pmatrix}
        1 \\ 1 \\ 1
    \end{pmatrix}
$$

Uitgebreide matrix naar rij-echelonvorm:
$$
    \begin{pmatrix}
        2 & 3  & -2 & 1 \\
        3 & 2  & 2  & 1 \\
        0 & -1 & 2  & 1
    \end{pmatrix}
    \rightsquigarrow
    \begin{pmatrix}
        1 & 0 & 2  & 0 \\
        0 & 1 & -2 & 0 \\
        0 & 0 & 0  & 1
    \end{pmatrix}
$$

De laatste kolom bevat een spil dus het stelsel is inconsistent. Er is geen oplossing.

\subsection*{Stelsel 4}
Gegeven stelsel:
\begin{align*}
    3x_1 + x_2 + 2x_3 - 2x_4 & = 1 \\
    2x_1 - x_2 + 2x_3        & = 2 \\
    x_1 + x_3                & = 3 \\
    -2x_1 - x_2 - x_3 + x_4  & = 4
\end{align*}

Matrix en vector:
$$
    A = \begin{pmatrix}
        3  & 1  & 2  & -2 \\
        2  & -1 & 2  & 0  \\
        1  & 0  & 1  & 0  \\
        -2 & -1 & -1 & 1
    \end{pmatrix}, \quad
    b = \begin{pmatrix}
        1 \\ 2 \\ 3 \\ 4
    \end{pmatrix}
$$

Uitgebreide matrix naar rij-echelonvorm:
$$
    \begin{pmatrix}
        3  & 1  & 2  & -2 & 1 \\
        2  & -1 & 2  & 0  & 2 \\
        1  & 0  & 1  & 0  & 3 \\
        -2 & -1 & -1 & 1  & 4
    \end{pmatrix}
    \rightsquigarrow
    \begin{pmatrix}
        1 & 0 & 1 & 0 & 3  \\
        0 & 1 & 0 & 0 & -4 \\
        0 & 0 & 0 & 1 & -3 \\
        0 & 0 & 0 & 0 & 0
    \end{pmatrix}
$$

Particuliere oplossing (neem $x_3 = 0$):
$$
    a = \begin{pmatrix} 3 \\ -4 \\ 0 \\ -3 \end{pmatrix}
$$

Kern van $A$ heeft basis $u = \begin{pmatrix} -1 \\ 0 \\ 1 \\ 0 \end{pmatrix}$

Algemene oplossing:
$$
    \left\{ \begin{pmatrix} 3 \\ -4 \\ 0 \\ -3 \end{pmatrix} + t \begin{pmatrix} -1 \\ 0 \\ 1 \\ 0 \end{pmatrix} : t \in \mathbb{R} \right\}
$$

\section*{Opgave 9.1.4}

Gegeven zijn de vectorruimten $V_1$ ($2 \times 2$ matrices) en $V_2$ ($3\times2$ matrices) met bases:
$$
    B = \left\{ \begin{pmatrix} 1 & 0 \\ 0 & 0 \end{pmatrix}, \begin{pmatrix} 0 & 1 \\ 0 & 0 \end{pmatrix}, \begin{pmatrix} 0 & 0 \\ 1 & 0 \end{pmatrix}, \begin{pmatrix} 0 & 0 \\ 0 & 1 \end{pmatrix} \right\}
$$

De lineaire afbeelding $T: V_1 \to V_2$ is gegeven door:
$$
    T(M) = \begin{pmatrix} 3 & 7 \\ -1 & 5 \\ 8 & 2 \end{pmatrix} \cdot M
$$

Bepaal de beelden van de basiselementen:

Voor $B_1 = \begin{pmatrix} 1 & 0 \\ 0 & 0 \end{pmatrix}$:
$$
    T(B_1) = \begin{pmatrix} 3 & 7 \\ -1 & 5 \\ 8 & 2 \end{pmatrix} \cdot \begin{pmatrix} 1 & 0 \\ 0 & 0 \end{pmatrix} = \begin{pmatrix} 3 & 0 \\ -1 & 0 \\ 8 & 0 \end{pmatrix}
$$

Voor $B_2 = \begin{pmatrix} 0 & 1 \\ 0 & 0 \end{pmatrix}$:
$$
    T(B_2) = \begin{pmatrix} 3 & 7 \\ -1 & 5 \\ 8 & 2 \end{pmatrix} \cdot \begin{pmatrix} 0 & 1 \\ 0 & 0 \end{pmatrix} = \begin{pmatrix} 0 & 3 \\ 0 & -1 \\ 0 & 8 \end{pmatrix}
$$

Voor $B_3 = \begin{pmatrix} 0 & 0 \\ 1 & 0 \end{pmatrix}$:
$$
    T(B_3) = \begin{pmatrix} 3 & 7 \\ -1 & 5 \\ 8 & 2 \end{pmatrix} \cdot \begin{pmatrix} 0 & 0 \\ 1 & 0 \end{pmatrix} = \begin{pmatrix} 7 & 0 \\ 5 & 0 \\ 2 & 0 \end{pmatrix}
$$

Voor $B_4 = \begin{pmatrix} 0 & 0 \\ 0 & 1 \end{pmatrix}$:
$$
    T(B_4) = \begin{pmatrix} 3 & 7 \\ -1 & 5 \\ 8 & 2 \end{pmatrix} \cdot \begin{pmatrix} 0 & 0 \\ 0 & 1 \end{pmatrix} = \begin{pmatrix} 0 & 7 \\ 0 & 5 \\ 0 & 2 \end{pmatrix}
$$

De coördinaten ten opzichte van basis $C$ zijn direct af te lezen. De matrix $[T]_B^C$ heeft als kolommen deze coördinaten:
$$
    [T]_B^C = \begin{pmatrix}
        3  & 0  & 7 & 0 \\
        0  & 3  & 0 & 7 \\
        -1 & 0  & 5 & 0 \\
        0  & -1 & 0 & 5 \\
        8  & 0  & 2 & 0 \\
        0  & 8  & 0 & 2
    \end{pmatrix}
$$

\end{document}
