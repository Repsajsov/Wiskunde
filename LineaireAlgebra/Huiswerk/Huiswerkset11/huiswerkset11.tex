\documentclass{article}
\usepackage[utf8]{inputenc}
\usepackage[dutch]{babel}
\usepackage{amsmath, amssymb, amsfonts, amsthm}
\usepackage[margin=2cm]{geometry}
\usepackage{cancel}
\usepackage{enumitem}
\setlength{\parindent}{0pt}
\newcommand{\R}{\mathbb{R}}
\newcommand{\f}[2]{\frac{#1}{#2}}
\begin{document}
{\Large \textbf{Lineaire Algebra}}
\bigskip
\textbf{Jasper Vos} \hfill \textbf{Huiswerkset 11} \hfill \today \\
Studentnr: \emph{s2911159}
\rule{\textwidth}{2pt}
\bigskip
\section*{Opgave 9.4.2}
We hebben de lijn $L \subset \R^2$ met $y = 2x$ en een orthogonale projectie $\pi: \R^2 \rightarrow \R^2$ op $L$.
\begin{enumerate}[label=(\arabic*)]
    \item Lijn $L = \text{span}\{(1,2)\}$, dus richtingsvector $a = (1, 2)$ met $\langle a, a\rangle = 5$.

Bereken projectie van basisvectoren:
\begin{align*}
\pi(e_1) &= \f{\langle e_1, a \rangle}{\langle a, a\rangle}\cdot (1, 2) = \f{1\cdot 1 + 0\cdot 2}{1\cdot 1 + 2\cdot 2}\cdot(1, 2) = \f{1}{5}(1, 2) = \left(\f{1}{5}, \f{2}{5}\right)\\
\pi(e_2) &= \f{\langle e_2, a \rangle}{\langle a, a\rangle}\cdot (1, 2) = \f{0\cdot 1 + 1\cdot 2}{1\cdot 1 + 2\cdot 2}\cdot(1, 2) = \f{2}{5}(1, 2) = \left(\f{2}{5}, \f{4}{5}\right)
\end{align*}

Gebruik de vectoren als kolommen in $[\pi]_B^B$:
\[
\boxed{
[\pi]_B^B = \begin{pmatrix}
\f{1}{5} & \f{2}{5} \\[0.5em]
\f{2}{5} & \f{4}{5}
\end{pmatrix}
}
\]

\item Kies $v_1 = (1,2)$ voor $L$ en $v_2 = (-2, 1)$ voor $L^\bot$. 

Check: $(1,2) \cdot (-2,1) = -2 + 2 = 0$

Basis $C = (v_1, v_2)$.

Projectie behoudt component langs $L$, zet component langs $L^\bot$ op $0$:
\[
\boxed{
[\pi]_C^C = \begin{pmatrix}
1 & 0 \\
0 & 0
\end{pmatrix}
}
\]

\item Basiswisselmatrix $P$ van $C$ naar $B$:
\[P = \begin{pmatrix} 1 & -2 \\ 2 & 1 \end{pmatrix}, \quad 
P^{-1} = \f{1}{5} \begin{pmatrix} 1 & 2 \\ -2 & 1 \end{pmatrix}\]

Bereken $[\pi]_B^B = P \cdot [\pi]_C^C \cdot P^{-1}$:
\begin{align*}
[\pi]_B^B &= \begin{pmatrix} 1 & -2 \\ 2 & 1 \end{pmatrix} 
\begin{pmatrix} 1 & 0 \\ 0 & 0 \end{pmatrix}
\f{1}{5} \begin{pmatrix} 1 & 2 \\ -2 & 1 \end{pmatrix} \\[0.5em]
&= \f{1}{5} \begin{pmatrix} 1 & -2 \\ 2 & 1 \end{pmatrix}
\begin{pmatrix} 1 & 2 \\ 0 & 0 \end{pmatrix}
= \f{1}{5} \begin{pmatrix} 1 & 2 \\ 2 & 4 \end{pmatrix}
\end{align*}

\[
\boxed{
[\pi]_B^B = \begin{pmatrix}
\f{1}{5} & \f{2}{5} \\[0.5em]
\f{2}{5} & \f{4}{5}
\end{pmatrix}
}
\]

Komt overeen met (1).
\end{enumerate}

\pagebreak
\section*{Opgave 10.1.2}

\subsection*{Te bewijzen}
Voor bovendriehoeksmatrix $A$ geldt: $\det(A) = \prod_{i=1}^{n} a_{ii}$

\subsection*{Bewijs via inductie}

Bovendriehoeksmatrix heeft vorm:
\[
A = \begin{pmatrix}
a_{11} & * & \dots  & * \\
0      & a_{22} & \dots  & * \\
\vdots & \vdots & \ddots & \vdots \\
0      & 0      & \dots  & a_{nn}
\end{pmatrix}
\]

\textbf{Basisstap} ($n=1$): $A = (a_{11})$, dus $\det(A) = a_{11}$

\textbf{Inductiestap}: Stel het geldt voor $(n-1) \times (n-1)$ matrices.

Laplace-expansie naar eerste kolom:
\[\det(A) = a_{11} \cdot \det(A_{11}) + 0 + \ldots + 0\]

Submatrix $A_{11}$ is ook bovendriehoeks met diagonaal $(a_{22}, \ldots, a_{nn})$.

Via inductiehypothese:
\[\det(A_{11}) = a_{22} \cdot a_{33} \cdot \ldots \cdot a_{nn}\]

Dus:
\[\det(A) = a_{11} \cdot (a_{22} \cdot \ldots \cdot a_{nn}) = \prod_{i=1}^{n} a_{ii}\]

\subsection*{Conclusie}
\[
\boxed{\det(A) = a_{11} \cdot a_{22} \cdot \ldots \cdot a_{nn}}
\]

\end{document}