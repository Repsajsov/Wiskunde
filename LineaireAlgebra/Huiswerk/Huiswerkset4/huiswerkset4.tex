\documentclass{article}
\usepackage[utf8]{inputenc}
\usepackage[dutch]{babel}
\usepackage{amsmath, amssymb, amsfonts, amsthm}
\usepackage[margin=2cm]{geometry}
\usepackage{cancel}
\usepackage{enumitem}

\setlength{\parindent}{0pt}

\newcommand{\R}{\mathbb{R}}
\newcommand{\en}{\text{ en }}
\newcommand{\dan}{\text{ dan }}
\newcommand{\geldt}{\text{ geldt }}
\newcommand{\met}{\text{ met }}

\begin{document}

{\Large \textbf{Lineaire Algebra Huiswerk}}

\bigskip

\textbf{Jasper Vos} \hfill \textbf{Huiswerkset 3} \hfill \today \\
Studentnr: \emph{s2911159}

\rule{\textwidth}{2pt}

\bigskip

\section*{Opgave 3.4.6}
\begin{proof}
    Laat $F$ een willekeurig lichaam zijn voor een willekeurige
    vectorruimte $V$ en laat met
    tegenvoorbeeld zien dat
    $L(I) \cap L(J) = L(I \cap J)$ niet waar kan zijn.
    Te bewijzen met een tegenvoorbeeld.

    \subsection*{Bekijk $L(I) \cap L(J)$:}
    Neem $I \in \R^2 \met I = \{(1,1)\}, \en J \in \R^2 \met J = \{(2,2)\}$ dan:
    \[L(I) =
        \{\lambda(1, 1) : \lambda \in F\} \quad (\text{Voor willekeurige $\lambda$})\]
    \begin{center}
        en
    \end{center}
    \[L(J) =
        \{\mu (2,2) : \mu \in F\} \quad (\text{Voor willekeurige $\mu$})\]
    \begin{center}
        daaruit volgt dus $L(I) \cap L(J)$:
    \end{center}
    \[ L(I) \cap L(J) = \{\lambda(1, 1) : \lambda \in F\} \quad (\text{Vool willekeurige $\lambda$})\]
    Dit komt voort omdat $L(I), \en L(J)$ dezelfde lijn opstellen,
    want $(2, 2)\in J$ is een opgeschaalde variant van $(1,1) \in I$.

    \subsection*{Bekijk $L(I \cap J)$:}
    Neem $I = \{1,1\}, \en J = \{2,2\}$, dan:
    \[ I \cap J = \{(1, 1)\} \cap \{(2, 2)\} = \emptyset\]
    Als we hier het lineaire omhulsel van nemen dan
    krijgen we $L(\emptyset) = \{0\}$.

    \bigskip

    \emph{Conclusie:}

    Dit betekent dus dat $L(I) \cap L(J)$ het lineaire omhulsel van $(1,1)$ is,
    en voor $L(I \cap J)$ geldt alleen de nulvector. Dit betekent dus dat:
    \[L(I) \cap L(J) \neq L(I \cap J)\]

\end{proof}
\newpage
\section*{Opgave 3.4.7(2)}
\begin{proof}
    Laat $V$ een verzameling zijn van
    alle oneven functies van $\R \rightarrow \R$,
    waarbij $V$ een deelruimte is op $\R^{\R}$.

    \subsection*{Het nulelement}
    Te bewijzen: $V$ bevat het nulelement,
    wat in dit geval de nulfunctie $f_0$ is.

    Zij $f_0:\R \rightarrow \{0\} \ f(x) = 0$, dan
    geldt voor alle $x \in \R$ dat:
    \[f(-x) = 0 \en -f(x) = -0 = 0\]
    Dit betekent dus dat $f_0 \in V$.

    \subsection*{Gesloten onder optelling}
    Te bewijzen: $f, g \in V \dan f + g \in V$.

    Zij $f, g \in V$ willekeurig gegeven dan
    geldt voor alle $x \in \R$ dat:
    \begin{align*}
        (f + g)(-x) & = f(-x) + g(-x)  \quad (\text{Definitie optellen functies}) \\
                    & = -f(x) + -g(x) \quad (\text{Eigenschap oneven functie})    \\
                    & = -(f(x) + g(x)) \quad (\text{Distributiviteit in $\R$})    \\
                    & = -((f + g)(x)) \quad (\text{Definitie optellen functies})
    \end{align*}
    Hieruit volgt dus dat voor alle
    $x \in \R \geldt
        (f + g)(-x) = -(f + g)(x)$, en dus
    $f + g \in V$.

    \subsection*{Gesloten onder scalaire vermedigvuldiging}
    Te bewijzen: $f \in V \en \lambda \in \R \dan \lambda f \in V$.

    Zij $f \in V \en \lambda \in \R$ dan:
    \[\lambda f(-x) = -\lambda f(x) \quad (\text{Eigenschap oneven functie})\]

    Dus voor elke $x \in \R$ geldt dat $\lambda f(-x) = -\lambda f(x)$, en dus $\lambda f \in V$.

    \bigskip

    \emph{Conclusie:}

    Door te bewijzen dat $V$ voldoet aan het nulelement, optelling en scalaire vermedigvuldiging
    hebben we bewezen dat $V$ een deelruimte is van $\R^{\R}$.
\end{proof}
\end{document}
