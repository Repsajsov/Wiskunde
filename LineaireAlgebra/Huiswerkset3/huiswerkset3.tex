\documentclass{article}
\usepackage[utf8]{inputenc}
\usepackage[dutch]{babel}
\usepackage{amsmath, amssymb, amsfonts, amsthm}
\usepackage[margin=2cm]{geometry}
\usepackage{cancel}
\usepackage{enumitem}

\setlength{\parindent}{0pt}

\newcommand{\R}{\mathbb{R}}


\begin{document}

{\Large \textbf{Lineaire Algebra Huiswerk}}

\bigskip

\textbf{Jasper Vos} \hfill \textbf{Huiswerkset 3} \hfill \today \\
Studentnr: \emph{s2911159}

\rule{\textwidth}{2pt}

\bigskip

\section*{Opgave 2.2.9 (4)}
\subsection*{Nulelement}
We stellen de nulfunctie op namelijk $f: \R \rightarrow \R$ met $f_0(x) = 0$, dan $f_0(3) = 0$ en dus $f_0 \in V$.
\subsection*{Optelling}
Zij $f_1, f_2$ willekeurig gekozen in $V$, en laat $g = f_1 + f_2$, dan:
\begin{align*}
    g(3) & = (f_1 + f_2)(3)  \\
         & = f_1(3) + f_2(3) \\
         & = 0 + 0           \\
         & = \boxed{0}
\end{align*}
Hieruit volgt dus dat $g(3) = 0$ en dus $g \in V$.
\subsection*{Vermedigvuldiging}
Zij $\lambda \in \R$ en $f$ willekeurig gekozen in $V$ dan:
\begin{align*}
    \lambda f(3) & = \lambda (0) \\
                 & = \boxed{0}
\end{align*}
\subsection*{Axioma's}
\subsubsection*{Additieve commutativiteit:}
Te bewijzen: Voor alle $f, g \in V$ geldt $f(x) + g(x) = g(x) + f(x)$.

Merk op dat $f(x), g(x) \in \R$, en voor $\R$ geldt dat termen commutatief zijn.
Dus $f(x) + g(x) = g(x) + f(x)$.
\subsubsection*{Additieve associativiteit:}
Te bewijzen: Voor alle $f, g, h \in V$ geldt dat $(f+(g+h))(x) = ((f + g) + h)(x)$.
\begin{align*}
    (f + (g + h))(x) & = f(x) + (g + h)(x)  \quad (f(x), g(x), h(x) \in \R \text{ en dus associatief}) \\
                     & = f(x) + g(x) + h(x)                                                            \\
                     & = (f + g)(x) + h(x)                                                             \\
                     & = ((f + g)+h)(x)
\end{align*}
\subsubsection*{Neutraal element:}




\end{document}