\documentclass{article}
\usepackage[utf8]{inputenc}
\usepackage[dutch]{babel}
\usepackage{amsmath, amssymb, amsfonts, amsthm}
\usepackage[margin=2cm]{geometry}
\usepackage{cancel}
\usepackage{enumitem}

\setlength{\parindent}{0pt}

\newcommand{\R}{\mathbb{R}}


\begin{document}

{\Large \textbf{Lineaire Algebra Huiswerk}}

\bigskip

\textbf{Jasper Vos} \hfill \textbf{Huiswerkset 3} \hfill \today \\
Studentnr: \emph{s2911159}

\rule{\textwidth}{2pt}

\bigskip

\section*{Opgave 2.2.9 (4)}
\subsection*{Nulelement}
We stellen de nulfunctie op namelijk $f: \R \rightarrow \R$ met $f_0(x) = 0$, dan $f_0(3) = 0$ en dus $f_0 \in V$.
\subsection*{Optelling}
Zij $f_1, f_2$ willekeurig gekozen in $V$, en laat $g = f_1 + f_2$, dan:
\begin{align*}
    g(3) & = (f_1 + f_2)(3)  \\
         & = f_1(3) + f_2(3) \\
         & = 0 + 0           \\
         & = \boxed{0}
\end{align*}
Hieruit volgt dus dat $g(3) = 0$ en dus $g \in V$.
\subsection*{Vermedigvuldiging}
Zij $\lambda \in \R$ en $f$ willekeurig gekozen in $V$ dan:
\begin{align*}
    \lambda f(3) & = \lambda (0) \\
                 & = \boxed{0}
\end{align*}
\subsection*{Axioma's}
\subsubsection*{1. Additieve commutativiteit:}
Te bewijzen: Voor alle $f, g \in V$ geldt $f(x) + g(x) = g(x) + f(x)$.
\begin{proof}
    Merk op dat $f(x), g(x) \in \R$, en voor $\R$ geldt dat termen commutatief zijn.
    Dus $f(x) + g(x) = g(x) + f(x)$.
\end{proof}
\subsubsection*{2. Additieve associativiteit:}
Te bewijzen: Voor alle $f, g, h \in V$ geldt dat $(f+(g+h))(x) = ((f + g) + h)(x)$.
\begin{proof}
    \begin{align*}
        (f + (g + h))(x) & = f(x) + (g + h)(x)  \quad (f(x), g(x), h(x) \in \R \text{ en dus associatief}) \\
                         & = f(x) + g(x) + h(x)                                                            \\
                         & = (f + g)(x) + h(x)                                                             \\
                         & = ((f + g)+h)(x)
    \end{align*}
\end{proof}
Dus $V$ is associatief.
\subsubsection*{3. Neutraal element:}
Te bewijzen: Voor alle $f \in V$ geldt $f + f_0 = f$
\begin{proof}
    Merk op dat $f_0(x) = 0$ en dus $f(x) + f_0(x) = f(x) + 0 = f(x)$.
\end{proof}
\subsubsection*{4. Bestaan van negatieven:}
Te bewijzen: Voor alle $f \in V$ bestaat er een $f' \in V$ zodanig dat $f + f' = f_0$.
\begin{proof}
    We weten dat $f(x) \in \R$ ligt en voor $\R$ geldt dat elk element een additieve inverse heeft.
    Dus neem $f'(x) = -f(x)$ dan $f(x) + f'(x) = f_0(x)$.
\end{proof}
\subsubsection*{5. Scalaire Vermedigvuldiging is associatief:}
Te bewijzen: Voor alle $\lambda, \mu \in \R$ en $f \in V$ geldt: $(\lambda \odot (\mu \odot f))(x) = ((\lambda \odot \mu) \odot f)(x)$.
\begin{proof}
    We weten dat $\lambda, \mu$ en $f(x)$ allemaal in $\R$ liggen en voor
    $\R$ geldt dat vermedigvuldiging associatief is dus:
    \begin{align*}
        (\lambda \odot (\mu \odot f))(x) & =  \lambda (\mu \odot f)(x)        \\
                                         & = \lambda(\mu f(x))                \\
                                         & = (\lambda \mu)f(x)                \\
                                         & = ((\lambda \odot \mu) \odot f)(x)
    \end{align*}
\end{proof}
\subsubsection*{6. Vermedigvuldiging met $1$ doet niks:}
Te bewijzen: Voor alle $f \in V$ geldt $1 \odot f = f$.
\begin{proof}
    Merk op $f(x) \in \R$, dan $1 \odot f(x) = f(x)$.
\end{proof}

\subsubsection*{7. Distributiviteit I:}
Te bewijzen: Voor alle $\lambda \in \R$ en voor alle $f, g \in V$ geldt $(\lambda \odot (f + g))(x) = (\lambda \odot f)(x) \oplus (\lambda \odot g)(x)$.
\begin{proof}
    Merk op dat $f(x) \in \R$ en voor $\R$ geldt dat het distributief is, en dus:
    \begin{align*}
        (\lambda \odot (f + g))(x) & = \lambda \odot (f(x) + g(x))                      \\
                                   & = (\lambda \odot f)(x) \oplus (\lambda \odot g)(x)
    \end{align*}
\end{proof}
\subsubsection*{8. Distributiviteit II:}
Te bewijzen: Voor alle $\lambda, \mu \in \R$ en voor alle $f \in V$ geldt $((\lambda + \mu) \odot f)(x) = (\lambda \odot f)(x) \oplus (\mu \odot f)(x)$.
\begin{proof}
    Merk op dat $f(x) \in \R$ en voor $\R$ geldt dat het distributief is, en dus:
    \begin{align*}
        ((\lambda \odot \mu) + f)(x) & = (\lambda \odot \mu)f(x)                      \\
                                     & = (\lambda \odot f)(x) \oplus (\mu \odot f)(x)
    \end{align*}
\end{proof}

$V$ heeft een Nulelement, optelling, vermedigvuldiging, en voldoet aan de acht axioma's
waardoor we kunnen zeggen at $V$ een vectorruimte is.



\end{document}